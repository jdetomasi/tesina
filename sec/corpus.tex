\chapter{Análisis de requerimientos}
\label{cap:corpus}

El primer paso en la construcción de cualquier sistema de software, incluyendo los sistemas de generación de lenguaje natural, es realizar un análisis de requerimientos y a partir de estos generar una especificación inicial para el sistema. 

Para el análisis de requerimientos, seguiremos el enfoque sugerido por Reiter y Dale~\cite{reiter_dale} en el cual se propone realizar un \emph{corpus} de textos de ejemplo y a partir de ellos obtener los requerimientos para nuestro trabajo.

\section{Corpus de descripciones}                 

Según la definición de la RAE~\cite{dicrae}, un \emph{corpus} es un \emph{``conjunto lo más extenso y ordenado posible de datos o textos científicos, literarios, etc., que pueden servir de base a una investigación''}. En particular, el \emph{corpus} utilizado para la construcción de un sistema de NLG estará formado por un conjunto de ejemplos de datos de entrada junto a la correspondiente salida (texto en lenguaje natural) para cada uno de estos. En nuestro caso, la entrada será un grupo de clases de prueba (en lenguaje Z) junto a las designaciones correspondientes a la especificación testeada, mientras que la salida estará formada por descripciones en lenguaje natural de las clases de prueba antes mencionadas. En lo posible, el \emph{corpus} de textos deberá cubrir todo el rango de textos que esperan ser producidos por el sistema de NLG; éste debería cubrir los casos más frecuentes, así como los casos mas inusuales que se pudieran ocurrir.

%~\footnote{Una vez recolectada una colección inicial de textos de ejemplo, es posible que sea necesario aplicar algunas modificaciones sobre esta. Esto se puede deber a que algunos de los textos recolectados resulten técnicamente implosibles de generar y sea necesarios removerlos del corpus, que existan diferentes variantes correspondientes a la misma entrada de texto y requiramos resolver posibles conflictos, textos que puedan ser mejorados y puedan ser modificados, etc. Llamaremos \emph{corpus} objetivo al resultado de aplicar al \emph{corpus} inicial todas las modificaciones antes mencionadas (de ser necesarias).}
Siguiendo la metodología propuesta por Reiter y Dale, para construir el \emph{corpus} que utilizaremos a lo largo de todo el trabajo deberemos elaborar, en primera instancia, un \emph{corpus inicial} y luego, de ser necesario, deberemos trabajar el mismo a fin de confeccionar un \emph{corpus objetivo} que será con el que finalmente utilizaremos. Para construir un \emph{corpus inicial} deberemos recolectar un conjunto lo suficientemente amplio de ejemplos que nos permita caracterizar la variedad de textos que deseamos generar y luego una persona capacitada (en nuestro caso será una persona capaz de leer Z) deberá describir en lenguaje natural los ejemplos antes mencionados. Esta colección de ejemplos y descripciones de los mismos constituirá nuestro \emph{corpus inicial}. Es posible que esta recopilación requiera algunas modificaciones, por ejemplo: podría ocurrir que alguno de los textos resulten técnicamente imposibles de generar y sea necesarios removerlos del corpus, también podrían existir diferentes variantes correspondientes a la misma entrada de texto y deberíamos resolver los posibles conflictos, etc. Llamaremos, finalmente, \emph{corpus objetivo} al resultado de aplicar las modificaciones necesarias al \emph{corpus inicial}. Este \emph{corpus objetivo} es el que utilizaremos para sustentar muchas de las decisiones que deberemos tomar a lo largo de este trabajo. Además nos será de utilidad para realizar una evaluación de nuestro sistema una vez desarrollado, comparando los textos generados por nuestra implementación con las descripciones del \emph{corpus} realizadas por la persona especializada.

En el apéndice~\ref{ape:corpus} podemos encontrar los textos incluidos en el \emph{corpus} utilizado para este trabajo. Para elaborar el mismo recolectamos una colección de clases de prueba generados con \emph{Fastest} a partir de distintas especificaciones y luego escribimos manualmente cada una de las descripciones para las mismas. En este proceso, intentamos abarcar todo el rango de textos que esperamos que nuestro sistema sea capaz de producir, para esto, tuvimos en cuenta incluir una gran variedad de clases de pruebas de modo que cubran todas las expresiones de Z contempladas dentro del alcance de este trabajo, considerando también las posibles combinaciones de estas expresiones. Trabajamos también con especificaciones sobre distintos dominios de aplicación a fin de lograr un \emph{corpus} que nos sea de utilidad para dar con una solución independiente del dominio de aplicación. En total incluimos clases de prueba generadas con \emph{Fastest} para 10 especificaciones distintas; del total se escogieron las más significativas para describir y se ignoraron aquellas que contenían algún operador no considerado dentro del alcance de este trabajo.


En la Figura~\ref{fig:ej_corpus} podemos observar uno de los ejemplos incluidos en el \emph{corpus} de descripciones utilizado para este trabajo. En este caso tenemos la clase de prueba \emph{LookUp\_SP\_1} (generada a partir de la especificación introducida previamente) y las designaciones correspondientes a la especificación en cuestión serán la entrada del sistema de NLG~\footnote{Estrictamente hablando, todas las designaciones de la especificación deberían formar parte de la entrada del sistema de NLG. Para simplificar el ejemplo incluimos sólo las designaciones relevantes a fin de simplificar el ejemplo.}, teniendo como salida la descripción en lenguaje natural presente en el ejemplo.

\begin{figure}[H]
\begin{itemize}
\item \emph{Clase de prueba para operación LookUp}\\
\begin{schema}{LookUp\_ SP\_ 1}\\
  LookUp\_ VIS 
  \where
    s? \in \dom st \\
    \dom st = \{ s? \}
\end{schema}

\item \emph{Designaciones SymbolTable}\\

\begin{itemize}[label={--}]
  \item símbolo a buscar $\approx s?$
  \item símbolos cargados en la tabla $\approx \dom st$
\end{itemize}

\bigskip
\item \emph{Descripción en lenguaje natural para LookUp\_SP\_1}\\

\fbox{\begin{minipage}{13 cm}
  \textbf{LookUp\_SP\_1:} Se busca un símbolo en la tabla.  
    \begin{itemize}
     \item{Cuando:}
     \begin{itemize}
  	    \item{El símbolo a buscar pertenece a los símbolos cargados en la tabla de símbolos.}
  	    \item{El símbolo a buscar es el único elemento del conjunto formado por los símbolos cargados en la tabla de símbolos.}   
     \end{itemize}
    \end{itemize}
\end{minipage}}
\end{itemize}
\caption{Corpus de textos.}
\label{fig:ej_corpus}
\end{figure}

\section{Análisis del corpus}
\label{sec:corpus_analisis}

En lo que queda de este capítulo nos encargaremos de analizar los ejemplos presentes en el \emph{corpus} a fin de extraer los requerimientos para nuestro sistema.

Comencemos observando las clases de prueba generadas por \emph{Fastest}. Podemos notar que estas se encuentran formadas siempre por conjunciones de predicados atómicos y que a cada uno de estos le corresponde exactamente una oración en lenguaje natural dentro del texto final. Además, estas oraciones dependen exclusivamente del predicado que describen, es decir, estas no contienen información sobre otros predicados ni hacen referencia a otras frases generadas. Por lo tanto, la tarea de describir el cuerpo de del esquema Z de la clase de prueba, se puede reducir básicamente a \emph{verbalizar} individualmente cada uno de estos predicados. Esta tarea, la \emph{verbalización} de predicados Z, resultará el desafío principal de este trabajo. Definir con precisión los detalles será de vital importancia para el desarrollo de las etapas de \emph{microplanning} y \emph{realization} de nuestro sistema de NLG (capítulo~\ref{cap:microplanning} y \ref{cap:realization}).

Por otro lado, veamos la estructura de las descripciones de ejemplo. En estos podemos ver que todas las descripciones de las clases de prueba poseen la misma estructura. En todas se comienza por el nombre de la clase de prueba, seguido por un pequeño detalle de la operación a testear y luego una lista de oraciones encargadas de describir cada uno de los predicados presentes en el cuerpo de la clase de prueba. Conocer la estructura del texto a generar nos será de utilidad para la elaboración del \emph{document plan} (capítulo ~\ref{cap:document_planning}).

Como ya mencionamos, la verbalización de expresiones Z resultará una de las tareas mas importantes para nuestro sistema de NLG. En lo que resta de esta sección nos concentraremos especialmente en analizar y especificar la misma.

Podemos notar dos aspectos fundamentales en relación a la \emph{verbalización} de expresiones Z. En primer lugar, podemos ver que hay textos que se repiten (con pequeñas variantes, que analizaremos posteriormente) independientemente del dominio de aplicación y estos surgen a raíz de los operadores de Z presentes en los predicados que se describen. Por otro lado, podemos observar el rol fundamental de las designaciones que posibilitan la introducción de texto dependiente del dominio de aplicación en las descripciones y a su vez estos se encuentran combinados con las frases antes mencionadas correspondientes a la descripción de operadores de Z. Consideremos, por ejemplo, las siguientes dos expresiones y sus respectivas descripciones en lenguaje natural, una pertenece al ejemplo de la figura~\ref{fig:ej_corpus} y la otra describe un predicado que forma parte de una clase de prueba para la especificación de un sistema bancario:

\bigskip
\begin{enumerate}
	\item $s? \in \dom$ $\rightarrow$ \emph{``El símbolo a buscar \textbf{pertenece a} los símbolos cargados en la tabla de símbolos.''}
	\item $s? \in \dom$ $\rightarrow$ \emph{``El número de cuenta \textbf{pertenece a} los números de cuenta cargados en el banco.''}
\end{enumerate}

\bigskip
Como vemos, el texto \emph{``\textbf{pertenece a}''} aparece en ambas descripciones como resultado de verbalizar el operador $\in$, diferenciándose ambas descripciones en el texto que antecede y precede al anterior. Por otro lado, estos últimos, bien podrían ser las verbalizaciones de las expresiones que se encuentran a la derecha e izquierda del operador $\in$ respectivamente, por lo que podríamos pensar en una verbalización recursiva sobre la estructura de las expresiones Z. Es posible encontrar estos tipos de patrones en todas las descripciones presentes en el \emph{corpus}. Esto resulta un buen punto de partida para intentar especificar nuestra tarea de verbalización. En un primer intento, entonces, definiremos la tarea de verbalización en base a los operadores que la componen. Especificaremos mediante esta tarea mediante una función que tomará como entrada una expresión Z y devolverá una descripción en lenguaje natural para la misma. Ésta será recursiva sobre la estructura de las expresiones Z y tendrá tantos casos en su definición como operadores contemplados dentro del alcance de este trabajo (además de las posibles combinaciones de los mismos que puedan resultar de interés y requieran un tratamiento particular). En base al ejemplo anterior, podríamos proponer la siguiente definición para verbalizar el operador $\in$:

{
\begin{figure}[H]
\center
$verb'(x \in y) = verb'(x) + \text{\emph{``pertenece a''}} + verb'(y)$ \footnotemark
\end{figure}
\footnotetext{El nombre de la función se encuentra primado intencionalmente. Más adelante presentaremos la definición final para la función de verbalización que hará uso de verb' como función auxiliar.}
}

Como mencionamos anteriormente para verbalizar términos compuestos, necesitemos verbalizar recursivamente las partes que componen los mismos. En el ejemplo anterior necesitaríamos conocer las verbalizaciones de las expresiones $x$ e $y$ para poder obtener una verbalización para $x \in y$.

Retomando el ejemplo de la figura~\ref{fig:ej_corpus}, nos concentraremos en verbalizar la primera expresión de la clase de prueba:

\begin{figure}[H]
\center
$s? \in \dom st$
\end{figure}

En este caso, según la verbalización propuesta anteriormente, podríamos verbalizar la expresión como:

\begin{figure}[H]
\center
$verb'(s? \in \dom st) \rightarrow verb'(s?) + \text{\emph{``pertenece a''}} + verb'(\dom st)$
\end{figure}

Teniendo que verbalizar $s?$ y $\dom st$. En este caso, ambas expresiones se encuentran designadas, lo que sería de gran ayuda para nuestra tarea de verbalización ya que en estas situaciones podremos construir la descripción en base al texto presente en la designación y no intentar verbalizar la expresión en base al término Z que la compone. En el ejemplo anterior, no deberíamos intentar verbalizar $\dom st$ como:

\begin{figure}[H]
\center
$verb'(\dom st) \rightarrow \text{\emph{``el dominio''}} + verb(st)$
\end{figure}

\noindent
ya que estaríamos perdiendo información valiosa para nuestras descripciones contenida en las designaciones ya que las designaciones son nuestra única forma de introducir texto referente al dominio de aplicación en las descripciones. En general, para verbalizar una expresión designada, deberemos utilizar el texto presente en las designaciones (en algunos casos , como veremos mas adelante, con algunas pequeñas modificaciones, para asegurarnos ciertas cuestiones de concordancia gramatical). Será un requerimiento, entonces, para nuestra tarea de verbalización contemplar en primera instancia si la expresión a describir se encuentra designada antes de intentar describirla en base a los operadores que la componen; de estar designada, deberemos construir una descripción en base a su designación. 

Como mencionamos previamente, el nombre de la función anterior (\emph{verb'}) fue primado intencionalmente a fin de utilizar ésta función como auxiliar para la definición de final de la tarea de verbalización. En la figura~\ref{fig:def-verb} introduciremos una nueva definición para ésta tarea donde consideraremos la designación de la expresión a verbalizar (si es que se encuentra designada) antes de intentar describir la misma en base a los operadores que la constituyen.

\begin{figure}[H]
\begin{align*}
\nlgfun{verb($exp$)} = & \text{\textbf{ if }} \nlgfun{esta\_designada($exp$)} \\
 & \text{\textbf{ then }} \nlgfun{designacion($exp$)} \\
 & \text{\textbf{ else }} \nlgfun{verb'($exp$)} \\
\end{align*}
\caption{Definición verbalización.}
\label{fig:def-verb}
\end{figure}


Como podemos ver en la figura anterior, haremos uso de \emph{verb'} que será la responsable de generar las verbalizaciones para las expresiones de Z en base a un conjunto de reglas dependientes del operador a describir, como la introducida anteriormente para el caso del operador $\in$. Por otro lado, abstraeremos por medio de \emph{esta\_designada()} la tarea de verificar si una expresión se encuentra designada y por medio de la función \emph{designacion()} la tarea de generar una descripción en base a la designación de la expresión a describir. En lo que queda de este capítulo presentaremos un conjunto de reglas (similares a la introducida anteriormente para la verbalización del operador $\in$) para la definición de \emph{verb'} necesarios para la generación de descripciones en base a los distintos operadores de Z contemplados dentro del alcance del trabajo. Además entraremos más en detalle sobre las tarea de verificar si una expresión se encuentra designada y la generación de un descripción en lenguaje natural para una expresión ya designada. Comenzaremos, a continuación, por completar la definición de \emph{verb'}, basándonos en los textos presentes en el \emph{corpus}, incluyendo los casos para todos los operadores considerados dentro del alcance del trabajo, así como las combinaciones más relevantes de los mismos. %Cabe aclarar, que en este caso, tendremos que contemplar si la designación correspondiente es una designación parametrizada. De no ser el caso, el resultado sería exactamente el texto que forma parte de la designación, del contrario, habrá primero que verbalizar el parámetro de la designación, teniendo luego que construir la descripción final a partir del texto de la designación parametrizada y la verbalización del parámetro.
%TODO~\footnote{TODO: aclarar que suponemos que el parámetro debe estar designado?}

\begin{minted}[escapeinside=@@]{haskell}

verb' (@$\{exp_1\} = exp_2$@)          = verb(@$exp_1$@) + 
                                "es el unico elemento de" + 
                                verb(@$exp_2$@)

verb' (@$exp_1 = \{\}$@)              = "no hay ningun elemento en" + 
                                 verb(@$exp_1$@) 

verb' (@$exp_1 \cap exp_2 = \{\}$@)        = verb(@$exp_1$@)  +  
                                "y" +  
                                verb(@$exp_2$@)  +  
                                "no tienen ningun elemento en comun"

verb' (@$exp_1 \cap \{exp_2\} = \{\}$@)      = verb(@$exp_2$@) +  
                                "no pertenece a" +  
                                verb(@$exp_1$@) 

verb' (@$exp_1 = exp_2$@)
     | (num == S) = verb(@$exp_1$@) + "es igual a" + verb(@$exp_2$@) 
     | (num == P) = verb(@$exp_1$@) + "son iguales a" + verb(@$exp_2$@) 
     where num = numero(@$exp_1$@)

verb' (@$exp_1 \neq \{\}$@)              = "existe al menos un elemento en" +  
                                 verb(@$exp_1$@) 

verb' (@$exp1 \cap exp2 \neq \{\}$@)        = verb(@$exp_1$@)  +  
                                 "y" +  
                                 verb(@$exp_2$@) +  
                                 "tienen al menos un elemento en comun" 

verb' (@$exp_1 \neq exp_2$@)
     | (num == S) = verb(@$exp_1$@) + "no es igual a" + verb(@$exp_2$@) 
     | (num == P) = verb(@$exp_1$@) + "no son iguales a" + verb(@$exp_2$@) 
     where num = numero(@$exp_1$@)

verb' (@$exp_1 < exp_2$@)             = verb(@$exp_1$@) +  
                                 "es menor a" +  
                                 verb(@$exp_2$@) 

verb' (@$exp_1 < 0$@)
     | (gen == M) = verb(@$exp_1$@) + "es negativo" 
     | (gen == F) = verb(@$exp_1$@) + "es negativa" 
     where gen = genero(@$exp_1$@)
                                 
verb' (@$exp_1 \leq exp_2$@)            = verb(@$exp_1$@) +  
                                "es menor o igual a" +  
                                verb(@$exp_2$@) 

verb' (@$exp_1 > exp_2$@)            = verb(@$exp_1$@) +  
                                "es mayor a" +  
                                verb(@$exp_2$@) 

verb' (@$exp_1 > 0$@)  =
     | (gen == M) = verb(@$exp_1$@) + "es positivo" 
     | (gen == F) = verb(@$exp_1$@) + "es positiva" 
     where gen = genero(@$exp_1$@)

verb' (@$exp_1 \geq exp_2$@)            = verb(@$exp_1$@) +  
                                "es mayor o igual a" +  
                                verb(@$exp_2$@) 

verb' (@$exp_1 \in exp_2$@)
     | (num == S) = verb(@$exp_1$@) + "pertenece a" + verb(@$exp_2$@) 
     | (num == P) = verb(@$exp_1$@) + "pertenecen a" + verb(@$exp_2$@) 
     where num = numero(@$exp_1$@)

verb' (@$exp_1 \notin exp_2$@)  = 
     | (num == S) = verb(@$exp_1$@) + "no pertenece a" + verb(@$exp_2$@) 
     | (num == P) = verb(@$exp_1$@) + "no pertenecen a" + verb(@$exp_2$@) 
     where num = numero(@$exp_1$@)

verb' (@$exp_1 \subset exp_2$@)
     | (gen == M && num == S) = verb(@$exp_1$@) + "esta incluido en" + verb(@$exp_2$@) 
     | (gen == M && num == P) = verb(@$exp_1$@) + "estan incluidos en" + verb(@$exp_2$@) 
     | (gen == F && num == S) = verb(@$exp_1$@) + "esta incluida en" + verb(@$exp_2$@) 
     | (gen == F && num == P) = verb(@$exp_1$@) + "estan incluidas en" + verb(@$exp_2$@) 
     where gen = genero(@$exp_1$@)
           num = numero(@$exp_1$@)

verb' (@$\{exp_1, ... , exp_n\} \subset exp_m$@)  = verb(@$exp_1$@) +  
                                ", ... , y" +  
                                verb(@$exp_n$@) +  
                                "pertenecen a" +
                                verb(@$exp_m$@)

verb' (@$exp_1 \not\subset exp_2$@)  = 
     | (gen == M && num == S) = verb(@$exp_1$@) + "no esta incluido en" + verb(@$exp_2$@) 
     | (gen == M && num == P) = verb(@$exp_1$@) + "no estan incluidos en" + verb(@$exp_2$@) 
     | (gen == F && num == S) = verb(@$exp_1$@) + "no esta incluida en" + verb(@$exp_2$@) 
     | (gen == F && num == P) = verb(@$exp_1$@) + "no estan incluidas en" + verb(@$exp_2$@) 
     where gen = genero(@$exp_1$@)
           num = numero(@$exp_1$@)

verb' (@$exp_1 \subseteq exp_2$@)             = verb(@$exp_1$@) +  
                                "esta incluido o es igual a" +  
                                verb(@$exp_2$@) 

verb' (@$\{exp_1, ... , exp_n\} \subseteq exp_m$@)  = verb(@$exp_1$@) +  
                                ", ... , y" +  
                                verb(@$exp_n$@) +  
                                "pertenecen a" +
                                verb(@$exp_m$@)

verb' (@$exp_1 \not\subseteq exp_2$@)            = "existe al menos un elemento en" +  
                                verb(@$exp_1$@) +  
                                "que no se esta en" +  
                                verb(@$exp_2$@) 

verb' (@$exp_1 \mapsto exp_2$@)           = "el par ordenado formado por:" +  
                                verb(@$exp_1$@) +  
                                "y" +  
                                verb(@$exp_2$@) 

verb' (@$\{\}$@)                    = "el conjunto vacio" 

verb' (@$\{exp_1\}$@)                = "el conjunto formado por"  +  
                                verb(@$exp_1$@) 

verb' (@$\{exp_1, ... ,exp_n\}$@)         = "el conjunto formado por" +  
                                verb(@$exp_1$@) +  
                                ", ... , y" +  
                                verb(@$exp_n$@) 

verb' (@$exp_1 \cup exp_2$@)            = "elementos en" +  
                                verb(@$exp_1$@) +  
                                "y en" +  
                                verb(@$exp_2$@) 

verb' (@$exp_1 \cap exp_2$@)             = "elementos en" +  
                                verb(@$exp_1$@) +  
                                "que tambien se encuentren en" +  
                                verb(@$exp_2$@) 

verb' (@$f~exp_1$@)                = verb(@$f$@) +  
                                "aplicada a" +  
                                verb(@$exp_1$@) 

verb' (@$dom(exp_1)$@)             = "dominio de" +  
                                verb(@$exp_1$@) 

verb' (@$ran(exp_1)$@)              = "rango de" +  
                                 verb(@$exp_1$@)  
\end{minted}

\bigskip
Como era de esperarse, podemos notar que, a fin de lograr descripciones mas naturales, deberemos contemplar algunas combinaciones entre los distintos términos posibles. Por ejemplo, podemos ver que en el caso de la igualdad entre conjuntos, se propone una descripción para el caso en que uno de los conjuntos esté formado por un único elemento y otra verbalización distinta en el caso de que éste conjunto se se encuentre vacío.

Por otro lado, podemos observar que en algunos casos el texto generado por nuestra verbalización puede variar de acuerdo a aspectos gramaticales, en particular, las distintas variantes de nuestras oraciones dependerán del género y número de los constituyentes de las mismas. En estos casos, nos debemos asegurar que las palabras generadas por nuestro sistema concuerden con rasgos gramaticales de, por ejemplo, otra palabra introducida por medio de una designación. Para esto deberemos considerar algunas reglas de \emph{concordancia gramatical}, como, la concordancia de número entre el verbo y el núcleo del sujeto para el caso del operador $\in$:


\begin{minted}[escapeinside=@@]{haskell}
verb' (@$exp_1 \in exp_2$@)
     | (num == S) = verb(@$exp_1$@) + "pertenece a" + verb(@$exp_2$@) 
     | (num == P) = verb(@$exp_1$@) + "pertenecen a" + verb(@$exp_2$@) 
     where num = numero(@$exp_1$@)
\end{minted} 

\bigskip
En estos casos, deberemos realizar una análisis del sujeto y según corresponda conjugar correctamente el verbo utilizando, para el ejemplo anterior, deberíamos utilizar el verbo ``\emph{pertenece}'' o ``\emph{pertenecen}'' según corresponda.

Hay algunos casos de la definición anterior en los que el verbo que forma parte del predicado del texto a generar no requieren ningún tratamiento a fin de que el mismo concuerde en numero y forma con el sujeto (esto fue contemplado de antemano y no depende de cuestiones externas como podría ser texto introducido por medio de designaciones). Por ejemplo, los siguientes casos:

\begin{minted}[escapeinside=@@]{haskell}
verb' (@$exp_1 = \{\}$@)              = "no hay ningun elemento en" + 
                                 verb(@$exp_1$@) 
                                 
verb' (@$exp_1 \cap exp_2 = \{\}$@)        = verb(@$exp_1$@)  +  
                                "y" +  
                                verb(@$exp_2$@)  +  
                                "no tienen ningun elemento en comun"
\end{minted} 

\bigskip
Por otro lado, podemos observar que las oraciones en las que debemos prestar especial atención a cuestiones de concordancia gramatical resultan oraciones bimembres, todas expresadas en tiempo presente, conformadas de la siguiente manera: 
\begin{figure}[H]
\center
\textbf{sujeto} + \textbf{verbo} + \textbf{objeto}
\end{figure}

\noindent
Como por ejemplo, los siguientes casos:

\bigskip
\begin{enumerate}
 \item \nlgfun{verb'($exp1 \in exp2$)} $\rightarrow$ \textbf{sujeto} + \emph{``pertenece(n)''} + \textbf{objeto}
 \item \nlgfun{verb'($exp1 \subset exp2$)} $\rightarrow$ \textbf{sujeto} + \emph{``está(n) incluido/a(s)''} + \textbf{objeto}
\end{enumerate}

\bigskip
Puntualmente observamos que será necesario conjugar el verbo de una oración de forma tal que concuerde con el número del sujeto de la misma. Algo parecido pasa con el atributo en los casos que utilizamos un verbo copulativo en el que deberemos hacer concordar el mismo con número y género del sujeto. Podemos ver que en el ejemplo anterior la palabra \emph{``incluido''} cumple el rol de atributo del verbo copulativo \emph{``estar''} y deberá concordar en número y forma con el sujeto de la oración. 

Otro punto a tener en cuenta al momento de verbalizar una expresión será que las designaciones documentadas por el usuario pueden no contener un artículo que acompañe al sustantivo. Nuestro sistema deberá identificar los casos en los que sea necesario y agregar el artículo apropiado de forma que concuerde en género y número con el sustantivo. 

Estas observaciones sobre la concordancia gramatical entre los constituyentes de nuestras oraciones serán importantes para el desarrollo del realizador lingüístico que estudiaremos en detalle en el capítulo \ref{cap:realization}.

Por último, retomemos a la verbalización de expresiones designadas. Como vimos en el capítulo~\ref{cap:designaciones} una designación podría encontrarse parametrizada. Volvamos a la función \emph{designacion} introducida anteriormente, para el caso en el que la expresión se encuentre designada mediante una designación no parametrizada el resultado de esta función debería ser el texto tal cual se encuentra en la designación. Por otro lado, si se tratase de una designación parametrizada el texto a generar deberá depender de la verbalización de su parámetro. Por ejemplo, teniendo en cuenta las designaciones introducidas en la figura~\ref{fig:ej_designacion}, obtendríamos la siguiente designación para el término $s?$:

\begin{figure}[H]
\center
$designacion(s?) \rightarrow \text{\emph{``símbolo a buscar''}}$
\end{figure}

\noindent
Por otro lado, si se tratara de un termino que machea con una designación parametrizada, como $st~s?$ deberíamos resolverla de la siguiente manera: 

\begin{figure}[H]
\center
$designacion(st~s?) \rightarrow \text{\emph{``información asociada a''}} + verb(s?)$
\end{figure}



