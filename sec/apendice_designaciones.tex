\chapter{¿Qué designar en una especificación Z?}
\label{ape:designaciones}

Las designaciones son la principal fuente de conocimiento para nuestro sistema, estas son fundamentales para que nuestro sistema de NLG pueda generar descripciones independientes del dominio de aplicación. A continuación enumeraremos una serie de pautas que deberemos tener en cuenta para que éste produzca textos lo más fluidos y naturales posibles.

\bigskip
A fin de un correcto funcionamiento de nuestro sistema, resulta indispensable designar los siguientes elementos de una especificación:

\bigskip
\begin{enumerate}
	\item Nombres de esquemas de operación totales.
	\item Variables de entrada/salida.
	\item Variables de estado.
	\item Tipos básicos.
	\item Nombres de constructores de tipos libres.
\end{enumerate}

\bigskip
Además, a partir de un análisis realizado sobre textos generados por nuestro sistema, observamos que, a fin de poder generar textos más fluidos y naturales para el lector, si el objeto a designar es una función es recomendable designar:

\bigskip
\begin{enumerate}
	\item La función (f).
	\item El dominio (dom f).
	\item La aplicación (f x).
\end{enumerate}

\bigskip
Esto se debe a que es muy común que aparezcan en las clases de prueba generadas por el TTF el dominio como la aplicación de función y observamos que es posible obtener mejores resultados si designamos las tres expresiones enumeradas anteriormente (de lo contrario nuestro sistema describirá los casos anteriores en término de funciones y no en términos del dominio de aplicación).
