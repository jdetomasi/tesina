\chapter{Guía de estilo para designaciones}
\label{ape:designaciones}

Las designaciones son la principal fuente de conocimiento del dominio. Éstas son fundamentales para que el sistema de NLG pueda generar descripciones independientes del dominio de aplicación. A continuación, se enumera una serie de pautas que el ingeniero realizando la especificación debe tener en cuenta para que el sistema produzca textos lo más fluidos y naturales posibles.

\bigskip
A fin de un correcto funcionamiento de el sistema, resulta indispensable designar los siguientes elementos de una especificación:

\bigskip
\begin{enumerate}
	\item Nombres de esquemas de operación totales
	\item Variables de entrada/salida
	\item Variables de estado
	\item Tipos básicos
	\item Nombres de constructores de tipos libres
\end{enumerate}

\bigskip
Cuando el objeto a designar sea una función es recomendable designar:

\bigskip
\begin{enumerate}
	\item La función (f)
	\item El dominio de la función (dom f)
	\item La aplicación de la función (f x)
	\item El rango de la función (ran f)
\end{enumerate}

\bigskip
Como mencionamos en la sección \ref{cap:designaciones}, la función que cumplen las designaciones en este trabajo difiere un poco de la propuesta por Jackson~\cite{jackson}. Jackson propone escribir la menor cantidad de designaciones posibles y definir otros términos en base a las éstas. Para nosotros, las designaciones resultan la principal fuente de conocimiento para el sistema de NLG y en algunos casos (como los enumerados anteriormente para designar una función) designar los ítems propuestos le permitirá a nuestro sistema la generación de descripciones de mayor calidad. De no hacerlo, si por ejemplo escribiéramos sólo la designación para $f$, nuestro sistema se vería obligado a verbalizar $\dom f$ de la siguiente manera:

\begin{figure}[H]
\center
$\texttt{verb'}(\dom f) \rightarrow \text{\emph{``el dominio''}} + \texttt{verb}(f)$
\end{figure}

Algo similar ocurriría para los casos en los que se quiera verbalizar $\ran f$ y $f~x$. Esto introduciría términos como ``dominio'' o ``rango'' y ``aplicación'', que refieren a elementos del modelo y no a elementos del dominio de aplicación del sistema a testear, generando de esta forma descripciones poco naturales.
