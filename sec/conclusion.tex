\chapter{Conclusiones y trabajo futuro}
\label{cap:conclusion}

\section*{Conclusión}
En este trabajo hemos presentado y desarrollado una solución para la generación automática de descripciones en lenguaje natural para clases de prueba generadas por el TTF. Para esto, fue necesario utilizar técnicas de generación de lenguaje natural. En particular, seguimos la metodología propuesta por Reiter y Dale \cite{reiter_dale}.

En primer instancia recolectamos y analizamos un corpus con textos de ejemplo. A partir de éste extrajimos los requerimientos más importantes para el desarrollo de nuestro sistema. Luego, estudiamos en detalle las tareas mas importantes que debían ser llevadas a cabo por nuestro sistema. En el capítulo \ref{cap:document_planning} hicimos especial incapie en la determinación de contenido y cómo el razonamiento con los datos puede mejorar notablemente la calidad de las descripciones generadas por nuestro sistema. En el capítulo \ref{cap:microplanning} utilizamos el conjunto de reglas de traducción obtenido como resultado del análisis del corpus para definir la tarea de lexicalización de expresiones Z, encargada de escoger que palabras y constructores sintácticos utilizar para comunicar la información requerida. Estudiamos además, el importante rol que cumplen las designaciones para este trabajo y como nuestra tarea de lexicalización hace uso de la información contenida en las mismas. Por último, en el capítulo \ref{cap:realization} nos encargamos fundamentalmente de estudiar los aspectos necesarios para, finalmente, generar texto sintáctica, ortográfica y gramaticalmente correcto.

Como resultado de este trabajo, hemos implementado un prototipo en base al desarrollo realizado. Utilizando para esto un diseño flexible capaz de admitir futuras posibilidades de evolución. 

El sistema desarrollado trabaja principalmente con la información contenida en las clases de prueba, haciendo un especial uso de las designaciones que acompañan a la especificación. Estas designaciones resultan una fuente fundamental de información que le permite a nuestro sistema generar las porciones de texto dependientes del dominio de aplicación y de esta forma lograr una generación de descripciones en lenguaje natural independiente del dominio de aplicación y de la cantidad de clases y casos de pruebas.

\section*{Trabajo futuro}

Consideramos que el trabajo realizado esta abierto a nuevos aportes. A continuación detallaremos algunos de los aportes que creemos que podrían realizarse como trabajo futuro.

El sistema desarrollado no cubre la totalidad de las expresiones y operadores de Z; dentro del alcance de este trabajo establecimos un subconjunto de éstas con las que trabajamos. Un aporte interesante podría ser la extensión de expresiones soportadas por nuestro sistema. Para esto debería expandirse el corpus, contemplando casos para los operadores a introducir y luego volver a realizar un análisis del mismo extrayendo nuevas reglas para describir los mismos. 

Por otro lado, creemos que es posible profundizar el trabajo sobre los datos de entrada e incluir nuevas tareas de razonamiento sobre los mismos en la etapa de \textit{document planning}. Esto le permitiría a nuestro sistema mejorar la calidad de los textos producidos. Por ejemplo, se podría trabajar sobre las expresiones redundantes que pueden aparecer en una clase de prueba.

Como mencionamos en el capítulo \ref{cap:microplanning} nuestro sistema no realiza tareas de agregación y generación de expresiones de referencia. Podría resultar interesante introducir tareas de este tipo a nuestro sistema.

El sistema desarrollado actualmente es capaz de producir únicamente textos en castellano. Un aporte importante podría ser expandir el mismo para que éste contemple la generación de textos en otras lenguas, por ejemplo en inglés. Para generar textos en inglés por ejemplo, existen realizadores lingüísticos de código abierto que podríamos utilizar para ayudarnos en este desarrollo; en este caso tendríamos que implementar un nuevo módulo de \textit{microplanning} de modo que sea capaz de generar las estructuras utilizadas por el realizador lingüístico a utilizar.