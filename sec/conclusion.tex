\chapter{Conclusiones y trabajos futuros}
\label{cap:conclusion}

En este trabajo hemos presentado y desarrollado una solución para la generación automática de descripciones en lenguaje natural de clases de prueba generadas por el TTF. Para esto, fue necesario utilizar técnicas de generación de lenguaje natural. En particular, seguimos la metodología propuesta por Reiter y Dale \cite{reiter_dale} que resulta una arquitectura estándar para sistemas de NLG.

En primera instancia, recolectamos y analizamos un corpus con textos de ejemplo. A partir de éste extrajimos los requerimientos más importantes para el desarrollo de nuestro sistema. Luego, estudiamos en detalle las tareas más importantes que deben ser llevadas a cabo por nuestro sistema. En el capítulo \ref{cap:document_planning} hicimos especial hincapié en la determinación de contenido y en cómo el razonamiento con los datos puede mejorar notablemente la calidad de las descripciones generadas por nuestro sistema. En el capítulo \ref{cap:microplanning} utilizamos el conjunto de reglas de traducción obtenido como resultado del análisis del corpus para definir la tarea de lexicalización de expresiones Z, encargada de escoger qué palabras y constructores sintácticos utilizar para comunicar la información requerida. Estudiamos, además, el importante rol que cumplen las designaciones para este trabajo y cómo nuestra tarea de lexicalización hace uso de la información contenida en las mismas. En el capítulo \ref{cap:realization} nos encargamos fundamentalmente de estudiar los aspectos necesarios para, finalmente, generar texto sintáctica, ortográfica y gramaticalmente correcto.

Como resultado de este trabajo, hemos implementado un prototipo en base al desarrollo realizado, utilizando para esto un diseño flexible capaz de admitir futuras posibilidades de evolución. El sistema desarrollado trabaja principalmente con la información contenida en las clases de prueba, haciendo un especial uso de las designaciones que acompañan a la especificación. Estas designaciones resultan una fuente fundamental de información que le permite a nuestro sistema generar las porciones de texto dependientes del dominio de aplicación y de esta forma lograr un sistema de generación de descripciones en lenguaje natural independiente del dominio de aplicación y de la cantidad de clases y casos de pruebas.

El trabajo descripto en esta tesis es el primero en el cual se aplican técnicas del estado del arte en la generación de lenguaje natural al problema de accesibilidad de las especificaciones formales. Esto continua con la línea de investigación iniciada por Cristiá y Plüss \cite{cristia_pluss} en busca de métodos y herramientas que traduzcan especificaciones formales de sistemas reales en documentos accesibles para expertos en el dominio sin formación en métodos formales específicos. El trabajo realizado por Cristiá y Plüss \cite{cristia_pluss} presenta una solución ad-hoc para un problema particular, haciendo a la solución dependiente del dominio de aplicación y el número de operaciones incluidas en la especificación. En este sentido, la solución propuesta en esta tesina resulta superadora ya que gracias a la información contenida en las clases de prueba y las designaciones que acompañan la especificación hemos sido capaces de dar con una solución independiente del dominio de aplicación y del número de operaciones del sistema. 

\section*{Trabajos futuros}

El trabajo realizado esta abierto a nuevos aportes. A continuación detallaremos algunos de los aportes que creemos que podrían realizarse como trabajo futuro.

Una continuación importante para nuestro trabajo sería la realización de una evaluación de los textos generados por el sistema. Para esto deberíamos juntar un grupo de personas capaz de leer Z y proveerles un conjunto de clases de prueba acompañado de sus descripciones en lenguaje natural generadas por nuestro prototipo. Estas personas deberán evaluar las mismas en términos de exactitud, fluidez del texto, etc. Podríamos además incluir descripciones traducidas manualmente por expertos en el dominio de aplicación para luego comparar los resultados de las descripciones generadas por el sistema contra descripciones redactadas por un experto.

El sistema desarrollado no cubre la totalidad de las expresiones y operadores de Z; dentro del alcance de este trabajo establecimos un subconjunto de éstos en el que nos enfocamos. Una continuación natural de este trabajo sería la extensión del conjunto de expresiones soportadas por nuestro sistema. Para esto debería expandirse el corpus, contemplando casos para los operadores a introducir y luego volver a realizar un análisis del mismo extrayendo nuevas reglas para describir las nuevas expresiones en lenguaje natural. 

También es posible profundizar el trabajo sobre los datos de entrada e incluir nuevas tareas de razonamiento sobre los mismos en la etapa de \textit{document planning}. Esto le permitiría a nuestro sistema mejorar la calidad de los textos producidos. Por ejemplo, se podría trabajar sobre las expresiones redundantes que pueden aparecer en una clase de prueba.

Como mencionamos en el capítulo \ref{cap:microplanning}, nuestro sistema no realiza tareas de agregación y generación de expresiones de referencia. Una extensión natural sería la de incorporar estas tareas como parte del \textit{document planning}.

El sistema desarrollado actualmente es capaz de producir únicamente textos en castellano. Un aporte importante sería expandir el mismo para permitir la generación de textos en otras lenguas. Para generar textos en inglés, por ejemplo, existen realizadores lingüísticos de código abierto que podrían ser utilizados en este desarrollo. En este caso tendríamos que implementar un nuevo módulo de \textit{microplanning} de modo que sea capaz de generar las estructuras utilizadas por el realizador lingüístico a utilizar.