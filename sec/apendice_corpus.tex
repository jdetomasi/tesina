\chapter{Corpus de descripciones}
\label{ape:corpus}

Para construir el corpus utilizado para este trabajo se utilizaron 5 especificaciones de distintos dominios de aplicación con sus respectivos casos y clases de prueba generados por Fastest. De estas clases de prueba se escogieron las más interesantes para traducir, acabando con un total de 98 clases de prueba.

A continuación se presentarán 3 de los ejemplos recolectados que forman parte del corpus utilizado y resultan representativos del total de ejemplos recolectados. En estos se incluyó tanto la especificación utilizada para generar las clases de prueba a traducir, como también las designaciones necesarias para poder describir las clases de prueba seleccionadas y las descripciones de las mismas.

\section*{Ejemplo: \textit{Symbol Table}}

\subsection*{Especificación y designaciones}

\begin{zed}
[SYM, VAL] \also
REPORT ::= ok | symbolNotPresent
\end{zed}

\begin{itemize}
  \item Tabla de símbolos $\approx st$ \\
  \item Símbolos cargados en la tabla $\approx dom~st$ \\
  \item Valor de $x$ $\approx st~x$ 
\end{itemize}

\begin{schema}{ST}
st: SYM \pfun VAL
\end{schema}

\begin{itemize}
  \item Intenta actualizar el valor de un símbolo $\approx Update$ \\
  \item Símbolo a actualizar $\approx s?$ \\
  \item Valor a actualizar  $\approx v?$ 
\end{itemize}

\begin{schema}{Update}
\Delta ST \\
s?: SYM \\
v?: VAL \\
rep!: REPORT
\where
st' = st \oplus \{s? \mapsto v?\} \\
rep! = ok
\end{schema}

\begin{itemize}
  \item Símbolo a buscar $\approx s?$ \\
  \item Valor del símbolo buscado  $\approx v?$ 
\end{itemize}

\begin{schema}{LookUpOk}
\Xi ST \\
s?: SYM \\
v!: VAL \\
rep!: REPORT
\where
s? \in \dom st \\
v! = st~s? \\
rep! = ok
\end{schema}

\begin{schema}{LookUpE}
\Xi ST \\
s?: SYM \\
rep!: REPORT
\where
s? \notin \dom st \\
rep! = symbolNotPresent
\end{schema}

\begin{itemize}
  \item Intenta buscar el valor de un símbolo $\approx LookUp$ 
\end{itemize}

\begin{zed}
LookUp == LookUpOk \lor LookUpE
\end{zed}

\begin{itemize}
  \item Símbolo a eliminar $\approx s?$ 
\end{itemize}

\begin{schema}{DeleteOk}
\Delta ST \\
s?: SYM \\
rep!: REPORT
\where
s? \in \dom st \\
st' = \{s?\} \ndres st \\
rep! = ok
\end{schema}

\begin{itemize}
  \item Intenta eliminar un símbolo de la tabla $\approx Delete$ 
\end{itemize}

\begin{zed}
DeleteE == LookUpE \also
Delete == DeleteOk \lor DeleteE
\end{zed}

\subsection*{Clases de prueba y descripciones}

\begin{schema}{LookUp\_ SP\_ 1}\\
 st : SYM \pfun VAL \\
 s? : SYM 
\where
 s? \in \dom st \\
 \dom st = \{ s? \}
\end{schema}

\begin{tcolorbox}[colback=gray!5!white,colframe=gray!50!black,
  colbacktitle=gray!75!black,title=LookUp\_ SP\_ 1]
  Se intenta buscar el valor de un símbolo, cuando:
     \begin{itemize}
        \item[--]{El símbolo a buscar pertenece a los símbolos cargados en la tabla.}
        \item[--]{El símbolo a buscar es el único elemento de los símbolos cargados en la tabla.}
     \end{itemize}
\end{tcolorbox}
%TODO aca se podria eliminar la primera restriccion

\begin{schema}{LookUp\_ SP\_ 4}\\
 st : SYM \pfun VAL \\
 s? : SYM 
\where
 s? \notin \dom st \\
 \dom st \neq \{ s? \} \\
 s? \in \dom st
\end{schema}

\begin{tcolorbox}[colback=gray!5!white,colframe=gray!50!black,
  colbacktitle=gray!75!black,title=LookUp\_ SP\_ 4]
  Se intenta buscar el valor de un símbolo, cuando:
     \begin{itemize}
        \item[--]{El símbolo a buscar no pertenece a los símbolos cargados en la tabla.}
        \item[--]{Los símbolos cargados en la tabla no son iguales al conjunto formado por el símbolo a buscar.}
        \item[--]{El símbolo a buscar pertenece a los símbolos cargados en la tabla.}
     \end{itemize}
\end{tcolorbox}

\begin{schema}{Update\_ SP\_ 1}\\
 st : SYM \pfun VAL \\
 s? : SYM \\
 v? : VAL 
\where
 st = \{ \} \\
 \{ s? \mapsto v? \} = \{ \}
\end{schema}

\begin{tcolorbox}[colback=gray!5!white,colframe=gray!50!black,
  colbacktitle=gray!75!black,title=Update\_ SP\_ 1]
  Se intenta actualizar el valor de un símbolo, cuando:
     \begin{itemize}
        \item[--]{No hay ningún elemento en la tabla de símbolos.}
        \item[--]{No hay elementos en el conjunto formado por el par ordenado por: el símbolo a actualizar y el valor a actualizar.}
      \end{itemize}
\end{tcolorbox}


\begin{schema}{Update\_ SP\_ 2}\\
 st : SYM \pfun VAL \\
 s? : SYM \\
 v? : VAL 
\where
 st = \{ \} \\
 \{ s? \mapsto v? \} \neq \{ \}
\end{schema}

\begin{tcolorbox}[colback=gray!5!white,colframe=gray!50!black,
  colbacktitle=gray!75!black,title=Update\_ SP\_ 2]
  Se intenta actualizar el valor de un símbolo, cuando:
     \begin{itemize}
        \item[--]{No hay ningún elemento en la tabla de símbolos.}
        \item[--]{Existe al menos un elemento en el conjunto formado por el par ordenado por: el símbolo a actualizar y el valor a actualizar.}
     \end{itemize}
\end{tcolorbox}


\begin{schema}{Update\_ SP\_ 3}\\
 st : SYM \pfun VAL \\
 s? : SYM \\
 v? : VAL 
\where
 st \neq \{ \} \\
 \{ s? \mapsto v? \} = \{ \}
\end{schema}

\begin{tcolorbox}[colback=gray!5!white,colframe=gray!50!black,
  colbacktitle=gray!75!black,title=Update\_ SP\_ 3]
  Se intenta actualizar el valor de un símbolo, cuando:
     \begin{itemize}
        \item[--]{Existe al menos un elemento en la tabla de símbolos.}
        \item[--]{No hay elementos en el conjunto formado por el par ordenado por: el símbolo a actualizar y el valor a actualizar.}
     \end{itemize}
\end{tcolorbox}


\begin{schema}{Update\_ SP\_ 4}\\
 st : SYM \pfun VAL \\
 s? : SYM \\
 v? : VAL 
\where
 st \neq \{ \} \\
 \{ s? \mapsto v? \} \neq \{ \} \\
 \dom st = \dom \{ s? \mapsto v? \}
\end{schema}

\begin{tcolorbox}[colback=gray!5!white,colframe=gray!50!black,
  colbacktitle=gray!75!black,title=Update\_ SP\_ 4]
  Se intenta actualizar el valor de un símbolo, cuando:
     \begin{itemize}
        \item[--]{Existe al menos un elemento en la tabla de símbolos.}
        \item[--]{Existe al menos un elemento en el conjunto formado por el par ordenado por: el símbolo a actualizar y el valor a actualizar.}
        \item[--]{El símbolo a actualizar pertenece a los símbolos cargados en la tabla.}
     \end{itemize}
\end{tcolorbox}

\begin{schema}{Update\_ SP\_ 5}\\
 st : SYM \pfun VAL \\
 s? : SYM \\
 v? : VAL 
\where
 st \neq \{ \} \\
 \{ s? \mapsto v? \} \neq \{ \} \\
 \dom \{ s? \mapsto v? \} \subset \dom st
\end{schema}

\begin{tcolorbox}[colback=gray!5!white,colframe=gray!50!black,
  colbacktitle=gray!75!black,title=Update\_ SP\_ 5]
  Se intenta actualizar el valor de un símbolo, cuando:
     \begin{itemize}
        \item[--]{Existe al menos un elemento en la tabla de símbolos.}
        \item[--]{Existe al menos un elemento en el conjunto formado por el par ordenado por: el símbolo a actualizar y el valor a actualizar.}
        \item[--]{El símbolo a actualizar es el único elemento de los símbolos cargados en la tabla.}
     \end{itemize}
\end{tcolorbox}


\begin{schema}{Update\_ SP\_ 6}\\
 st : SYM \pfun VAL \\
 s? : SYM \\
 v? : VAL 
\where
 st \neq \{ \} \\
 \{ s? \mapsto v? \} \neq \{ \} \\
 ( \dom st \cap \dom \{ s? \mapsto v? \} ) = \{ \}
\end{schema}

\begin{tcolorbox}[colback=gray!5!white,colframe=gray!50!black,
  colbacktitle=gray!75!black,title=Update\_ SP\_ 6]
  Se intenta actualizar el valor de un símbolo, cuando:
     \begin{itemize}
        \item[--]{Existe al menos un elemento en la tabla de símbolos.}
        \item[--]{Existe al menos un elemento en el conjunto formado por el par ordenado por: el símbolo a actualizar y el valor a actualizar.}
        \item[--]{Los símbolos cargados en la tabla están incluidos en el conjunto formado por el símbolo a actualizar.}
     \end{itemize}
\end{tcolorbox}


\begin{schema}{Delete\_ SP\_ 2}\\
 st : SYM \pfun VAL \\
 s? : SYM 
\where
 s? \in \dom st \\
 st \neq \{ \} \\
 \{ s? \} = \{ \}
\end{schema}

\begin{tcolorbox}[colback=gray!5!white,colframe=gray!50!black,
  colbacktitle=gray!75!black,title=Delete\_ SP\_ 2]
  Se intenta eliminar un símbolo de la tabla, cuando:
     \begin{itemize}
        \item[--]{El símbolo a eliminar pertenece a los símbolos cargados en la tabla.}
        \item[--]{Existe al menos un elemento en la tabla de símbolos.}
        \item[--]{No hay ningún elemento en el conjunto formado por: el símbolo a eliminar.}
     \end{itemize}
\end{tcolorbox}


\begin{schema}{Delete\_ SP\_ 3}\\
 st : SYM \pfun VAL \\
 s? : SYM 
\where
 s? \in \dom st \\
 st \neq \{ \} \\
 \{ s? \} = \dom st
\end{schema}

\begin{tcolorbox}[colback=gray!5!white,colframe=gray!50!black,
  colbacktitle=gray!75!black,title=Delete\_ SP\_ 3]
  Se intenta eliminar un símbolo de la tabla, cuando:
     \begin{itemize}
        \item[--]{El símbolo a eliminar pertenece a los símbolos cargados en la tabla.}
        \item[--]{Existe al menos un elemento en la tabla de símbolos.}
        \item[--]{El símbolo a eliminar es el único elemento de los símbolos cargados en la tabla.}
     \end{itemize}
\end{tcolorbox}


\begin{schema}{Delete\_ SP\_ 4}\\
 st : SYM \pfun VAL \\
 s? : SYM 
\where
 s? \in \dom st \\
 st \neq \{ \} \\
 \{ s? \} \neq \{ \} \\
 \{ s? \} \subset \dom st
\end{schema}

\begin{tcolorbox}[colback=gray!5!white,colframe=gray!50!black,
  colbacktitle=gray!75!black,title=Delete\_ SP\_ 4]
  Se intenta eliminar un símbolo de la tabla, cuando:
     \begin{itemize}
        \item[--]{El símbolo a eliminar pertenece los símbolos cargados en la tabla.}
        \item[--]{Existe al menos un elemento en la tabla de símbolos.}
        \item[--]{Existe al menos un elemento en el conjunto formado por: el símbolo a eliminar.}
        \item[--]{El símbolo a eliminar pertenece los símbolos cargados en la tabla.}
     \end{itemize}
\end{tcolorbox}


\begin{schema}{Delete\_ SP\_ 5}\\
 st : SYM \pfun VAL \\
 s? : SYM 
\where
 s? \in \dom st \\
 st \neq \{ \} \\
 \{ s? \} \neq \{ \} \\
 \{ s? \} \cap \dom st = \{ \}
\end{schema}

\begin{tcolorbox}[colback=gray!5!white,colframe=gray!50!black,
  colbacktitle=gray!75!black,title=Delete\_ SP\_ 5]
  Se intenta eliminar un símbolo de la tabla, cuando:
     \begin{itemize}
        \item[--]{El símbolo a eliminar pertenece los símbolos cargados en la tabla.}
        \item[--]{Existe al menos un elemento en la tabla de símbolos.}
        \item[--]{Existe al menos un elemento en el conjunto formado por: el símbolo a eliminar.}
        \item[--]{El conjunto formado por el símbolo a eliminar y los símbolos cargados en la tabla no tienen ningún elemento en común.}
     \end{itemize}
\end{tcolorbox}


\begin{schema}{Delete\_ SP\_ 7}\\
 st : SYM \pfun VAL \\
 s? : SYM 
\where
 s? \in \dom st \\
 st \neq \{ \} \\
 \{ s? \} \cap \dom st \neq \{ \} \\
 \lnot \dom st \subseteq \{ s? \} \\
 \lnot \{ s? \} \subseteq \dom st
\end{schema}

\begin{tcolorbox}[colback=gray!5!white,colframe=gray!50!black,
  colbacktitle=gray!75!black,title=Delete\_ SP\_ 7]
  Se intenta eliminar un símbolo de la tabla, cuando:
     \begin{itemize}
        \item[--]{El símbolo a eliminar pertenece los símbolos cargados en la tabla.}
        \item[--]{Existe al menos un elemento en la tabla de símbolos.}
        \item[--]{El conjunto formado por el símbolo a eliminar y los símbolos cargados en la tabla tienen tienen al menos un elemento en común.}
        \item[--]{Existe al menos un elemento en los símbolos cargados en la tabla que no pertenece al conjunto formado por: el símbolo a eliminar.}
        \item[--]{El símbolo a eliminar no pertenece al los símbolos cargados en la tabla.}
     \end{itemize}
\end{tcolorbox}


\begin{schema}{Delete\_ SP\_ 12}\\
 st : SYM \pfun VAL \\
 s? : SYM 
\where
 s? \notin \dom st \\
 st \neq \{ \} \\
 \{ s? \} \neq \{ \} \\
 \{ s? \} \cap \dom st = \{ \}
\end{schema}

\begin{tcolorbox}[colback=gray!5!white,colframe=gray!50!black,
  colbacktitle=gray!75!black,title=Delete\_ SP\_ 12]
  Se intenta eliminar un símbolo de la tabla, cuando:
     \begin{itemize}
        \item[--]{El símbolo a eliminar no pertenece los símbolos cargados en la tabla.}
        \item[--]{Existe al menos un elemento en la tabla de símbolos.}
        \item[--]{Existe al menos un elemento en el conjunto formado por: el símbolo a eliminar.}
        \item[--]{El conjunto formado por el símbolo a eliminar y los símbolos cargados en la tabla no tienen ningún elemento en común.}
     \end{itemize}
\end{tcolorbox}


\begin{schema}{Delete\_ SP\_ 13}\\
 st : SYM \pfun VAL \\
 s? : SYM 
\where
 s? \notin \dom st \\
 st \neq \{ \} \\
 \{ s? \} \cap \dom st \neq \{ \} \\
 \dom st \subset \{ s? \}
\end{schema}

\begin{tcolorbox}[colback=gray!5!white,colframe=gray!50!black,
  colbacktitle=gray!75!black,title=Delete\_ SP\_ 13]
  Se intenta eliminar un símbolo de la tabla, cuando:
     \begin{itemize}
        \item[--]{El símbolo a eliminar no pertenece los símbolos cargados en la tabla.}
        \item[--]{Existe al menos un elemento en la tabla de símbolos.}
        \item[--]{El conjunto formado por el símbolo a eliminar y los símbolos cargados en la tabla tienen tienen al menos un elemento en común.}
        \item[--]{Los símbolos cargados en la tabla están incluidos en el conjunto formado por: el símbolo a eliminar.}
     \end{itemize}
\end{tcolorbox}

\section*{Ejemplo: Banco}

\subsection*{Especificación y designaciones}

\begin{zed}
[NCTA] \also

SALDO == \nat \also

MENSAJES ::= ok | numeroClienteEnUso | noPoseeSaldoSuficiente | saldoNoNulo
\end{zed}

\begin{itemize}
  \item Cajas de ahorro existentes en el banco $\approx cajas$ \\
  \item Números de cuenta cargados en el banco $\approx dom~cajas$ \\
  \item Dinero depositado en la caja de ahorro de $x$ $\approx caja~x$ 
\end{itemize}

\begin{schema}{Banco}
cajas: NCTA \pfun SALDO
\end{schema}

\begin{itemize}
  \item Número de cuenta del cliente $\approx num?$ 
\end{itemize}

\begin{schema}{DepositarOk}
\Delta Banco \\
num?: NCTA; m?: \num
\where
num? \in \dom cajas \\
m? > 0 \\
cajas' = cajas \oplus \{num? \mapsto cajas~num? + m?\}
\end{schema}

\begin{zed}
DepositarE1 == [\Xi Banco; num?:NCTA | num? \notin \dom cajas]
\end{zed}

\begin{zed}
DepositarE2 == [\Xi Banco; m?: \num | m? \leq 0]
\end{zed}

\begin{itemize}
  \item Intenta depositar dinero en una cuenta $\approx Depositar$ 
\end{itemize}

\begin{zed}
Depositar == DepositarOk \lor DepositarE1 \lor DepositarE2
\end{zed}

\begin{itemize}
  \item Número de cuenta del nuevo cliente $\approx num?$ 
\end{itemize}

\begin{schema}{NuevoClienteOk}
\Delta Banco \\
num?:NCTA \\
rep!:MENSAJES
\where
num? \notin \dom cajas \\
cajas' = cajas \cup \{num? \mapsto 0\} \\
rep! = ok
\end{schema}

\begin{schema}{NuevoClienteE}

\Xi Banco \\
num?:NCTA \\
rep!:MENSAJES
\where
num? \in \dom cajas \\
rep! = numeroClienteEnUso
\end{schema}

\begin{itemize}
  \item Intenta cargar un nuevo cliente $\approx NuevoCliente$ 
\end{itemize}

\begin{zed}
NuevoCliente == NuevoClienteOk \lor NuevoClienteE
\end{zed}

\begin{itemize}
  \item Número de cuenta del cliente $\approx num?$ 
\end{itemize}

\begin{schema}{ExtraerOk}
\Delta Banco \\
num?:NCTA \\
m?:SALDO \\
rep!:MENSAJES
\where
num? \in \dom cajas \\
0 < m? \\
m? \leq cajas~num? \\
cajas' = cajas \oplus \{num? \mapsto (cajas~num?) - m?\} \\
rep! = ok
\end{schema}

\begin{zed}
ExtraerE1 == DepositarE1 \also

ExtraerE2 == DepositarE2
\end{zed}

\begin{schema}{ExtraerE3}
\Xi Banco \\
num?:NCTA \\
m?:SALDO \\
rep!:MENSAJES
\where
m? > cajas~num? \\
num? \in \dom cajas \\
rep! = noPoseeSaldoSuficiente
\end{schema}

\begin{itemize}
  \item Intenta realizar una extracción de dinero $\approx Extraer$ 
\end{itemize}

\begin{zed}
ExtraerE == ExtraerE1 \lor ExtraerE2 \lor ExtraerE3 \also

Extraer == ExtraerOk \lor ExtraerE
\end{zed}

\begin{itemize}
  \item Número de cuenta del cliente $\approx num$ 
\end{itemize}

\begin{schema}{PedirSaldoOk}
\Xi Banco \\
num?:NCTA \\
saldo!:SALDO \\
rep!:MENSAJES
\where
num? \in \dom cajas \\
saldo! = cajas~num? \\
rep! = ok
\end{schema}

\begin{itemize}
  \item Intenta consultar el saldo de un cliente $\approx PedirSaldo$ 
\end{itemize}

\begin{zed}
PedirSaldoE == DepositarE1 \also

PedirSaldo == PedirSaldoOk \lor PedirSaldoE
\end{zed}

\subsection*{Clases de prueba y descripciones}

\begin{schema}{NuevoCliente\_ SP\_ 2}\\
 cajas : NCTA \pfun SALDO \\
 num? : NCTA 
\where
 num? \notin \dom cajas \\
 cajas = \{ \} \\
 \{ num? \mapsto 0 \} \neq \{ \}
\end{schema}

\begin{tcolorbox}[colback=gray!5!white,colframe=gray!50!black,
  colbacktitle=gray!75!black,title=NuevoCliente\_SP\_2]
  Se intenta cargar un nuevo cliente, cuando:
     \begin{itemize}
        \item[--]{El número de cuenta del nuevo cliente no pertenece a los números de cuenta cargados en el banco.}
        \item[--]{No hay ningún elemento en las cajas de ahorro existentes en el banco.}
        \item[--]{Existe al menos un elemento en el conjunto formado por el par: número de cuenta y 0.}
     \end{itemize}
\end{tcolorbox}


\begin{schema}{NuevoCliente\_ SP\_ 4}\\
 cajas : NCTA \pfun SALDO \\
 num? : NCTA 
\where
 num? \notin \dom cajas \\
 cajas \neq \{ \} \\
 \{ num? \mapsto 0 \} \neq \{ \} \\
 \dom cajas = \dom \{ num? \mapsto 0 \}
\end{schema}

\begin{tcolorbox}[colback=gray!5!white,colframe=gray!50!black,
  colbacktitle=gray!75!black,title=NuevoCliente\_SP\_4]
  Se intenta cargar un nuevo cliente, cuando:
     \begin{itemize}
        \item[--]{El número de cuenta del nuevo cliente no pertenece a los números de cuenta cargados en el banco.}
        \item[--]{Existe al menos un elemento en las cajas de ahorro existentes en el banco.}
        \item[--]{Existe al menos un elemento en el conjunto formado pope el par: número de cuenta y 0.}
        \item[--]{El número de cuenta del nuevo cliente es el único elemento de los números de cuenta cargados en el banco.}
     \end{itemize}
\end{tcolorbox}


\begin{schema}{NuevoCliente\_ SP\_ 5}\\
 cajas : NCTA \pfun SALDO \\
 num? : NCTA 
\where
 num? \notin \dom cajas \\
 cajas \neq \{ \} \\
 \{ num? \mapsto 0 \} \neq \{ \} \\
 \dom \{ num? \mapsto 0 \} \subset \dom cajas
\end{schema}

\begin{tcolorbox}[colback=gray!5!white,colframe=gray!50!black,
  colbacktitle=gray!75!black,title=NuevoCliente\_SP\_5]
  Se intenta cargar un nuevo cliente, cuando:
     \begin{itemize}
        \item[--]{El número de cuenta del nuevo cliente no pertenece a los números de cuenta cargados en el banco.}
        \item[--]{Existe al menos un elemento en las cajas de ahorro existentes en el banco.}
        \item[--]{Existe al menos un elemento en el conjunto formado pone el par: número de cuenta y 0.}
        \item[--]{El número de cuenta del nuevo cliente pertenece a los números de cuenta cargados en el banco.}
     \end{itemize}
\end{tcolorbox}


\begin{schema}{NuevoCliente\_ SP\_ 6}\\
 cajas : NCTA \pfun SALDO \\
 num? : NCTA 
\where
 num? \notin \dom cajas \\
 cajas \neq \{ \} \\
 \{ num? \mapsto 0 \} \neq \{ \} \\
 ( \dom cajas \cap \dom \{ num? \mapsto 0 \} ) = \{ \}
\end{schema}

\begin{tcolorbox}[colback=gray!5!white,colframe=gray!50!black,
  colbacktitle=gray!75!black,title=NuevoCliente\_SP\_6]
  Se intenta cargar un nuevo cliente, cuando:
     \begin{itemize}
        \item[--]{El número de cuenta del nuevo cliente no pertenece a los números de cuenta cargados en el banco.}
        \item[--]{Existe al menos un elemento en las cajas de ahorro existentes en el banco.}
        \item[--]{Existe al menos un elemento en el conjunto formado pode el par: número de cuenta y 0.}
        \item[--]{El número de cuenta del nuevo cliente no pertenece a los números de cuenta cargados en el banco.}
     \end{itemize}
\end{tcolorbox}


\begin{schema}{NuevoCliente\_ SP\_ 7}\\
 cajas : NCTA \pfun SALDO \\
 num? : NCTA 
\where
 num? \notin \dom cajas \\
 cajas \neq \{ \} \\
 \{ num? \mapsto 0 \} \neq \{ \} \\
 \dom cajas \subset \dom \{ num? \mapsto 0 \}
\end{schema}

\begin{tcolorbox}[colback=gray!5!white,colframe=gray!50!black,
  colbacktitle=gray!75!black,title=NuevoCliente\_SP\_7]
  Se intenta cargar un nuevo cliente, cuando:
     \begin{itemize}
        \item[--]{El número de cuenta del nuevo cliente no pertenece a los números de cuenta cargados en el banco.}
        \item[--]{Existe al menos un elemento en las cajas de ahorro existentes en el banco.}
        \item[--]{Los números de cuenta cargados en el banco están incluidos en el conjunto formado por el número de cuenta del nuevo cliente.}
     \end{itemize}
\end{tcolorbox}


\begin{schema}{NuevoCliente\_ SP\_ 8}\\
 cajas : NCTA \pfun SALDO \\
 num? : NCTA 
\where
 num? \notin \dom cajas \\
 cajas \neq \{ \} \\
 \{ num? \mapsto 0 \} \neq \{ \} \\
 ( \dom cajas \cap \dom \{ num? \mapsto 0 \} ) \neq \{ \} \\
 \lnot \dom \{ num? \mapsto 0 \} \subseteq \dom cajas \\
 \lnot \dom cajas \subseteq \dom \{ num? \mapsto 0 \}
\end{schema}

\begin{tcolorbox}[colback=gray!5!white,colframe=gray!50!black,
  colbacktitle=gray!75!black,title=NuevoCliente\_SP\_8]
  Se intenta cargar un nuevo cliente, cuando:
     \begin{itemize}
        \item[--]{El número de cuenta del nuevo cliente no pertenece a los números de cuenta cargados en el banco.}
        \item[--]{Existe al menos un elemento en las cajas de ahorro existentes en el banco.}
        \item[--]{Existe al menos un elemento en el conjunto formado por el par: número de cuenta y 0.}
        \item[--]{El número de cuenta del nuevo cliente no pertenece a los números de cuenta cargados en el banco.}
        \item[--]{Existe al menos un elemento en los números de cuenta cargados en el banco que no está en el conjunto formado por el número de cuenta del nuevo cliente.}
     \end{itemize}
\end{tcolorbox}


\begin{schema}{NuevoCliente\_ SP\_ 12}\\
 cajas : NCTA \pfun SALDO \\
 num? : NCTA 
\where
 num? \in \dom cajas \\
 cajas \neq \{ \} \\
 \{ num? \mapsto 0 \} \neq \{ \} \\
 \dom cajas = \dom \{ num? \mapsto 0 \}
\end{schema}


\begin{tcolorbox}[colback=gray!5!white,colframe=gray!50!black,
  colbacktitle=gray!75!black,title=NuevoCliente\_SP\_12]
  Se intenta cargar un nuevo cliente, cuando:
     \begin{itemize}
        \item[--]{El número de cuenta del nuevo cliente pertenece a los números de cuenta cargados en el banco.}
        \item[--]{Existe al menos un elemento en las cajas de ahorro existentes en el banco.}
        \item[--]{Existe al menos un elemento en el conjunto formado por el par: número de cuenta y 0.}
        \item[--]{El número de cuenta del nuevo cliente es el único elemento de los números de cuenta cargados en el banco.}
     \end{itemize}
\end{tcolorbox}


\begin{schema}{PedirSaldo\_ SP\_ 1}\\
 cajas : NCTA \pfun SALDO \\
 num? : NCTA 
\where
 num? \in \dom cajas \\
 \dom cajas = \{ num? \}
\end{schema}

\begin{tcolorbox}[colback=gray!5!white,colframe=gray!50!black,
  colbacktitle=gray!75!black,title=PedirSaldo\_SP\_1]
  Se intenta consultar el saldo de un cliente, cuando:
     \begin{itemize}
        \item[--]{El número de cuenta indicado pertenece a los números de cuenta cargados en el banco.}
        \item[--]{El número de cuenta indicado es el único elemento de los números de cuenta cargados en el banco.}
     \end{itemize}
\end{tcolorbox}


\begin{schema}{PedirSaldo\_ SP\_ 2}\\
 cajas : NCTA \pfun SALDO \\
 num? : NCTA 
\where
 num? \in \dom cajas \\
 \dom cajas \neq \{ num? \} \\
 num? \in \dom cajas
\end{schema}

\begin{tcolorbox}[colback=gray!5!white,colframe=gray!50!black,
  colbacktitle=gray!75!black,title=PedirSaldo\_SP\_2]
  Se intenta consultar el saldo de un cliente, cuando:
     \begin{itemize}
        \item[--]{El número de cuenta indicado pertenece a los números de cuenta cargados en el banco.}
        \item[--]{Los números de cuenta cargados en el banco no son iguales al conjunto formado por el número de cuenta indicado.}
        \item[--]{El número de cuenta indicado pertenece a los números de cuenta cargados en el banco.}
     \end{itemize}
\end{tcolorbox}


\begin{schema}{Depositar\_ SP\_ 3}\\
 cajas : NCTA \pfun SALDO \\
 num? : NCTA \\
 m? : \num 
\where
 num? \in \dom cajas \\
 m? > 0 \\
 cajas \neq \{ \} \\
\end{schema}

\begin{tcolorbox}[colback=gray!5!white,colframe=gray!50!black,
  colbacktitle=gray!75!black,title=Depositar\_SP\_3]
  Se intenta depositar dinero en una cuenta, cuando:
     \begin{itemize}
        \item[--]{El número de cuenta indicado pertenece a los números de cuenta cargados en el banco.}
        \item[--]{El monto a depositar es positivo.}
        \item[--]{Existe al menos un elemento en las cajas de ahorro existentes en el banco.}
     \end{itemize}
\end{tcolorbox}


\begin{schema}{Depositar\_ SP\_ 4}\\
 cajas : NCTA \pfun SALDO \\
 num? : NCTA \\
 m? : \num 
\where
 num? \in \dom cajas \\
 m? > 0 \\
 cajas \neq \{ \} \\
 \dom cajas = \dom \{ num? \mapsto ( cajas~num? + m? ) \}
\end{schema}

\begin{tcolorbox}[colback=gray!5!white,colframe=gray!50!black,
  colbacktitle=gray!75!black,title=Depositar\_SP\_4]
  Se intenta depositar dinero en una cuenta, cuando:
     \begin{itemize}
        \item[--]{El número de cuenta indicado pertenece a los números de cuenta cargados en el banco.}
        \item[--]{El monto a depositar es positivo.}
        \item[--]{Existe al menos un elemento en las cajas de ahorro existentes en el banco.}
        \item[--]{El número de cuenta indicado es el único elemento de los números de cuenta cargados en el banco.}
     \end{itemize}
\end{tcolorbox}


\begin{schema}{Depositar\_ SP\_ 5}\\
 cajas : NCTA \pfun SALDO \\
 num? : NCTA \\
 m? : \num 
\where
 num? \in \dom cajas \\
 m? > 0 \\
 cajas \neq \{ \} \\
 \dom \{ num? \mapsto ( cajas~num? + m? ) \} \subset \dom cajas
\end{schema}

\begin{tcolorbox}[colback=gray!5!white,colframe=gray!50!black,
  colbacktitle=gray!75!black,title=Depositar\_SP\_5]
  Se intenta depositar dinero en una cuenta, cuando:
     \begin{itemize}
        \item[--]{El número de cuenta indicado pertenece a los números de cuenta cargados en el banco.}
        \item[--]{El monto a depositar es positivo.}
        \item[--]{Existe al menos un elemento en las cajas de ahorro existentes en el banco.}
        \item[--]{El número de cuenta indicado pertenece a los números de cuenta cargados en el banco.}
     \end{itemize}
\end{tcolorbox}


\begin{schema}{Depositar\_ SP\_ 6}\\
 cajas : NCTA \pfun SALDO \\
 num? : NCTA \\
 m? : \num 
\where
 num? \in \dom cajas \\
 m? > 0 \\
 cajas \neq \{ \} \\
 ( \dom cajas \cap \dom \{ num? \mapsto ( cajas~num? + m? ) \} ) = \{ \}
\end{schema}

\begin{tcolorbox}[colback=gray!5!white,colframe=gray!50!black,
  colbacktitle=gray!75!black,title=Depositar\_SP\_6]
  Se intenta depositar dinero en una cuenta, cuando:
     \begin{itemize}
        \item[--]{El número de cuenta indicado pertenece a los números de cuenta cargados en el banco.}
        \item[--]{El monto a depositar es positivo.}
        \item[--]{Existe al menos un elemento en las cajas de ahorro existentes en el banco.}
        \item[--]{El número de cuenta indicado no pertenece a los números de cuenta cargados en el banco.}
     \end{itemize}
\end{tcolorbox}


\begin{schema}{Depositar\_ SP\_ 7}\\
 cajas : NCTA \pfun SALDO \\
 num? : NCTA \\
 m? : \num 
\where
 num? \in \dom cajas \\
 m? > 0 \\
 cajas \neq \{ \} \\
 \dom cajas \subset \dom \{ num? \mapsto ( cajas~num? + m? ) \}
\end{schema}

\begin{tcolorbox}[colback=gray!5!white,colframe=gray!50!black,
  colbacktitle=gray!75!black,title=Depositar\_SP\_7]
  Se intenta depositar dinero en una cuenta, cuando:
     \begin{itemize}
        \item[--]{El número de cuenta indicado pertenece a los números de cuenta cargados en el banco.}
        \item[--]{El monto a depositar es positivo.}
        \item[--]{Existe al menos un elemento en las cajas de ahorro existentes en el banco.}
        \item[--]{Los números de cuenta cargados en el banco están incluidos en el conjunto formado por el número de cuenta indicado.}
     \end{itemize}
\end{tcolorbox}


\begin{schema}{Depositar\_ SP\_ 8}\\
 cajas : NCTA \pfun SALDO \\
 num? : NCTA \\
 m? : \num 
\where
 num? \in \dom cajas \\
 m? > 0 \\
 cajas \neq \{ \} \\
 ( \dom cajas \cap \dom \{ num? \mapsto ( cajas~num? + m? ) \} ) \neq \{ \} \\
 \lnot \dom \{ num? \mapsto ( cajas~num? + m? ) \} \subseteq \dom cajas \\
 \lnot \dom cajas \subseteq \dom \{ num? \mapsto ( cajas~num? + m? ) \}
\end{schema}

\begin{tcolorbox}[colback=gray!5!white,colframe=gray!50!black,
  colbacktitle=gray!75!black,title=Depositar\_SP\_8]
  Se intenta depositar dinero en una cuenta, cuando:
     \begin{itemize}
        \item[--]{El número de cuenta indicado pertenece a los números de cuenta cargados en el banco.}
        \item[--]{El monto a depositar es positivo.}
        \item[--]{Existe al menos un elemento en las cajas de ahorro existentes en el banco.}
        \item[--]{El número de cuenta indicado no pertenece a los números de cuenta cargados en el banco}
        \item[--]{Existe al menos un elemento en los números de cuenta cargados en el banco que no está en el conjunto formado por el número de cuenta indicado.}
     \end{itemize}
\end{tcolorbox}

\section*{Ejemplo: Sistema de sensores}

\subsection*{Especificación y designaciones}

\begin{itemize}
  \item $x$ es un identificador de sensor válido $\approx x \in SENSOR$
\end{itemize}

\begin{zed}
[SENSOR]
\end{zed}

\begin{itemize}
  \item Conjunto de identificadores válidos $\approx dom~smax$ \\
  \item $x$ es un identificador válido $\approx x \in dom~smax$ \\
  \item Valor máximo registrado para $x$ $\approx smax~x$ 
\end{itemize}

\begin{zed}
MaxReadings == [smax: SENSOR \pfun \num]
\end{zed}

\begin{itemize}
  \item Identificador del sensor leído $\approx s?$ \\
  \item Valor de medición leído $\approx r?$ 
\end{itemize}

\begin{schema}{KeepMaxReadingOk}
\Delta MaxReadings \\
s?:SENSOR; r?:\num \\
\where
s? \in \dom smax \\
smax~s? < r?\\
smax' = smax \oplus \{s? \mapsto r?\}
%smax' = (\dom \{s? \mapsto r?\} \ndres smax) \oplus \{s? \mapsto r?\}
\end{schema}

\begin{zed}
KeepMaxReadingE1 == [\Xi MaxReadings; s?:SENSOR | s? \notin \dom smax]
\end{zed}

\begin{schema}{KeepMaxReadingE2}
\Xi MaxReadings \\
s?:SENSOR; r?:\num
\where
s? \in \dom smax\\
r? \leq smax~s? 
\end{schema}

\begin{itemize}
  \item Intenta actualizar un valor máximo sensado $\approx KeepMaxReading$
\end{itemize}

\begin{zed}
KeepMaxReading == KeepMaxReadingOk \lor KeepMaxReadingE1 \lor KeepMaxReadingE2
\end{zed}

\subsection*{Clases de prueba y descripciones}

\begin{schema}{KeepMaxReading\_ SP\_ 3}\\
  smax : SENSOR \pfun \num \\
  s? : SENSOR \\
  r? : \num
\where
  s? \in \dom smax \\
  smax~s? < r? \\
  smax~s? < 0 \\
  r? > 0
\end{schema}

\begin{tcolorbox}[colback=gray!5!white,colframe=gray!50!black,
  colbacktitle=gray!75!black,title=KeepMaxReading\_SP\_3]
  Se intenta actualizar un valor máximo sensado, cuando:
     \begin{itemize}
  	    \item[--]{El identificador del sensor leído pertenece al conjunto de identificadores válidos.}
  	    \item[--]{El valor máximo registrado para el identificador del sensor leído es menor a el valor de medición leído.}
	      \item[--]{El valor máximo registrado para el identificador del sensor leído es negativo.}
	      \item[--]{El valor de medición leído es positivo.}
     \end{itemize}
\end{tcolorbox}

\begin{schema}{KeepMaxReading\_ SP\_ 4}\\
  smax : SENSOR \pfun \num \\
  s? : SENSOR \\
  r? : \num
\where
  s? \in \dom smax \\
  smax~s? < r? \\
  smax~s? = 0 \\
  r? > 0
\end{schema}

\begin{tcolorbox}[colback=gray!5!white,colframe=gray!50!black,
  colbacktitle=gray!75!black,title=KeepMaxReading\_SP\_4]
  Se intenta actualizar un valor máximo sensado, cuando:
     \begin{itemize}
        \item[--]{El identificador del sensor leído pertenece al conjunto de identificadores válidos.}
        \item[--]{El valor máximo registrado para el identificador del sensor leído es menor a el valor de medición leído.}
        \item[--]{El valor máximo registrado para el identificador del sensor leído es igual a cero.}
        \item[--]{El valor de medición leído es positivo.}
     \end{itemize}
\end{tcolorbox}


\begin{schema}{KeepMaxReading\_ SP\_ 7}\\
  smax : SENSOR \pfun \num \\
  s? : SENSOR \\
  r? : \num
\where
  s? \notin \dom smax \\
  smax~s? < 0 \\
  r? = 0
\end{schema}

\begin{tcolorbox}[colback=gray!5!white,colframe=gray!50!black,
  colbacktitle=gray!75!black,title=KeepMaxReading\_SP\_7]
  Se intenta actualizar un valor máximo sensado, cuando:
     \begin{itemize}
        \item[--]{El identificador del sensor leído no pertenece al conjunto de identificadores válidos.}
        \item[--]{El valor máximo registrado para el identificador del sensor leído es negativo.}
        \item[--]{El valor de medición leído es igual a cero.}
     \end{itemize}
\end{tcolorbox}

\begin{schema}{KeepMaxReading\_ SP\_ 14}\\
  smax : SENSOR \pfun \num \\
  s? : SENSOR \\
  r? : \num
\where
  s? \in \dom smax \\
  r? \leq smax~s? \\
  smax~s? = 0 \\
  r? > 0
\end{schema}

\begin{tcolorbox}[colback=gray!5!white,colframe=gray!50!black,
  colbacktitle=gray!75!black,title=KeepMaxReading\_SP\_14]
  Se intenta actualizar un valor máximo sensado, cuando:
     \begin{itemize}
        \item[--]{El identificador del sensor leído pertenece al conjunto de identificadores válidos.} 
        \item[--]{El valor de medición leído es menor o igual a el valor máximo registrado para el identificador del sensor leído.}
        \item[--]{El valor máximo registrado para el identificador del sensor leído es positivo.}
        \item[--]{El valor de medición leído es positivo.}
     \end{itemize}
\end{tcolorbox}