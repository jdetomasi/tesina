\chapter{Realización de superficie}
\label{cap:realization}
En los capítulos anteriores vimos los procesos que necesitan ser llevados a cabo para lograr una especificación del texto a generar. Finalmente, en esta última etapa nos concentraremos en estudiar las tareas que deben ser realizadas a fin de transformar esta especificación en texto de superficie, formado por frases, símbolos de puntuación y algunas etiquetas de mark-up necesarias.

En las próximas secciones introduciremos las tareas que deberemos llevar a cabo durante esta etapa, repasaremos brevemente la entrada y salida de la misma y finalmente estudiaremos en detalle las tareas de \emph{realización estructural} y \emph{realización lingüística} realizadas en este trabajo.

%Como mencionamos en el capítulo~\ref{cap:microplanning}, nuestra especificación del texto es un árbol en el que las hojas representan las frases individuales y los nodos internos establecen como éstas están agrupadas (según estructuras como párrafos, secciones, listas de ítems, etc). En base a esta diferenciación entre los elementos de nuestra especificación del texto podemos distinguir dos grandes aspectos dentro de esta tarea: la \emph{realización estructural} responsable de transformar los nodos internos de nuestra especificación de texto en anotaciones particulares del sistema de presentación y por otro lado, la tarea de \emph{realización lingüística} que se concentrará en generar frases sintáctica, morfológica, ortográficamente correctas.
%Además, como señalamos en el capítulo~\ref{sec:corpus_analisis}, deberemos agregar un artículo como determinante en el caso de ser necesario. 

\section{Tareas del realizador de superficie}

Como mencionamos en el capítulo \ref{cap:nlg_intro} el realizador de superficie será el encargado de transformar la especificación del texto producida en la etapa de \textit{microplanning} en el documento final. Deberá transformar esta representación abstracta del documento en frases, símbolos de puntuación y algunas etiquetas de mark-up necesarias (estas últimas dependerán del sistema de presentación utilizado).

En el capítulo anterior vimos que la especificación del documento es un árbol, cuyas hojas son especificaciones de frases y sus nodos internos representan cómo las anteriores deberán estar agrupadas en el documento final (en general estos nodos representan: secciones, párrafos, listas de ítem, etc). Para transformar esta especificación en texto de superficie nuestro realizador deberá llevar a cabo dos tareas: \emph{realización estructural} y \emph{realización lingüística}. La primera se encargará de \emph{mapear} los nodos internos antes mencionados en anotaciones para el sistema de presentación a utilizar. Por otro lado, la realización lingüística deberá convertir las especificaciones de frase en oraciones sintáctica, morfológica y ortográficamente correctas.

En la sección \ref{cap:structure_realization} veremos en detalle la tarea de \emph{realización estructural} que llevaremos a cabo en este trabajo y al final de este capítulo, en la sección \ref{cap:linguistic_realization}, estudiaremos las tareas involucradas en la \emph{realización lingüística} de nuestro sistema de NLG.

\section{Entrada y salida del realizador de superficie}

El objetivo del realizador de superficie será transformar la especificación abstracta del documento, producida por la etapa de \textit{microplanning} en texto de superficie. Este texto estará formado por anotaciones dependientes del sistema de presentación y oraciones generadas por la tarea de realización lingüística.

Será la tarea de realización lingüística la encargada de que en el documento final tengamos una oración por cada especificación de frase incluida en la especificación del texto. Por otro lado, las anotaciones para el sistema de presentación serán generadas por la realización estructural, siendo entonces ésta claramente dependiente del sistema de presentación utilizado. Por ejemplo, si nuestro sistema debe producir un documento \LaTeX para luego ser compilado y generar un PDF, será el realizador de superficie (más específicamente de la tarea de realización estructural) el responsable de producir el código necesario para que luego éste se compile apropiadamente. Por ejemplo, en la figura \ref{fig:text_spec} podemos observar un documento producido para la especificación del texto de la figura \ref{fig:ej_latex}.

\begin{figure}[H]
  \begin{verbatim}
  \documentclass{article}
  \title{Descripción clase de prueba del ejemplo: SymbolTable}
  
  \begin{document}
  \maketitle

  LookUp\_SP\_1: Se busca un símbolo en la tabla.  
  Cuando:
  \begin{itemize}
    \item{El símbolo a buscar pertenece ...}
    \item{El símbolo a buscar es el único ...}   
  \end{itemize}
  
  \end{document}
  \end{verbatim}
  \caption{Texto final en \LaTeX.}
  \label{fig:ej_latex}
\end{figure}

A continuación, en las próximas secciones, estudiaremos las tareas de realización estructural y lingüística realizadas para este trabajo.

\section{Realización estructural}
\label{cap:structure_realization}

%La tarea de realización estructural puede resultar relativamente sencilla, pero es importante que se realice por separado a fin de poder generar textos para distintos sitemas de presentacion utilizando la misma realizacion linguistica. Es decir, separar las cuestiones del sistema de presentacion utilizado de las frases u oraciones generadas por nuestro sistema.

Esta tarea deberá transformar los constructores lógicos existentes en la especificación del texto en constructores del sistema de presentación. La mayoría de los sistemas de procesamiento de texto nos permiten indicar mediante el uso de símbolos o etiquetas la naturaleza de una estructura determinada; luego estas serán procesadas y presentadas de manera apropiada permitiéndole al lector una correcta visualización. Por ejemplo, para crear una lista en \LaTeX, debemos utilizar el entorno \emph{itemize}, para lo cual tenemos que inicializar y terminar el mismo mediante las sentencias \verb|\begin{itemize}| y \verb|\end{itemize}| respectivamente y entre las anteriores utilizar el comando \verb|\item| para indicar los elemento de la lista.

Es evidente que esta etapa será dependiente del sistema de presentación escogido para el documento final y a su vez independiente del proceso de realización lingüística de las oraciones o frases a generar.  Por ejemplo, podemos observar en la figura~\ref{fig:ej_latex} el código \LaTeX resultado de la realización de superficie para la especificación de frase introducida en el capítulo anterior (cap. \ref{fig:text_spec}). En la figura~\ref{fig:ej_html} podemos ver otra realización para la misma especificación de texto pero en lenguaje HTML, donde es posible notar que el texto contenido en ambos ejemplos es exactamente el mismo, diferenciándose únicamente en las anotaciones utilizadas para cada sistema de presentación.

\begin{figure}[H]
  \begin{verbatim}
<html>
<body>
  <h1>Descripción clase de prueba del ejemplo: SymbolTable</h1>

  LookUp_SP_1: Se busca un símbolo en la tabla.  
  <ul>
    <li>Cuando:</li>
    <ul>
      <li>El símbolo a buscar pertenece ...</li>
      <li>El símbolo a buscar es el único ...</li>
    </ul>
  </ul>

</body>
</html>
  \end{verbatim}
  \caption{Texto final en HTML.}
  \label{fig:ej_html}
\end{figure}


A continuación, estudiaremos la tarea de realización estructural para nuestro sistema asumiendo que debemos generar únicamente código \LaTeX para luego ser compilado y convertido a un archivo PDF, resultando esto fácilmente extensible para la generación de código para otros sistemas de presentación. Para esto, tenemos que considerar cómo \emph{mapear} los dos constructores lógicos de nuestra especificación de texto en anotaciones o sentencias de \LaTeX. Para realizar \emph{TSDocumento} deberemos agregar las etiquetas de encabezado necesarias, especificar el tipo del documento, su título y tendremos que delimitar el principio y final del texto a incluir en el documento. Luego deberemos realizar recursivamente los elementos contenidos en el documento (\emph{TSListaItems}) y ubicarlos adecuadamente en el cuerpo del mismo.

\begin{algorithm}
  \caption{Realización superficie}\label{alg:realization}
  \begin{algorithmic}[1]
  \Function {realizar}{TSDocumento $doc$}
  
  \State resultado $\gets$ ``\verb|\documentclass{article}|''
  \State resultado $\gets$ resultado + ``\verb|\title{|'' + \Call{verbalizar}{doc.titulo} + ``\verb|}|''
  
  \State resultado $\gets$ resultado + ``\verb|\begin{document}|''
  \State resultado $\gets$ resultado + ``\verb|\maketitle|''
  
  \For{\textbf{each} desc \textbf{in} doc.descs}
    \State resultado $\gets$ resultado + \Call{realizar\_items}{desc}  
  \EndFor

  \State ret $\gets$ ret + ``\verb|\end{document}|''
  \State \textbf{return} $resultado$
  \EndFunction
  \Statex
  \Function {realizar\_items}{TSListaItems $li$}
    \State resultado $\gets$  \Call{verbalizar}{li.encabezado}
	\State resultado $\gets$ resultado + ``\verb|Cuando:|''
    \State resultado $\gets$ resultado + ``\verb|\begin{itemize}|''
    \For{\textbf{each} ef \textbf{in} li.elementos}
      \State resultado $\gets$ resultado + ``\verb|\item{|'' + \Call{verbalizar}{ef} + ``\verb|}|'' 
    \EndFor  
    \State resultado $\gets$ resultado + ``\verb|\end{itemize}|''
    \State \textbf{return} $resultado$
  \EndFunction
\end{algorithmic}
\end{algorithm}

En el algoritmo anterior podemos ver el pseudocódigo, para la tarea de realización de superficie que nos permitirá transformar nuestras especificaciones de texto en código \LaTeX. Para esto necesitaremos construir primero el código necesario para el preámbulo, aquí es donde definiremos el título del documento, donde podemos notar la llamada a la función \textsc{verbalizar}; esta será la encargada de realizar la tarea de realización lingüística y producir una frase en lenguaje natural en base a una especificación de frase dada para el título. Luego construiremos el cuerpo del documento, procesando una por una las descripciones de las clases de prueba a generar, delegando la responsabilidad de la realización de cada una de estas en la función \textsc{REALIZAR\_ITEMS}. Esta agregará el código necesario para construir la lista de ítems, requiriendo también para este caso la realización lingüística de las especificaciones de frases presentes en cada \emph{TSListaItems}.

Para completar la realización de superficie será necesario definir el comportamiento de la función \textsc{verbalizar} antes mencionada. Esta realizará la tarea de realización lingüística y en la próxima sección nos encargaremos especificar la misma.

%En nuestro caso sólo tenemos que considerar dos elementos: \emph{TSDocumento} y \emph{TSListaItems}.

%En la figura~\ref{fig:ej_latex2} podemos ver un ejemplo del código que deberíamos generar para la realización estructural del documento, la realización de \emph{TSListaItems} se puede apreciar en el ejemplo anterior de la figura~\ref{fig:ej_latex}.


% introducir reglas del leguaje (concordancia)
% explicar lexicalizacion para cada elemento de nuestra especificacion de frase

\section{Realización lingüística}
\label{cap:linguistic_realization}

La realización lingüística desempeñará su tarea al nivel de la oración. La misión de ésta será transformar las especificaciones de frases producidas por la etapa anterior en oraciones bien formadas. Entendiendo por oraciones bien formadas a aquellas que cumplen con las reglas gramaticales de la lengua en la cual se encontrará redactado el documento final (castellana en nuestro caso). La realización lingüística, entonces, consistirá en aplicar alguna caracterización de estas reglas a cada especificación de frase a fin de producir un texto que sea sintáctica, ortográfica y gramaticalmente correcto.

En primer lugar, para que el texto producido sea sea ortográficamente correcto, asumiremos que las designaciones provistas por el usuario son ortogáficamente correctas y por nuestro lado, nos deberemos comprometer a que las palabras generadas por nuestro sistema también lo sean. Además de esto, deberemos encargarnos de que la primer palabra de una oración comience con mayúscula y de colocar un signo de puntuación al final de la misma.

En el capítulo \ref{sec:corpus_gramatica} introdujimos algunas consideraciones sobre concordancia gramatical, las cuales deberán ser tenidas en cuenta por nuestra tarea de realización lingüística. A continuación, describiremos las reglas de concordancia mencionadas, para luego, al finalizar el capitulo definir la realización lingüística para los distintos elementos de nuestra especificación de frase.

\medskip
\noindent
\textbf{\emph{Concordancia entre artículo y sustantivo.}} Establece que el artículo debe concordar en género y número con el sustantivo al que acompaña.
\noindent
Como mencionamos anteriormente, el usuario podría no incluir el artículo en el texto utilizado en una designación. Por ejemplo, podríamos haber omitido el artículo en una de las designaciones utilizadas en los capítulos anteriores (figura \ref{fig:ej_designacion}) y en lugar de designar:

\begin{figure}[H]
	\center
    $s? \approx \text{el símbolo a buscar}$
\end{figure}

\noindent
haber designado:

\begin{figure}[H]
	\center
    $s? \approx \text{símbolo a buscar}$
\end{figure}

Luego, para verbalizar la expresión $s? \in \dom st$, por ejemplo, necesitaremos incluir el artículo en el caso que no se encuentre presente:

\begin{figure}[H]
	\center
    $verb(s? \in \dom st) \rightarrow$ \emph{``\textbf{El} símbolo a buscar pertenece a los símbolos cargados en la tabla de símbolos''}.
\end{figure}

\noindent
Será entonces nuestro sistema el encargado de agregar este artículo en el caso de ser necesario\footnote{Podría no ser necesario, por ejemplo si el núcleo de la frase designada fuese un nombre propio.}, asegurándose de que el mismo concuerde con el sustantivo. Para esto deberemos ser capaces de identificar el género y número del sustantivo\footnote{Esto podríamos realizarlo mediante el uso de un analizador morfológico que determinará la forma, clase o categoría gramatical de cada palabra de una oración.}.

\medskip
\noindent
\textbf{\emph{Concordancia entre sujeto y atributo de verbo copulativo.}} Establece que el atributo de un verbo copulativo (ser, estar, parecer) debe concordar en género y número con el núcleo del sujeto de la oración. Veamos por ejemplo el siguiente caso incluido en la definición presentada en el capítulo \ref{sec:corpus_reglas}:

\begin{mdframed}[style=codebox]
\begin{minted}[escapeinside=@@]{haskell}
verb' (@$exp_1 \subset exp_2$@)
    | (gen == M && num == S) = verb(@$exp_1$@) + 
                               "esta incluido en" + 
                               verb(@$exp_2$@) 
    | (gen == M && num == P) = verb(@$exp_1$@) + 
                               "estan incluidos en" + 
                               verb(@$exp_2$@) 
    | (gen == F && num == S) = verb(@$exp_1$@) + 
                               "esta incluida en" + 
                               verb(@$exp_2$@) 
    | (gen == F && num == P) = verb(@$exp_1$@) + 
                               "estan incluidas en" + 
                               verb(@$exp_2$@) 
    where gen = genero(@$exp_1$@)
          num = numero(@$exp_1$@)
\end{minted}
\end{mdframed}

\noindent
Donde podemos ver que tenemos el verbo copulativo \emph{``estar''} acompañado del atributo \emph{``incluido''}. En este caso deberemos escoger el género y número del atributo de forma tal que concuerde con el sujeto.

%TODO agregar en el capitulo del corpus que las frases estarán siempre en tercera persona
\medskip
\noindent
\textbf{\emph{Concordancia entre sujeto y verbo.}} El verbo debe concordar con el sujeto en número y persona. En el caso de haber varios sujetos, la concordancia debe hacerse con el verbo en plural. Como vimos durante el análisis de requerimientos en el capítulo \ref{cap:corpus}, las frases a generar por nuestro sistema se encontrarán siempre en tercera persona por lo cual no será necesario que nuestro sistema deba preocuparse por la concordancia de persona entre sujeto y verbo de la oración. Por lo tanto nuestro realizador deberá asegurarse únicamente de que el número del verbo y el sujeto concuerden.
\noindent
Veamos, por ejemplo, el siguiente caso también incluido en la definición presentada en el capítulo \ref{sec:corpus_reglas}, que el verbo \emph{``pertenecer''} deberá concordar en número y persona con el número del sujeto (en este caso $exp_1$). 

\begin{mdframed}[style=codebox]
\begin{minted}[escapeinside=@@]{haskell}
verb' (@$exp_1 \in exp_2$@)
    | (num == S) = verb(@$exp_1$@) + "pertenece a" + verb(@$exp_2$@) 
    | (num == P) = verb(@$exp_1$@) + "pertenecen a" + verb(@$exp_2$@) 
    where num = numero(@$exp_1$@)
\end{minted}
\end{mdframed}

\bigskip
Otra cuestión a observar, además de las reglas de concordancia ya vistas, es que nuestro sistema deberá generar casi exclusivamente oraciones bimembres ordenadas como:

\begin{center}
\emph{\textbf{sujeto}} + \emph{\textbf{verbo}} + \emph{\textbf{objeto}}
\end{center}

Nuestro realizador deberá siempre respetar este orden de palabras para realizar los elementos de tipo \emph{Oracion} de nuestra especificación de frase. 

Debemos aclarar que en la definición introducida en el capítulo \ref{sec:corpus_reglas} podemos observar oraciones en las cuales el orden de palabras difiere del mencionado anteriormente, por ejemplo, la frase:

\begin{center}
\emph{``no hay elementos en la tabla de símbolos''}
\end{center}

\noindent
cuyo orden es: 

\begin{center}
\emph{verbo} + \emph{sujeto} + \emph{objeto}
\end{center}

En estos casos no modelaremos estas frases utilizando elementos de tipo \emph{Oracion}. En estas situaciones en particular, conocemos de antemano el sujeto y el verbo de las oraciones a generar por lo que no tendremos la necesidad de trabajar con los mismos a fin de asegurarnos concordancia gramatical entre estos. Es por esto que, podremos modelar estas frases, no mediante el elemento \emph{Oracion} presentado en el capítulo anterior. Por ejemplo, modelaremos la frase presentada anteriormente mediante un \emph{ElementosYuxtapuestos} entre una \emph{FraseEnlatada} con el texto ``no hay elementos en'' y la especificación de frase correspondiente al objeto al que hace referencia, para el ejemplo anterior será una \emph{FraseNominal} utilizada para modelar el texto ``la tabla de símbolos'' proveniente de las designaciones de la especificación.

%hablar un poco sobre tiempo verbal
%Introducir acá con ejemplos las reglas sobre concordancia necesarias
%TODO agregar ejemplos para cada frase
%TODO hablar algo de negacion?

\bigskip
En base a los aspectos mencionados anteriormente analizaremos la tarea de realización lingüística para cada elemento de nuestra especificación de frase, esta tarea será recursiva sobre la estructura de las especificaciones antes mencionadas. Para ilustrar el comportamiento de la misma presentaremos una definición en pseudocódigo para cada posible elemento de una especificación de frase. Llamaremos \textsc{verbalizar} a la función encargada de llevar a cabo esta tarea; será la encargada de generar una oración sintáctica y gramaticalmente correcta  en base a una especificación de frase dada. Ésta resultará la función principal de nuestro realizador lingüístico dando como resultado una oración a la que sólo deberemos realizarle algunas pequeñas modificaciones para satisfacer los aspectos ortográficos antes mencionados y dar por finalizada la tarea de realización lingüística.

A continuación analizaremos la realización lingüística de cada uno de los elementos que pueden formar parte de una especificación de frase.

\medskip
\noindent
\textbf{\emph{FraseEnlatada.}} La realización de una frase enlatada resulta trivial, simplemente habrá que extraer el texto contenido en la misma sin necesidad de realizar ningún tipo de procesamiento.

\begin{algorithm}[H]
\caption{Realización lingüística frase enlatada.}
\begin{algorithmic}[1]
\Function {verbalizar}{FraseEnlatada frase}
\State \textbf{return} frase.texto
\EndFunction
\end{algorithmic}
\end{algorithm}


\medskip
\noindent
\textbf{\emph{ElementosYuxtapuestos.}} Representa una concatenación de frases. El resultado de la verbalización deberá ser un texto que resulte de la unión de las verbalizaciones individuales de cada uno de los elementos contenidos, agregando los espacios correspondientes entre estos\footnote{Suponemos que la función \textsc{concat} además de concatenar los elementos agrega un espacio entre cada uno de éstos.} y respetando el orden en que se encuentren. 

\begin{algorithm}[H]
\caption{Realización lingüística elementos yuxtapuestos.} 
\begin{algorithmic}[1]
\Function {verbalizar}{ElementosYuxtapuestos elem}
\For{\textbf{each} e \textbf{in} elem.elementos}
\State resultado $\gets$ \Call{concat}{resultado, \Call{verbalizar}{e}}
\EndFor
\State \textbf{return} resultado
\EndFunction
\end{algorithmic}
\end{algorithm}

\medskip
\noindent
\textbf{\emph{ElementosCoordinados.}} Se trata de una serie de elementos que deberán ser verbalizados individualmente y unidos, de una forma similar a la anterior, a fin de obtener una conjunción de frases.

\begin{algorithm}[H]
\caption{Realización lingüística elementos yuxtapuestos.}
\begin{algorithmic}[1]
\Function {verbalizar}{ElementosCoordinados elem}
\For{\textbf{each} e \textbf{in} elem.elementos}
\If{\Call{es\_primer\_elemento}{e, elem}}
\State resultado $\gets$ \Call{verbalizar}{e}
\ElsIf{\Call{es\_ultimo\_elemento}{e, elem}}
\State resultado $\gets$ \Call{concat}{resultado, \emph{``y''}, \Call{verbalizar}{e}}
\Else
\State resultado $\gets$ \Call{concat}{resultado, \emph{``,''}, \Call{verbalizar}{e}}
\EndIf
\EndFor
\State \textbf{return} resultado
\EndFunction`
\end{algorithmic}
\end{algorithm}

\medskip
\noindent
\textbf{\emph{FraseNominal.}} Para verbalizar una frase nominal, deberemos unir en orden: \emph{determinante} (de ser requerido), \emph{núcleo} y verbalizando recursivamente su \emph{complemento}, agregando los espacios correspondientes entre medio.

\begin{algorithm}[H]
\caption{Realización lingüística FraseNominal.}
\begin{algorithmic}[1]
\Function {verbalizar}{FraseNominal frase}
\State nucleo $\gets$ frase.nucleo
\State complemento $\gets$ \Call{verbalizar}{frase.complemento}
\If{\Call{es\_nombre}{nucleo}}
\State determinante $\gets$ \Call{determinar\_articulo}{frase}
\State resultado $\gets$ \Call{concat}{determinante, nucleo, complemento}
\Else
\State resultado $\gets$ \Call{concat}{nucleo, complemento}
\EndIf
\Statex
\State \textbf{return} resultado
\EndFunction
\end{algorithmic}
\end{algorithm}

\noindent
Observemos que se deberá verbalizar recursivamente el complemento de la frase (éste probablemente sea una frase enlatada o una yuxtaposición de éstas). Finalmente se construye un texto con el \emph{determinante}, \emph{núcleo} y la verbalización del \emph{complemento}, respetando el orden antes mencionado.

La función \textsc{determinar\_articulo} deberá recuperar el determinante apropiado para la frase, en el caso de encontrarse ya establecido de antemano tendrá que devolver el mismo y en caso contrario (de ser necesario) deberá determinar el artículo indicado según número y género del núcleo de la frase. Es razonable suponer que el sustantivo utilizado en el texto de una designación, identifica algo conocido y por lo tanto deberá ser un artículo definido el que lo preceda. Los artículos definidos son aquellos que hablan de algo conocido y que se puede identificar, en particular son artículos definidos: el/lo/las/los. Nuestro sistema podrá entonces tener este conjunto de artículos previamente indexado por número y género de la siguiente manera:

\begin{table}[H]
\centering
\begin{tabular}{|c|c|c|c|}
\hline
\multicolumn{2}{|c|}{\textbf{Masculino}} & \multicolumn{2}{c|}{\textbf{Femenino}} \\ \hline
\textbf{Singular}    & \textbf{Plural}   & \textbf{Singular}   & \textbf{Plural}  \\ \hline
el                   & los               & la                  & las              \\ \hline
\end{tabular}
\end{table}

\noindent
y de esta forma determinar según el número y género del sujeto presente en el texto de la designación el artículo apropiado para la misma.

\medskip
\noindent
\textbf{\emph{FraseVerbal.}} No realizaremos un análisis individual de este caso ya que la realización del mismo dependerá de otros elementos dentro de la oración. Al verbalizar un elemento de tipo \emph{Oracion} analizaremos la \emph{FraseVerbal} que corresponde al predicado del mismo.


\medskip
\noindent
\textbf{\emph{Oracion.}} Este es el caso más interesante para nuestro verbalizador. Deberemos generar una oración correcta en base a las reglas de concordancia gramatical antes mencionadas. 

\begin{algorithm}[H]
\caption{Realización lingüística Oracion.}
\begin{algorithmic}[1]
\Function {verbalizar}{Oracion oracion}
\State sujeto $\gets$ \Call{verbalizar}{oracion.sujeto}
\State fverbal $\gets$ oracion.predicado
\State verbo $\gets$ \Call{determinar\_verbo}{fverbal.verbo, oracion.sujeto}
\State atributo $\gets$ \Call{determinar\_atributo}{fverbal.atributo, oracion.sujeto}
\State complemento $\gets$ \Call{verbalizar}{fverbal.complemento}
\State resultado $\gets$ \Call{concat}{sujeto, verbo, atributo, complemento}
\State \textbf{return} resultado
\EndFunction
\end{algorithmic}
\end{algorithm}

%TODO ampliar esto
%ejemplificar aca y el determinar_articulo de arriba?
\noindent
A fin de contemplar estas reglas, le prestaremos especial atención al verbo y al atributo de la frase verbal correspondiente al predicado de la oración, verbalizando recursivamente tanto el sujeto (generalmente una frase nominal producto de una designación) como el complemento de la frase verbal. Veremos en el capítulo \ref{cap:implementacion} cómo nuestro sistema se encargará de agregar información morfológica a las especificaciones; esta será necesaria en esta etapa para, por ejemplo, conocer el número y género del núcleo del sujeto con el fin de poder determinar el la conjugación apropiada para el verbo y el atributo a utilizar.

La función \textsc{determinar\_verbo} será la encargada de conjugar el verbo (que en la especificación de frase se encuentra en su forma canónica, es decir, el infinitivo del verbo) de manera que concuerde en número y persona con el sujeto, mientras que la función \textsc{determinar\_atributo} tendrá la tarea de determinar el atributo de forma que concuerde en número y género también con el sujeto. Finalmente se unirán los constituyentes en el orden mencionado previamente: \emph{sujeto - verbo - predicado}.

Tanto para determinar la conjugación del verbo como el atributo a utilizar deberemos contar con un léxico en el que cada palabra se encuentre indexada por: forma canónica, número y género que nos permita obtener el verbo o atributo correctamente conjugado en base a las características anteriores. En nuestro caso, sabemos de antemano los verbos y atributos utilizados por el sistema, por lo que podremos construir esta relación para los casos necesarios. De esta forma, para la conjugación de los verbos utilizados en las oraciones de nuestro sistema tendremos:

\begin{table}[H]
\centering
\begin{tabular}{|l|l|l|}
\hline
\multicolumn{1}{|c|}{\textbf{Verbo}} & \multicolumn{1}{c|}{\textbf{Singular}} & \multicolumn{1}{c|}{\textbf{Plural}} \\ \hline
\textbf{Estar}                       & \textit{está}                                   & \textit{están}                                \\ \hline
\textbf{Pertenecer}                  & \textit{pertenece}                              & \textit{pertenecen}                           \\ \hline
\textbf{Ser}                         & \textit{es}                                     & \textit{son}                                  \\ \hline
\end{tabular}
\end{table}

\noindent
Por otro lado, usaremos las siguientes relaciones para decidir qué atributo debemos utilizar:

\begin{table}[H]
\centering
\begin{tabular}{l|l|l|l|l|}
\cline{2-5}
                                 & \multicolumn{2}{c|}{\textbf{Masculino}}                                       & \multicolumn{2}{c|}{\textbf{Femenino}}                                        \\ \hline
\multicolumn{1}{|c|}{\textbf{Atributo}} & \multicolumn{1}{c|}{\textbf{Singular}} & \multicolumn{1}{c|}{\textbf{Plural}} & \multicolumn{1}{c|}{\textbf{Singular}} & \multicolumn{1}{c|}{\textbf{Plural}} \\ \hline
\multicolumn{1}{|l|}{\textbf{Igual}}    & \textit{igual}                                  & \textit{iguales}                              & \textit{igual}                                  & \textit{iguales}                              \\ \hline
\multicolumn{1}{|l|}{\textbf{Incluido}} & \textit{incluido}                               & \textit{incluidos}                            & \textit{incluida}                               & \textit{incluidas}                            \\ \hline
\end{tabular}
\end{table}

Por ejemplo, para la realización lingüística de la oración presentada en la figura \ref{fig:phase_spec_ej} del capítulo anterior, de la cual sabremos que el núcleo del sujeto es singular, y el género del mismo es masculino, nuestro realizador deberá determinar el verbo adecuado de la siguiente manera:

\begin{figure}[H]
\centering
\textbf{determinar\_verbo}(\emph{``pertenecer''}, \emph{``símbolo''}) $\rightarrow$ \emph{``pertenece''}
\end{figure}

\noindent
De forma similar deberá comportarse la determinación de atributo en caso de ser de ser necesaria. 

\bigskip
En este capítulo vimos cómo finalmente, a partir de una especificación del documento, nuestro sistema, a través de las tareas de realización estructural y realización lingüística es capaz de producir el documento final formado por frases en lenguaje natural, símbolos de puntuación y anotaciones necesarias para el sistema de presentación. En el próximo capítulo nos concentraremos en estudiar los aspectos más relevantes de la implementación realizada para este trabajo.