\documentclass[spanish,11pt,a4paper]{memoir} % Change font size here (allowable values are 9pt-12pt), change the paper size, specify one or two sided printing and specify whether to show trimming lines
\usepackage[spanish]{babel}
\usepackage[utf8]{inputenc}
\usepackage{amsmath}
\usepackage[]{algorithm}
\makeatletter
\renewcommand{\ALG@name}{Algoritmo}
\renewcommand{\listalgorithmname}{Índice de \ALG@name s}
\makeatother
\usepackage{algpseudocode}
\usepackage{czt}
\usepackage{float}
\usepackage{url}
\raggedbottom
\usepackage{graphicx}
\usepackage{fancyvrb}
%saca los espacios adicionales de las listas tipo itemize
\usepackage{enumitem}
\setlist{nolistsep}
%no identar en la primer linea de cada parrafo!
%\setlength{\parindent}{0cm}
%formateo itemizes
%\renewcommand{\labelitemi}{$\bullet$}
\newtheorem{example}{Ejemplo}[section] % Defines the example environment
\usepackage{minted}
\usemintedstyle{bw}
\usepackage{xcolor}
\definecolor{bg}{rgb}{0.95,0.95,0.95}
\usepackage[framemethod=default]{mdframed}
\mdfdefinestyle{codebox}{backgroundcolor=bg,skipabove=10,linewidth=0}
\usepackage[most]{tcolorbox}

%TODO \usepackage{program}
% ver de emprolijar algoritmo y recuadro

\setsecnumdepth{subsection} % activa la numeracion de subsecciones
\setcounter{tocdepth}{2} %mostrar sub secciones en indice

\newcommand{\nlgtext}[1] {``\emph{#1}''}
\newcommand{\nlgfun}[1] {\texttt{#1}}
\newcommand{\reglaverb}[2]{\item #1 \\ \hspace*{0.56cm} $\rightarrow$ #2}

% fix para 
\renewcommand*\Call[2]{\textproc{#1}(#2)}

\usepackage{hyperref}
\hypersetup{pdftex,colorlinks=true,allcolors=blue}
\usepackage{hypcap}

\begin{document} 

%%%---%%%---%%%---%%%---%%%---%%%---%%%---%%%---%%%---%%%---%%%---%%%---%%%
%   TITLEPAGE
%
%   due to variety of titlepage schemes it is probably better to make titlepage manually
%
%%%---%%%---%%%---%%%---%%%---%%%---%%%---%%%---%%%---%%%---%%%---%%%---%%%
\thispagestyle{empty}

{%%%
\sffamily
\centering
\Large

~\vspace{\fill}

{\huge 
Generación de lenguaje natural a partir de clases de prueba del \textit{test template framework}
}\\[0.6em]
Tesina de Grado\\
Licenciatura en Ciencias de la Computación

\vspace{1.5cm}

{\LARGE
Julian De Tomasi
}

\vspace{1.5cm}

{\tiny Directores}\\
Maximiliano Cristiá\\
Brian Plüss\\

\vspace{2.5cm}

\includegraphics[scale=0.8]{img/unr.png}\\
Departamento de Ciencias de la Computación\\
Facultad de Ciencias Exactas, Ingeniería y Agrimensura\\
Universidad Nacional de Rosario

\vspace{\fill}

Abril 2016

%%%
}%%%

\cleardoublepage
%%%---%%%---%%%---%%%---%%%---%%%---%%%---%%%---%%%---%%%---%%%---%%%---%%%
%%%---%%%---%%%---%%%---%%%---%%%---%%%---%%%---%%%---%%%---%%%---%%%---%%%


\tableofcontents

\cleardoublepage
\listoffigures

\cleardoublepage
\listofalgorithms
\addcontentsline{toc}{chapter}{Índice de algoritmos}

\chapter{Introducción}
\label{introduccion}

\section{Motivación y objetivo general}
% TODO introducir inportancia de designaciones. comentar temas a tratar en cada capitulo ?
El \textit{testing} basado en modelos (abreviado \textbf{MBT} del inglés \emph{``model based testing''}) es una de las técnicas de testing más prometedoras para la verificación de software crítico. Estas metodologías comienzan con un modelo formal (o especificación) del software a testear, y a partir del mismo son generados los casos de prueba.
%La hipótesis fundamental detras del testing basado en modelos es que, un programa es correcto si verifica su especificación, entonces la especificación resulta una excelente fuente para obtener casos de prueba. %Una vez que los casos de prueba son derivados del modelo, estos son refinados al nivel del lenguaje de implementacion y ejecutados. Luego la salida del programa es abstraida al nivel de la especificacion y el modelo es usado nuevamente para verificar si el caso de prueba ha detectado un error.

%TODO reescribir esto
Un caso particular del testing basado en modelos es el \emph{Test Template Framework} (abreviado \textbf{TTF}) descrito por Stocks y Carrington~\cite{stocks}. El TTF utiliza como modelo de entrada una especificación formal escrita en notación \emph{Z}~\cite{spivey} y establece cómo generar \emph{casos de prueba} para las operaciones incluidas en la especificación. 

Esta técnica genera descripciones lógicas, también en lenguaje Z, de los casos de prueba. El TTF propone en primera instancia obtener casos de prueba abstractos a partir de una especificación llamados \emph{clases de prueba} y luego, a partir de los mismos, generar los \emph{casos de prueba concretos}.

Por otro lado, el desarrollo de software crítico usualmente requiere de procesos independientes de validación y verificación. Estos procesos son llevados a cabo por expertos en el dominio de aplicación, quienes usualmente no poseen conocimientos técnicos para leer los casos de prueba (generados mediante el TTF, por ejemplo) y comprender lo que está siendo testeado con los mismos. En estos casos, una descripción en lenguaje natural de cada caso de prueba debería acompañar a los mismos a fin de hacerlos accesibles para los expertos en el dominio.  

Por ejemplo, en la figura~\ref{fig:intro_tcl} podemos observar, a modo ilustrativo, un caso de prueba (generado mediante el TTF) para la operación \emph{Update} perteneciente a una especificación para una tabla de símbolos y luego, en la figura~\ref{fig:intro_tcl1}, una posible descripción en lenguaje natural del mismo. Una tabla de símbolos es una estructura de datos creada y mantenida por un compilador con el fin de almacenar información (ubicación, ámbito, etc.) relativa a los distintos elemento del código fuente como: variables, nombres de funciones, objetos, etc. Ésta provee al menos dos operaciones: búsqueda e inserción (\textit{Update} en nuestro caso será la encargada de insertar y actualizar símbolos en la tabla). En la sección \ref{sec:ej-symbolTable} nos explayaremos más en detalle sobre el funcionamiento de una tabla de símbolos y presentaremos la especificación Z utilizada para el ejemplo.

\begin{figure}[H]
	\centering
  \begin{schema}{Update\_ SP\_ 4\_ TCASE}\\
   Update\_ SP\_ 4 
  \where
   st = \{ ( sym0 , val0 ) \} \\
   s? = sym0 \\
   v? = val0
 \end{schema}
 \caption{Caso de prueba para operación \emph{Update}.}
  \label{fig:intro_tcl}
\end{figure}
 
\begin{figure}[H]
 \begin{tcolorbox}[colback=gray!5!white,colframe=gray!50!black,
  colbacktitle=gray!75!black,title=Update\_SP\_4]
  Se actualiza un símbolo en la tabla, cuando:
     \begin{itemize}
  	    \item[--]{El símbolo a actualizar es el único símbolo cargado en la tabla de símbolos.}
     \end{itemize}
 \end{tcolorbox}
 \caption{Posible descripción en lenguaje natural para \emph{Update\_SP\_4}.}
 \label{fig:intro_tcl1}
\end{figure}

Contar con una descripción en lenguaje natural como la de la figura \ref{fig:intro_tcl1}, sería de gran ayuda para que la persona a cargo de la validación y verificación comprenda lo que está siendo testeado y además, fundamentalmente, para conectar la especificación del caso de prueba con lo que esto significa en la implementación.


En sistemas en los que hay una gran cantidad de casos de prueba, traducir manualmente los mismos podría introducir errores humanos, reduciendo la calidad de las descripciones además de incrementar el costo del testing.

El objetivo de este trabajo, será entonces, desarrollar una solución para la generación automática de descripciones para los casos de prueba generados por el TTF. Para esto, utilizaremos técnicas de \emph{generación de lenguaje natural} (abreviado \textbf{NLG} del inglés  \emph{``natural language generation''}). En particular, seguiremos la metodología más comúnmente aceptada para la construcción de sistemas de NLG, propuesta por Reiter y Dale~\cite{reiter_dale}. Trabajaremos principalmente con las \emph{clases de prueba}, ya que estas son las que contienen la información referente a las alternativas funcionales que se intentan testear mediante cada \emph{caso de prueba} generado. Además deseamos idear una solución independiente del número de operaciones del modelo así como del dominio de aplicación del mismo. Para esto, junto con la utilización de técnicas de NLG, haremos uso de la información contenida en las designaciones~\cite{jackson} (asociaciones entre los elementos de la especificación y elementos que refieren al dominio de la aplicación) que deberían acompañar la especificación formal. 


Como resultado de este trabajo también se realizará la implementación de un prototipo, a desarrollarse en lenguaje Java e integrarse a \emph{Fastest}\footnote{\url{http://www.fceia.unr.edu.ar/~mcristia/fastest-1.6.tar.gz}} (una implementación del TTF desarrollada por Crstiá y Monetti~\cite{fastest1} capaz de generar casos de prueba a partir de una especificación Z) permitiendo la generación de descripciones de casos de prueba interactivamente desde la herramienta. 

\section{Antecedentes}

Se han hecho variados esfuerzos para producir versiones en lenguaje natural de especificaciones formales. Punshon \cite{punshon} usó un caso de estudio para presentar el sistema REVIEW \cite{review}. REVIEW parafraseaba automáticamente especificaciones desarrolladas con Metaview \cite{metaview}, un meta-sistema que facilita la construcción de entornos CASE (\textit{Computer Aided Software Engineering}) para soportar tareas de especificación de software. Coscoy \cite{coscoy} desarrolló un mecanismo basado en la extracción de programas, para generar explicaciones de pruebas formales en el cálculo de construcciones inductivas, implementado en Coq \cite{coq}. Lavoie \cite{lavoie} presentó MODEX, una herramienta que genera descripciones personalizables de las relaciones entre clases en modelos orientados a objetos especificados en el estándar ODL \cite{odl}. Bertani \cite{bertani} describió un enfoque para la traducción de especificaciones formales escritas en una extensión de TRIO \cite{trio} en lenguaje natural controlado, transformando arboles sintácticos de TRIO en arboles sintácticos del lenguaje controlado.  

Cristiá y Plüss \cite{cristia_pluss} un método de generación de lenguaje natural basado en \textit{templates} para la traducción de casos de prueba generados a partir de una especificación Z para un estándar aeroespacial. El trabajo presenta una solución ad-hoc basada en \textit{templates}, donde los \textit{templates} utilizados son dependientes del dominio de aplicación y de la cantidad de operaciones en la especificación. Basado en este trabajo, intentaremos desarrollar una solución independiente del dominio de aplicación y del número de operaciones del sistema. Para esto, trabajaremos fundamentalmente sobre las clases de prueba (que nos permitirán generar mejores descripciones), utilizando también la información contenida en las designaciones para lograr un resultado independiente del dominio.

\section{Alcance del trabajo}

Como mencionamos anteriormente, trabajaremos fundamentalmente a partir de clases de prueba generadas por el TTF escritas en notación Z.
Para esto, tendremos en cuenta un subconjunto del total de los operadores presentes en Z, pudiéndose en el futuro ampliar o extender el mismo. Creemos que el conjunto de operadores escogidos es lo suficientemente abarcativo, permitiéndonos trabajar con una gran variedad de especificaciones y casos de pruebas generados a partir de las mismas.

En la figura~\ref{fig:alcance} podemos ver todos los operadores contemplados para este trabajo. Es decir, nuestro sistema de NLG deberá ser capaz de generar descripciones en lenguaje natural para todas las expresiones formadas a partir de estos operadores (incluyendo todas las expresiones que puedan surgir como combinaciones de los mismos).

\begin{figure}[H]
  \fbox{\begin{minipage}{13 cm}
        \begin{enumerate}[itemsep=0pt]
        \item =
        \item $\neq$
        \item $\in$
        \item $\notin$
        \item $\subset$
        \item $\subseteq$
        \item $\mapsto$
        \item $\{a,...,b\}$
        \item $\cup$
        \item $\cap$
        \item f~x (aplicación de función)
        \item $\dom$
        \item $\ran$
        \item +
        \item -
        \item *
        \item $\div$
        \item $mod$
        \end{enumerate}
  \end{minipage}}
  \caption{Expresiones soportadas}
  \label{fig:alcance}
\end{figure}

\subsection{Estructura de la tesina}
En este capítulo presentamos los objetivos principales para este trabajo. Establecimos cómo uno de los estos, lograr una solución superadora al trabajo realizado por Cristiá y Plüss \cite{cristia_pluss} (único hasta el el momento en trabajar sobre generación de lenguaje natural para Z y el TTF). Puntualmente, a diferencia del anterior, nuestro objetivo será lograr una generación de descripciones en lenguaje natural independiente del dominio de aplicación y de la cantidad de clases y casos de pruebas.

En el próximo capítulo introduciremos conceptos básicos acerca del \textit{test template framework} fundamentales para el desarrollo de este trabajo. Luego, en el capítulo \ref{cap:nlg_intro}, presentaremos la metodología a utilizar para diseñar y construir nuestro sistema de NLG. En el capítulo \ref{cap:corpus} realizaremos un análisis de requerimientos en base al corpus de datos recolectado, que resultará de vital importancia para el desarrollo de las distintas tareas que deberá realizar nuestro sistema - las cuales estudiaremos en profundidad en los capítulos \ref{cap:document_planning}, \ref{cap:microplanning} y \ref{cap:realization}. En el capítulo \ref{cap:implementacion} veremos los aspectos más relevantes de la implementación realizada y finalmente en el capítulo \ref{cap:conclusion} presentaremos las conclusiones de esta tesina, así como también posibles trabajos futuros.
\chapter[\textit{Testing} Basado en Modelos]{Fundamentos del \textit{testing} basado en modelos}
\label{cap:fundamentos}

En este capítulo introduciremos algunos conceptos básicos con los que trabajaremos a lo largo de esta tesina. Presentaremos el \textit{testing} basado en modelos y veremos mediante un ejemplo como derivar clases y casos de prueba a partir de una especificación Z. Además, presentaremos el funcionamiento de Fastest, la herramienta utilizada para generar clases y casos de prueba de manera automática, en la que basaremos todo el desarrollo a realizar como parte de este trabajo. Finalmente estudiaremos el importante rol que cumplirán las designaciones nuestro sistema de NLG, fundamental para la generación de textos independientes del dominio de aplicación.

\section{\textit{Testing} basado en modelos}

La hipótesis fundamental detrás del \textit{testing} basado en modelos (MBT) es que un programa es correcto si verifica su especificación, entonces, la especificación resulta una excelente fuente para obtener casos de prueba. Las técnicas de MBT utilizan la especificación, en primer instancia, para derivar casos de pruebas abstractos (al nivel del modelo). Estos luego deben ser refinados al nivel del lenguaje de implementación y ejecutados por el programa que supuestamente implementa la especificación. Finalmente la salida del programa será abstraída al nivel de la especificación y la misma será utilizada (nuevamente) para verificar si el caso de prueba ha detectado un error. En el esquema de la figura~\ref{fig:proc_mbt} podemos ver el proceso recién mencionado.

\begin{figure}[H]
\begin{center}
\includegraphics[scale=0.25]{img/proc_mbt.png}
\caption{Proceso de \textit{testing} basado en modelos}
\label{fig:proc_mbt}
\end{center}
\end{figure}


El \textit{Test Template Framework} (TTF) descrito por Stocks y Carrington~\cite{stocks} es un método de MBT que permite efectuar un \textit{testing} muy completo de un sistema del cual se posee una especificación Z~\cite{spivey}, utilizando la misma como entrada y estableciendo cómo generar casos de prueba para testear las distintas operaciones incluidas en el modelo.

A continuación introduciremos brevemente el TTF mediante un ejemplo, asumiendo que el lector se encuentra familiarizado con la notación Z\footnote{Introducción a la notación Z: \url{http://www.fceia.unr.edu.ar/asist/z-a.pdf}}.

\subsection{Ejemplo: \emph{Symbol Table}}
\label{sec:ej-symbolTable}

Una tabla de símbolos es una estructura de datos utilizada por un compilador o intérprete durante el proceso de traducción de un lenguaje de programación donde cada símbolo en el código del programa (variables, constantes, funciones, etc.) se asocia con información como la ubicación, tipo de datos, \textit{scope} de variables, etc. 
En general, en una tabla de símbolos se realizan dos operaciones: inserción y búsqueda; la primera para agregar un símbolo a la tabla y la segunda operación nos permitirá recuperar la información correspondiente a un símbolo ya cargado en la misma.


\bigskip
\noindent
\textbf{Tipos elementales.} Es irrelevante para nuestro modelo especificar en detalle la estructura con la cual representaremos el conjunto de símbolos aceptados por el compilador/interprete, así como la información relativa a los mismos. Para esto, podemos abstraer los conjuntos antes mencionados utilizando los siguientes tipos básicos: 

\begin{zed}
[SYM, VAL] \also
REPORT ::= ok | symbolNotPresent
\end{zed}

El tipo básico \emph{SYM} representará el conjunto de todos los símbolos aceptados por el compilador/interprete, mientras que \emph{VAL} abstraerá el conjunto de toda la información que pudiese estar asociada a un símbolo. Por otro lado, \emph{REPORT} será utilizado para modelar la salida de las operaciones incluidas en la especificación. Una operación podrá ser ejecutada exitosamente o fallar (por ejemplo en caso de realizar una búsqueda de un símbolo que no se encuentre presente en la tabla).
 
\bigskip
\noindent
\textbf{Estado de la tabla.} Es natural pensar que la tabla de símbolos establece una relación funcional entre los símbolos aceptados y la información asociada a cada uno de ellos. Por otro lado, el estado de nuestra tabla está formado únicamente por los símbolos cargados y la información de cada uno. Por lo tanto, podemos modelar el conjuntos de estados de la tabla mediante una única función parcial\footnote{Utilizamos una función parcial ya que no todos los símbolos aceptados por el compilador estarán presentes en la tabla.} de la siguiente manera:

\begin{schema}{ST}
st: SYM \pfun VAL
\end{schema}

\bigskip
\noindent
\textbf{Operaciones.} Como mencionamos anteriormente, dos operaciones se realizarán sobre la tabla de símbolos: inserción y búsqueda.

\bigskip
Comenzaremos por modelar la operación de inserción o actualización. Esta operación modificará el estado de la tabla de símbolos; para esto, será necesario brindarle a la operación el símbolo y la información correspondiente al mismo.

Para actualizar la información de un símbolo en la tabla, simplemente actualizaremos la relación funcional con un par ordenado construido a partir del símbolo y la información dada. 

\begin{schema}{Update}
  \Delta ST \\
  s?: SYM \\
  v?: VAL \\
  rep!: REPORT
  \where
  st' = st \oplus \{s? \mapsto v?\} \\
  rep! = ok
\end{schema}

Por último, deberemos especificar la operación encargada de recuperar la información vinculada a un símbolo. Se trata de una operación relativamente simple que no modifica el estado de la tabla de símbolos. Ésta requerirá de un símbolo a buscar y retornará la información vinculada al mismo. A continuación especificaremos el caso exitoso para esta operación:

\begin{schema}{LookUpOk}
\Xi ST \\
s?: SYM \\
v!: VAL \\
rep!: REPORT
\where
s? \in \dom st \\
v! = st~s? \\
rep! = ok
\end{schema} 

Utilizamos la variable de salida \emph{rep!} para para comunicar el éxito o fracaso de la operación, y la variable \emph{v!} para retornar la información solicitada. 

El esquema \emph{LookUpOk} modela solamente el caso en el que el símbolo a buscar se encuentre cargado en la tabla, pero deberemos contemplar también el caso en que se intente buscar un símbolo que no haya sido cargado en la misma. Modelaremos esta situación con el esquema de error \emph{LookUpE}, presentado a continuación:

\begin{schema}{LookUpE}
\Xi ST \\
s?: SYM \\
rep!: REPORT
\where
s? \notin \dom st \\
rep! = symbolNotPresent
\end{schema}

Finalmente, la operación total (que es la que se debe programar) queda definida de la siguiente manera:

\bigskip
$LookUp == LookUpOk \lor LookUpE$

\subsection{\emph{Test Template Framework}}

La propuesta de Stocks y Carrington~\cite{stocks} es utilizar la especificación Z de una operación como fuente de la cual obtener casos de prueba para testear el programa que supuestamente la implementa. La idea se basa en que la especificación del programa contiene todas las alternativas funcionales que el ingeniero consideró imprescindible describir para que el programador implemente el programa correcto. Por lo tanto, para saber si el programa funciona correctamente es necesario probarlo para cada una de estas alternativas funcionales. 

\subsection{Clases y casos de prueba}

Las alternativas funcionales antes mencionadas pueden expresarse como restricciones sobre las variables de entrada y de estado definidas para la especificación. Estas alternativas pueden ser especificadas mediante esquemas Z a los que llamaremos \emph{clases de prueba}. Por ejemplo:

\begin{zed}
   LookUp_{1}^{FND} == [st: SYM \pfun VAL; s?: SYM  | s? \in \dom st] \\
\end{zed}

\noindent
define una clase de prueba para el esquema \emph{LookUp}, introducido anteriormente, donde $st$ y $s?$ son variables de estado y de entrada respectivamente, y $s? \in \dom st$ es la restricción sobre las mismas. 

Por otro lado, necesitaremos buscar valores específicos, o constantes, para las variables de una \emph{clase de prueba} (que reduzcan su restricción a verdadero) a fin de luego refinarla y poder ejecutar el caso de prueba en el programa que implemente la especificación. Llamaremos \emph{caso de prueba} a una tupla de valores para las variables involucradas en la clase de prueba que cumplan con la restricción de la misma. Nos bastará con escoger un único \emph{caso de prueba} para cada clase de prueba generada (observemos que en algunos casos, como por ejemplo para la clase de prueba antes presentada, podríamos encontrar infinitos valores posibles para las variables de la clase de prueba que cumplan con la restricción).

Por otro lado, en muchos casos, necesitaremos definir las constantes necesarias para los tipos involucrados en la \emph{clase de prueba}, por ejemplo, para los tipos $SYM$ y $VAL$ involucrados en la clase de prueba presentada anteriormente, podríamos definir:

\begin{axdef}
sym_{1}: SYM \\
val_{1}: VAL
\end{axdef}

Finalmente, estamos en condiciones de definir un \emph{caso de prueba} correspondiente a la clase de prueba $LookUp_{1}^{FND}$ de la siguiente manera:

\begin{figure}[H]
\center
$LookUp_{TC}^{1} == [LookUp_{1}^{FND}  | st = \{sym_{1} \mapsto val_{1} \} \land s? = sym_{1}]$
\end{figure}

\subsection{Generación de clases de prueba}
\label{sec:tacticas-testing}

En esta sección introduciremos brevemente, por medio de un ejemplo, el proceso del TTF. Continuaremos trabajando sobre el ejemplo introducido anteriormente e intentaremos generar algunos casos de prueba para testear la operación \emph{LookUp}. En particular nos concentraremos en la generación de clases de prueba, que resultarán de vital interés para nuestro trabajo de NLG.

El primer paso del TTF será definir el espacio de entrada (\emph{IS}, por \emph{Input Space}) para la operación. Este será el conjunto definido por todos los posibles valores de entrada y estado de la misma. Por ejemplo, el \emph{IS} para la operación \emph{LookUp} será:

\begin{zed}
  IS == [st: SYM \pfun VAL; s?: SYM]
\end{zed}

En los casos en los que las operaciones son parciales, no tendrá sentido probar el sistema con casos de prueba para los cuales no está definida la operación, es por esto que, a partir del \emph{IS}, el TTF definirá luego el espacio válido de entrada (\emph{VIS}, por \emph{Valid Input Space}). Este será un subconjunto del anterior, formado por los elementos pertenecientes al \emph{IS} que cumplan con la precondición de la operación en cuestión, es decir:

\begin{zed}
  VIS_{Op} == [IS | pre~Op]
\end{zed}

El \emph{VIS} será un subconjunto del \emph{IS} para los cuales tiene sentido testear el programa. En el caso de \emph{LookUp}, su \emph{VIS} es igual a su \emph{IS}, ya que la operación es total. El \emph{IS} y el \emph{VIS} no coincidirían si la operación \emph{LookUp}, por ejemplo, no se hubiera incluido el esquema de error. En ese caso tendríamos:

\begin{zed}
  IS == [st: SYM \pfun VAL; s?: SYM] \\
  VIS_{LookUpOk} == [IS | s? \in \dom st]
\end{zed}

Luego de determinar el \emph{VIS}, el TTF propone dividir el mismo, de modo tal que cada una de estas particiones represente una alternativa funcional distinta de la operación a testear. Estas particiones serán las \emph{clases de prueba}, introducidas anteriormente, y serán el resultado de aplicar distintas tácticas de \textit{testing} sobre el \emph{VIS}. Luego, será posible aplicar nuevas tácticas sobre estas particiones generadas a fin de dividir nuevamente las mismas, obteniendo como resultado nuevas clases de prueba; este proceso se podrá repetir hasta que el ingeniero de \textit{testing} considere que todas las alternativas funcionales importantes de la operación están representadas (cada una de estas alternativas corresponderá a una única clase de prueba). El último paso del proceso será la selección de al menos un \emph{caso de prueba} para cada clase de prueba, que, como vimos anteriormente, consistirá en buscar valores para las variables de la misma que reduzcan su restricción a verdadero. 

A continuación, mostraremos la aplicación de algunas tácticas de \textit{testing} sobre la operación \emph{LookUp}. En primer lugar aplicaremos Forma Normal Disyuntiva (DNF del inglés \emph{``Disjunctive Normal Form''}) y luego aplicaremos la táctica de Partición Estándar (SP del inglés \emph{``Standard Partition''}) a la expresión: ``$s? \in \dom st$''.

\bigskip
\noindent
\textbf{Forma Normal Disyuntiva.} Suele ser la primer táctica que se aplica. Esta expresará la operación como una disyunción de esquemas en los cuales únicamente habrá conjunciones de literales o de negaciones de literales y luego dividirá el \emph{VIS} con las pre-condiciones de cada esquema. 

Continuando con el ejemplo, \emph{LookUp} ya es una disyunción y cada uno de los esquemas que la forman se encuentra en DNF. Por lo tanto, solo deberemos dividir el \emph{VIS} con la precondición de cada uno de ellos, de la siguiente forma:

\begin{zed}
  LookUp_{1}^{DNF} == [VIS_{LookUp} | s? \in \dom st] \\
  LookUp_{2}^{DNF} == [VIS_{LookUp} | s? \notin \dom st]
\end{zed}

De esta forma obtenemos una partición del \emph{VIS}\footnote{Esto no se cumple en todos los casos, podrían ``solaparse'' las particiones obtenidas, pero de cualquier forma lograríamos un cubrimiento funcional básico.}. Además, observemos que si tomamos un caso de prueba para cada una de las clases obtenidas estaríamos probando el sistema en las siguientes situaciones:

\begin{enumerate}
\item Intentar buscar un símbolo cargado en la tabla de símbolos.
\item Intentar buscar un símbolo que no fue cargado previamente en la tabla de símbolos.
\end{enumerate}

\bigskip
\noindent
\textbf{Partición Estándar.} Esta táctica trata con los operadores matemáticos de una operación. Una partición estándar es una partición del dominio del operador en conjuntos llamados \emph{sub-dominios}. En la figura~\ref{ej:partition_in} podemos ver una partición estándar para los operadores $\in$ y $\notin$.

\begin{figure}[H]
\begin{framed}
  \begin{tabular}{l l l l}
    1. $A = \{\}$ & 3. $A = \{b\}$ & 5. $A = \{b, c\}$  & 7. $\{b, c\} \subset A, a \notin A$ \\ 
    2. $A = \{a\}$ & 4. $A = \{a, b\}$ & 6. $\{a, b\} \subset A$ &   \\ 
  \end{tabular}
  \end{framed}
  \caption{Partición estándar para a $\protect\in$ A; se asume que $a, b \text{ y } c$ son tres elementos diferentes}
  \label{ej:partition_in}
\end{figure}

Para aplicar esta táctica primero hay que seleccionar un operador de un predicado incluido en el esquema de la operación a testear.

Siguiendo con el ejemplo anterior, aplicaremos la técnica de partición estándar a la clase de prueba \emph{LookUp\_DNF\_1}. En primer lugar, deberemos seleccionar una aparición de un operador en el esquema de la operación \emph{LookUp}. 
Nosotros elegiremos el operador $\in$ de la expresión ``$s? \in \dom st$''.


El siguiente paso será reemplazar los parámetros formales que aparecen en la descripción de la partición por las expresiones usadas en la especificación.  En particular, para nuestro ejemplo, deberemos reemplazar los parámetros de las expresiones que aparecen en la figura~\ref{ej:partition_in} con las expresiones ``$s?$'' y ``$\dom st$''. Finalmente, a partir de la clase de prueba sobre la cual se quiere aplicar la táctica, tendremos que generar las nuevas particiones para cada uno de los sub-dominios definidos en la partición estándar. En consecuencia, obtendremos las siguientes clases:


\begin{zed}
  LookUp_{1}^{SP} == [LookUp_{1}^{DNF} | \dom st = \{\}] \\
  LookUp_{2}^{SP} == [LookUp_{1}^{DNF} | \dom st = \{s?\}] \\
  LookUp_{3}^{SP} == [LookUp_{1}^{DNF} | \dom st = \{b\}] \\
  LookUp_{4}^{SP} == [LookUp_{1}^{DNF} | \dom st = \{s?, b\}] \\
  LookUp_{5}^{SP} == [LookUp_{1}^{DNF} | \dom st = \{b, c\}] \\
  LookUp_{6}^{SP} == [LookUp_{1}^{DNF} | \{s?, b\} \subset \dom st] \\
  LookUp_{7}^{SP} == [LookUp_{1}^{DNF} | \{b, c\} \subset \dom st \land s? \notin \dom st] \\
\end{zed}

Luego podríamos continuar aplicando nuevas tácticas y particionando aún más el \emph{VIS}. En este caso se trata de una operación relativamente sencilla y consideramos que todas las alternativas funcionales importantes se encuentran representadas mediante las clases de prueba obtenidas.

\subsection{Fastest}
\label{sec:fastest}

\emph{Fastest}\footnote{\url{http://www.fceia.unr.edu.ar/~mcristia/fastest-1.6.tar.gz}} es una herramienta que implementa la teoría del TTF desarrollada en primer instancia por Maximiliano Cristiá y Pablo Rodriguez Monetti~\cite{fastest1}. El desarrollo de la misma fue impulsado para intentar automatizar, lo máximo posible, el proceso de \textit{testing} funcional basado en especificaciones Z. \emph{Fastest} se encuentra implementado mayormente en lenguaje Java y hace uso de las librerías del \textit{framework} CZT\footnote{\url{http://czt.sourceforge.net/}} (\emph{Community Z Tools}) para contar con utilidades relacionadas al lenguaje de especificación Z. 

A continuación, ilustraremos el funcionamiento de \emph{Fastest} mostrando los comandos necesarios para generar las clases de prueba de la sección anterior\footnote{Puede ser necesario modificar la partición estándar para el operador $\in$ utilizada por \emph{Fastest} para obtener los mismos resultados ya que la versión 1.6 de \emph{Fastest} utiliza una Partición Estándar diferente para el operador $\in$.}.


\begin{figure}[H]
\begin{Verbatim}[frame=single,fontsize=\scriptsize]
Fastest version 1.6, (C) 2013, Maximiliano Cristiá
Loading pruning rewrite rules...
Loading pruning theorems...
Fastest> loadspec symbolTable.tex
Loading specification..
Specification loaded.
Fastest> selop LookUp
Fastest> genalltt 
Generating test tree for 'LookUp' operation.
Fastest> addtactic LookUp_DNF_1 SP \in s? \in \dom st
Fastest> genalltt                                    
Fastest> showsch -tcl

\begin{schema}{LookUp\_ DNF\_ 1}\\
 LookUp\_ VIS 
\where
 s? \in \dom st
\end{schema}


\begin{schema}{LookUp\_ SP\_ 1}\\
 LookUp\_ DNF\_ 1 
\where
 \dom st = \{ \}
\end{schema}


\begin{schema}{LookUp\_ SP\_ 2}\\
 LookUp\_ DNF\_ 1 
\where
 \dom st = \{ s? \}
\end{schema}

...
\end{Verbatim}
\caption{Comandos para generación de clases de prueba en \emph{Fastest}}
\label{ej:comandos_fastest}
\end{figure}


En lo que respecta a la generación de lenguaje natural, \emph{Fastest} fue utilizado en el pasado para testear el software a bordo de un satélite~\cite{satelite} y posteriormente se utilizaron técnicas de generación de lenguaje natural basada en \textit{templates} para traducir los casos de prueba obtenidos \cite{cristia_pluss}. La situación propuesta en el trabajo recién mencionado fue una solución ad-hoc, dependiente del dominio de aplicación y de la cantidad de operaciones.


Como mencionamos anteriormente, uno de los objetivos de este trabajo será el de extender la implementación de \emph{Fastest} para permitir al usuario generar descripciones, independientes del dominio de aplicación y de la cantidad de operaciones de las clases de prueba generadas con la herramienta.
El sistema de NLG a desarrollar en este trabajo será íntegramente implementado en Java e integrado como una extensión de \emph{Fastest} (trabajando para esto con la versión 1.6 de la herramienta). En el capítulo \ref{cap:implementacion} veremos algunos de los detalles más relevantes de esta implementación.

\section{Designaciones}
\label{cap:designaciones}

Un modelo formal, como una especificación Z en este caso, es una abstracción de la realidad. Sin embargo, se realiza una especificación para escribir un programa que finalmente es usado en el mundo real. En consecuencia, existe una relación entre el modelo y la realidad.
Normalmente en estos casos, cuando especificamos un sistema formalmente, es una practica común incluir asociaciones entre elementos de la especificación (operaciones, esquemas de estado, variables, constantes, etc.) y elementos que refieran al dominio de aplicación. Estas asociaciones son llamadas \emph{designaciones}~\cite{jackson}.
Sin esta documentación el modelo sería nada más que una teoría axiomática más sin conexión con la realidad. 

Para documentar las designaciones usaremos la sintaxis propuesta por Jackson~\cite{jackson}:

\begin{figure}[H]
  \centering
  \emph{texto informal} $\approx$ \textbf{término\_formal}
\end{figure}

El símbolo $\approx$ demarca la frontera entre el mundo real (a la izquierda) y el mundo formal o lógico (a la derecha). Del lado derecho estará el término formal a designar, este será un elemento de la especificación, mientras que del otro lado tendremos texto informal en lenguaje natural que permitirá reconocer el fenómeno designado.

Continuando con el ejemplo de la tabla de símbolos (sección ~\ref{sec:ej-symbolTable}), podríamos contar (entre otras) con las siguientes designaciones:

\begin{figure}[H]
  \begin{align*} 
    &\text{Símbolo a buscar} && \approx &&&s? \\
    &\text{Información asociada a $x$} && \approx &&&st~x
  \end{align*}
  \caption{Algunas designaciones para \emph{SymbolTable}}
  \label{fig:ej_designacion}
\end{figure}


Para Jackson, las designaciones sirven en primera instancia cuando se empieza a escribir la especificación para diferenciar un fenómeno en particular y darle un nombre. Luego, le será de utilidad al programador a la hora de leer la especificación. Jackson propone construir un \emph{``puente angosto''} entre la especificación y los elementos del dominio, escribiendo la menor cantidad de designaciones posibles y definiendo otros términos en base a las anteriores.


La función que cumplirán las designaciones en éste trabajo difiere un poco de la propuesta por Jackson. Para nosotros las designaciones resultarán la principal fuente de conocimiento para nuestro sistema de NLG. Y serán fundamentales para que éste pueda generar descripciones independientes del dominio de aplicación.

Veamos, por ejemplo, las siguientes expresiones pertenecientes a dos clases de prueba de dos especificaciones distintas:

\begin{figure}[H]
\begin{enumerate}
\item $\dom st = \{ s? \}$
\item $\dom cajas = \{ num? \}$
\end{enumerate}
\end{figure}

La primera expresión pertenece a una clase de prueba generada para la operación \emph{LookUp}, la segunda es parte de una clase de prueba generada para una operación perteneciente a la especificación de un sistema bancario. Estas expresiones resultan equivalentes (de hecho, hasta podríamos haber usado los mismos nombres de variables para ambas especificaciones); será gracias a las designaciones que podremos otorgarles una descripción en lenguaje natural (acorde al dominio de aplicación de cada especificación) a cada una de estas expresiones. En particular, para estas, las descripciones podrían ser las siguientes:

\begin{figure}[H]
\begin{enumerate}
\item \emph{``El símbolo a buscar es el único cargado en la tabla de símbolos.''}
\item \emph{``El número de caja de ahorro ingresado es el único cargado en el banco.''}
\end{enumerate}
\end{figure}

Los sistemas de generación de lenguaje natural generalmente utilizan un diccionario de palabras o frases, las cuales se utilizan para referirse a fenómenos del dominio. En nuestro caso, el dominio de aplicación dependerá de la especificación en cuestión y de lo que se modele con la misma, por lo tanto, las designaciones resultarán nuestra única fuente de textos dependientes del dominio y es por eso que serán un elemento fundamental para nuestro sistema de NLG.

Por otro lado, en algunas situaciones, contar con una mayor cantidad de designaciones nos permitirá generar mejores descripciones. Designaciones que podrían resultar redundantes para una persona que lea la especificación podrían, por ejemplo, permitirle a nuestro sistema de NLG generar textos más naturales. En el apéndice \ref{ape:designaciones} podemos encontrar una pequeña guía sobre qué designar a fin de proveerle la información necesaria a nuestro sistema de NLG para que pueda producir textos más fluidos y naturales.

Por último, cabe mencionar, que es posible que aparezcan parámetros (pertenecientes al término formal) también del lado izquierdo de la designación, como es el caso de la segunda designación presente en la figura~\ref{fig:ej_designacion}. Llamaremos \emph{designaciones parametrizadas} a este tipo de designaciones, y marcaremos esta diferencia ya que deberemos darle un tratamiento especial a fin de utilizar el texto de estas designaciones en nuestro sistema de NLG. En el capítulo~\ref{sec:verbalizacion_designaciones} desarrollaremos más en detalle esta particularidad. 


\section{Resumen del capítulo}
En este capítulo introdujimos conceptos fundamentales para el desarrollo de este trabajo. Presentamos el \textit{test template framework}, definimos las nociones de clase y caso de prueba con las que trabajaremos a lo largo de todo el trabajo. Finalmente profundizamos sobre el rol que cumplirán las designaciones en este trabajo, resultando fundamentales para la generación de descripciones independientes del dominio de aplicación. En el próximo capítulo introduciremos las tareas que deberán ser llevadas a cabo por nuestro sistema de NLG para generar descripciones para las clases de prueba generadas por Fastest.
\chapter{Generación de lenguaje natural.}
\label{cap:nlg_intro}
La generación de lenguaje natural (NLG) es un una rama de la lingüística computacional y la inteligencia artificial encargada de estudiar la construcción de sistemas computacionales capaces de producir texto en español o cualquier otra lengua humana a partir de algún tipo de representación no-lingüística de la información a comunicar. Estos sistema combinan conocimientos tanto del lenguaje en cuestión cómo del dominio de aplicación para producir automáticamente documentos, reportes, mensajes o cualquier otro tipo de textos.

Dentro de la comunidad desarrolladora e investigadora de la NLG hay un cierto consenso sobre la funcionalidad lingüística general de un sistema de NLG.
En este trabajo se optó por seguir la metodología más comúnmente aceptada, propuesta por Reiter y Dale\cite{reiterdale}.
A continuación describiremos brevemente los aspectos mas importantes de esta metodología y en capítulos posteriores desarrollaremos más en profundidad en los puntos mas relevantes para nuestro trabajo.

\section{Análisis de requerimientos}
El primer paso en la construcción de cualquier sistema de software, incluyendo los sistemas de generación de lenguaje natural, será el de realizar un análisis de requerimientos y a partir de ahí generar una especificación inicial del sistema. 

Para el análisis de requerimientos, Reiter y Dale proponen realizar un \emph{corpus} de textos de ejemplo y a partir de ellos obtener una especificación para el sistema a desarrollar. Estos ejemplos estarán compuestos por una colección de datos de entrada del sistema con sus respectivas salidas (texto en lenguaje natural). Estos textos deberán estar redactados por un humano experto y deberían caracterizar todas las salidas posibles que se espera que el sistema genere.

En el capítulo~\ref{cap:corpus} profundizaremos más sobre este tema, describiremos y analizaremos el \emph{corpus de descripciones} utilizado para este trabajo.

\section{Tareas de la generación de lenguaje natural}

Dentro de la comunidad desarrolladora e investigadora de la generación de lenguaje natural, hay cierto consenso sobre las tareas que deben llevarse a cabo para, a partir de los datos de entrada generar texto final en lenguaje natural. 

La más comúnmente aceptada es la clasificación de Reiter y Dale que distingue siete tareas que deben ser realizadas a lo largo de todo el proceso: \emph{determinación del contenido}, \emph{estructuración del documento}, \emph{agregación}, \emph{lexicalización}, \emph{generación de expresiones de referencia}, \emph{realización lingüística} y \emph{realización de la estructura}.

\begin{comment}
\textbf{Determinación del contenido:} es el proceso de determinar que información debe ser comunicada en el texto final. 

\textbf{Estructuración del documento:} es el proceso de imponer un orden y estructura sobre los textos generados. 

\textbf{Agregación:} Es el proceso de agrupar mensajes para formar nuevas frases.

\textbf{Lexicalización:} Es el proceso de decidir que palabras y frases especificas usar para expresar los distintos conceptos y relaciones del dominio.

\textbf{Generación de expresiones de referencia:} es la tarea de elegir que palabras y frases usar para identificar entidades del dominio de aplicación.

\textbf{Realización lingüística:} es el proceso de aplicar reglas gramaticales para producir texto que sea sintáctica, morfológica y ortográficamente correcto.

\textbf{Realización de la estructura:} es la tarea de convertir estructuras abstractas, como párrafos y secciones, en símbolos marcados comprensibles por el componente de presentación del documento.

Hay varias cuestiones que deben abordarse para poder producir un texto a partir de una entrada dada. Reiter y Dale\cite{reiterdale} proponen descomponer este proceso en tres módulos conceptuales: \emph{``documment planning''}, \emph{``microplanning''} y \emph{``realization''}. Y en una categorización más fina se puede dividir el procesamiento realizado por estos módulos en termino de tareas de \emph{determinación de contenido}, \emph{estructuración del documento}, \emph{lexicalización}, \emph{agregación}, \emph{generación de expresiones de referencia}, \emph{realización lingüística} y \emph{realización de estructura}.

\begin{table}[h]
\centering
\begin{tabular}{lll}
\toprule \textsc{Modulo} & \textsc{Tareas de contenido} & \textsc{Tareas estructurales}\\
    
\midrule \emph{Document planning} & Determinación de contenido & Estructuración del documento\\
\emph{Miroplanning} & Lexicalización & Agregación\\
 & Generación de expresiones de referencia & \\
\emph{Realization} & Realización lingüística & Realización de estructura\\
\bottomrule
\end{tabular}
\end{table}

A continuación se describirá el procesamiento realizado por cada una de estas tareas.

\end{comment}

\subsection{Determinación del contenido.}
Esta tarea será la encarga de determinar que información debe ser comunicada en el texto final. 
Generalmente este proceso consiste en el filtrado y resumen de los datos de entrada. 
El filtrado o selección consiste en en la elección de un subconjunto de los datos disponibles para ser comunicados en el texto final.
El resumen, por otro lado, será necesario cuando los datos de la fuente de información son demasiado detallados para comunicarlos directamente o si la información importante es una generalización de los datos dados en lugar de los datos en sí.

En algunos casos se podrán utilizar los datos de entrada tal y como son proporcionados por la fuente de información, pero generalmente, se tendrá que procesar los mismos para poder utilizarlos posteriormente generando alguna estructura interna que permita representar las entidades, conceptos y relaciones del dominio de aplicación.

Hay algunos investigadores como Evans\cite{completar} que no consideran a ésta etapa como una tarea de la NLG, pero indudablemente se trata de un problema que en algún momento hay que resolver.


% TODO completar con referencia al trabjo realizado


\subsection{Estructuración del documento}
La tarea de estructuración del documento consiste en agrupar y ordenar la información a comunicar de forma que el texto resulte coherente y no como una generación de textos ordenados al azar.

Los datos pueden ser agrupados conceptualmente para ser presentados de acuerdo a la información que comunican. Por ejemplo, podríamos agrupar los elementos de modo que en un párrafo se encuentre toda la información referida a un mismo tema y en el siguiente párrafo la información relativa a otro; o podríamos tener un primer párrafo con información general seguido de otro que detalle algún elemento del primero.

También podríamos establecer relaciones retóricas o de discurso entre los elementos o grupos de elementos del texto. Relaciones como \emph{ejemplificación}, \emph{contraste} o \emph{elaboración}, entre otras, podrían relacionar elementos del texto.  

% TODO completar con referencia al trabjo realizado

\subsection{Lexicalización}
La lexicalización es el proceso de elegir las frases y palabras adecuadas con el fin de comunicar la información requerida.
En esta etapa se deberá establecer como se expresa un significado conceptual concreto, descrito en términos del modelo del dominio, usando elementos léxicos en lenguaje natural. Por otro lado, estos elementos podrían estar asociados a más de una frase o palabra. Por ejemplo, un elemento podría estar relacionado con varias palabras sinónimas. En estos casos la lexicalización deberá incluir una subtarea para elegir el término correcto. En sistemas multilingüe, los elementos harán referencia a una palabra en cada lengua.

% TODO acá ayudaría un ejemplo
% TODO completar con referencia al trabjo realizado

\subsection{Generación de expresiones de referencia}
En esta tarea se determina que expresiones se deben usar para referirse o identificar a las distintas entidades del dominio de aplicación.
Una misma entidad del dominio de aplicación podría ser referida de distintas formas. Será la etapa de generación de expresiones de referencia la encargada de elegir que expresiones usar para describir una las mismas de modo que el lector pueda identificarla en un contexto dado. La descripción que se elija para hacer referencia a una entidad por primera vez (referencia inicial) dependerá de la razón por la cual se introduce a esta entidad y que se pretende comunicar en posteriormente en el texto. Si se vuelve a hacer referencia a la entidad después de haber aparecido una vez (referencia posterior) la preocupación será la de poder diferenciarla de las otras entidades con las que se podría confundir, pero sin que resulte un texto poco fluido.

% TODO acá ayudaría un ejemplo
% TODO no repetir tanto la palabra entidad
% TODO completar con referencia al trabjo realizado

\subsection{Agregación}
El proceso de agregación se encarga de combinar varios elementos informativos con el fin de conseguir un texto más fluido y legible. La agregación decide que partes de las estructuras se pueden combinar para realizarlas como oraciones complejas para que se pueda generar un texto conciso, cohesionado y que a la vez el significado del texto se mantenga casi igual que sin agregación. 

% TODO acá ayudaría un ejemplo
% TODO completar con referencia al trabjo realizado

\subsection{Realización lingüística}
Es el proceso de convertir las representaciones abstractas del texto en texto real.
Al igual que los textos no son secuencias de oraciones ordenadas al azar, las oraciones no son secuencias de palabras ordenadas al azar. Cada lengua está definida por un conjunto de reglas gramaticales que especifican lo que es una oración bien formada en esa lengua. Estas reglas determinan tanto la morfología, que se ocupa de como se forman las palabras (genero, numero, etc.), la sintaxis, que trata de cómo se forman las oraciones y la ortografía (TODO completar). La realización lingüística consiste en aplicar alguna caracterización de estas reglas de la gramática a una representación abstracta para producir un texto que sea sintáctica y morfológicamente correcto y ortográficamente correcto.

% TODO completar con referencia al trabjo realizado

\subsection{Realización de la estructura}
Esta etapa se encarga de convertir estructuras abstractas como párrafos y secciones en texto comprensible por el componente de presentación del documento. Por ejemplo, la salida del sistema de NLG podría ser código LaTeX para luego ser post-procesado, en este caso sería esta etapa la encargada de agregar delimitadores y comandos de LaTeX para generar el documento. 

% TODO completar con referencia al trabjo realizado

\section{Arquitectura para la generación de lenguaje natural.}
\include{sec/alcance}
\chapter{Análisis de requerimientos}
\label{cap:corpus}

El primer paso en la construcción de cualquier sistema de software, incluyendo los sistemas de generación de lenguaje natural, será el de realizar un análisis de requerimientos y a partir de ahí generar una especificación inicial del sistema. 

Para el análisis de requerimientos, seguiremos el enfoque propuesto por Reiter y Dale~\cite{reiter_dale} en el cual se propone realizar un \emph{corpus} de textos de ejemplo y a partir de ellos obtener una especificación para nuestro sistema.  

\section{Corpus de descripciones}                 

Este \emph{corpus de textos} constará de una colección de ejemplos, formados por la entrada y la salida esperada de nuestro sistema. En nuestro caso, la entrada de nuestro sistema será: la especificación formal en lenguaje Z, un conjunto de designaciones y un grupo de clases de prueba generadas de antemano, mientras que la salida estará formada por descripciones en lenguaje natural de las clases de prueba antes mencionadas. En lo posible, el \emph{corpus} de textos debe cubrir todo el rango de textos esperados a ser producidos por el sistema de NLG; éste debería cubrir los casos más frecuentes, así como los casos mas inusuales que se puedan dar.

Siguiendo la metodología propuesta por Reiter y Dale, para obtener un \emph{corpus objetivo} deberíamos recolectar un conjunto lo suficientemente amplio de ejemplos a fin de caracterizar la variedad de textos que deseamos generar. Luego una persona capacitada (en nuestro caso sería una persona capaz de leer Z) debería describir en lenguaje natural los ejemplos antes mencionados. Finalmente se debería revisar este \emph{corpus inicial} y modificarlo en caso de existir algún algún texto que resulte técnicamente imposible de generar o prohibitivamente caro a nivel computacional. Una vez finalizado este proceso tendríamos en nuestro poder un \emph{corpus objetivo} que nos servirá para sustentar muchas de las decisiones que tomaremos a lo largo de este trabajo. Ésta colección de ejemplos también nos servirá para realizar una evaluación de nuestra implementación, comparando los textos generados por nuestro sistema con las descripciones del \emph{corpus} realizadas manualmente por una persona.

En el apéndice~\ref{ape:corpus} se pueden consultar los textos incluidos en el \emph{corpus} utilizado para este trabajo. Para elaborar el mismo recolectamos una serie de clases de prueba generados con \emph{Fastest} a partir de distintas especificaciones y luego escribimos manualmente cada una de las descripciones de estas clases de prueba. Con las especificaciones y clases de prueba incluidas en el \emph{corpus}, intentamos abarcar todo el rango de textos que esperamos que nuestro sistema sea capaz de producir, para esto tuvimos en cuenta incluir clases de pruebas que cubran todas las expresiones de Z contempladas dentro del alcance de este trabajo y sus posibles combinaciones. También trabajamos con especificaciones sobre distintos dominios de aplicación con el fin de lograr \emph{corpus} que nos sea de utilidad para dar con una solución independiente del dominio de aplicación. En total trabajamos con clases de prueba generadas con \emph{Fastest} para 10 especificaciones distintas; del total de clases de prueba correspondientes a estas especificaciones se escogieron las más significativas para describir y se ignoraron aquellas que contenían algún operador no considerado dentro del alcance de este trabajo.

La Figura~\ref{fig:ej_desc_lookup_sp_1} muestra a modo de ejemplo una descripción para la clase de prueba \emph{LookUp\_SP\_1} generada a partir de la especificación para una tabla de símbolos introducida anteriormente (pág.~\pageref{fig:spec_symbol_table}). 
El \emph{corpus de descripciones} utilizado para este trabajo, constará entonces de una colección de ejemplos similares a éste.

%TODO agregar designaciones
\begin{figure}[H]
  \centering
   \begin{schema}{LookUp\_ SP\_ 1}\\
  		LookUp\_ VIS 
  		\where
  	 	s? \in \dom st \\
 		\dom st = \{ s? \}
  	\end{schema}
  \caption{Clase de prueba para operación LookUp.}
  \label{fig:ej_lookup_sp_1}
\end{figure}

\begin{figure}[H]
Descripciones SymbolTable 

\bigskip
\textbf{LookUp\_SP\_1:} Se busca un símbolo en la tabla.  
  \begin{itemize}
   \item{Cuando:}
   \begin{itemize}
  	  \item{El símbolo a buscar pertenece a los símbolos cargados en la tabla de símbolos.}
  	  \item{El símbolo a buscar es el único elemento del conjunto formado por los símbolos cargados en la tabla de símbolos.}   
   \end{itemize}
  \end{itemize}
  \caption{Descripción en lenguaje natural para \emph{LookUp\_SP\_1}.}
  \label{fig:ej_desc_lookup_sp_1}
\end{figure}

\section{Análisis del corpus}
\label{sec:corpus_analisis}

Como mencionamos previamente muchas decisiones de diseño de nuestro sistema estarán fundamentados en observaciones obtenidas a partir del \emph{corpus}, mediante un análisis del mismo podremos obtener una especificación para nuestro sistema d NLG.

Lo primero que podemos observar de los textos de ejemplo es la estructura del texto que se mantiene en la descripción de cada clase de prueba. En todas, se comenzará por el nombre de la clase de prueba, seguido por un pequeño detalle de la operación a testear y luego una lista de oraciones, una por cada restricción de la clase de prueba que se está describiendo. Esto nos ayudará en el capítulo~\ref{cap:document_planning} para definir la estructura de nuestro documento. 

El desafío principal de nuestro sistema será describir en lenguaje natural las distintas expresiones Z que pueden aparecer en el cuerpo de los esquemas de una clase de prueba. Nuestro sistema de NLG deberá ser capaz de construir una oración en lenguaje natural a partir de una expresión Z y un conjunto de designaciones. En lo que queda de esta sección nos concentraremos en la tarea de \emph{verbalizar} expresiones Z; definir con precisión los detalles de la misma será de vital importancia para especificar las etapas de \emph{microplanning} y \emph{lexicalización} de nuestro sistema en los capítulos~\ref{cap:microplanning} y \ref{cap:realization}.

Siguiendo con el análisis del \emph{corpus}, podemos notar, como era de esperarse, que los textos que surgen a partir de los operadores de Z se repiten (con pequeñas variantes que analizaremos mas adelante) independientemente del dominio de aplicación; y las diferencias que aparecen entre los mismos se deben principalmente al texto incluido en las designaciones. Por ejemplo, consideremos las siguientes dos expresiones y sus respectivas descripciones en lenguaje natural, una pertenece al ejemplo anterior y la otra la otra forma parte de una clase de prueba para la especificación de un sistema bancario:

\bigskip
\begin{enumerate}
	\item $s? \in \dom$ $\rightarrow$ \emph{``El símbolo a buscar \textbf{pertenece a} los símbolos cargados en la tabla de símbolos.''}
	\item $s? \in \dom$ $\rightarrow$ \emph{``El número de cuenta \textbf{pertenece a} los números de cuenta cargados en el banco.''}
\end{enumerate}

\bigskip
Como vemos, el texto \emph{``pertenece a''} aparece en ambas descripciones como resultado de verbalizar el operador $\in$ de Z y como mencionamos antes, se diferencian en el texto que antecede y precede al anterior en base a descripciones generadas a partir de las designaciones de cada sistema. Podríamos entonces intentar definir la tarea de verbalización mediante una función que tome como entrada una expresión Z y devuelva una descripción en lenguaje natural para la misma. La definición de esta función dependerá del argumento, y para el caso anterior podría ser la siguiente:

\begin{figure}[H]
\center
$verbalizar(x \in y) \rightarrow verbalizar(x) + \text{\emph{``pertenece a''}} + verbalizar(y)$
\end{figure}

A partir de esto podemos ver que para verbalizar términos compuestos de Z, necesitaremos verbalizar recursivamente las partes que los componen, en la definición anterior necesitaríamos conocer las verbalizaciones de las expresiones $x$ e $y$ para poder obtener una verbalización para $x \in y$.

Retomemos el ejemplo de la figura~\ref{fig:ej_lookup_sp_1}, nos concentraremos en verbalizar la primera expresión de la clase de prueba:

\begin{figure}[H]
\center
$s? \in \dom st$
\end{figure}

En este caso, según la verbalización propuesta anteriormente, podríamos verbalizar la expresión como:

\begin{figure}[H]
\center
$verb(s? \in \dom st) \rightarrow verb(s?) + \text{\emph{``pertenece a''}} + verb(\dom st)$
\end{figure}

Teniendo que verbalizar las expresiones $s?$ y $\dom st$. En este caso, ambas expresiones se encuentran designadas. En estas situaciones nuestra tarea de verbalización debería construir la descripción en base al texto presente en la designación y no intentar describir la expresión en base al término en cuestión. En el ejemplo anterior, no deberíamos intentar verbalizar el término $\dom st$ como $\text{\emph{``el dominio''}} + verb(st)$, por ejemplo, ya que estaríamos perdiendo información valiosa para nuestras descripciones contenida en las designaciones. Entonces, será un requerimiento para nuestra tarea de verbalización contemplar en primera instancia si la expresión a describir no se encuentra designada antes de intentar describir el término Z correspondiente; de estar designada, se deberá construir una descripción en base a su designación. 

En la figura~\ref{fig:algoritmo-verbalizacion} podemos ver un algoritmo en pseudocódigo para nuestra tarea de verbalización, siendo la función \emph{verb} la encargada de generar las verbalizaciones para las expresiones de Z en base a un conjunto de reglas dependientes de el término a describir, como la introducida anteriormente para el caso del operador $\in$.

\begin{figure}[H]
\begin{algorithmic}
\Function {verbalizacion}{$exp$}
\If{$esta\_designada(exp)$}
\State $ret\gets \text{designacion}(exp)$
\Else
\State $ret\gets \text{verb}(exp)$
\EndIf
\State \textbf{return} $ret$
\EndFunction
\end{algorithmic}
\caption{Bosquejo verbalización.}
\label{fig:algoritmo-verbalizacion}
\end{figure}

En el bosquejo anterior, abstraemos mediante la función \emph{designacion} la tarea de generar una descripción para una expresión designada en base al texto presente en la misma. Cabe aclarar, que en este caso, tendremos que contemplar si la designación correspondiente es una designación parametrizada. De no ser el caso, el resultado sería exactamente el texto que forma parte de la designación, del contrario, habrá primero que verbalizar el parámetro de la designación, teniendo luego que construir la descripción final a partir del texto de la designación parametrizada y la verbalización del parámetro.
%TODO~\footnote{TODO: aclarar que suponemos que el parámetro debe estar designado?}

Finalmente, presentaremos un conjunto de reglas para la verbalización de expresiones, construido a partir de un análisis de los textos de ejemplo presentes en el \emph{corpus}.

\bigskip
\begin{itemize} [label={$\triangleright$}, leftmargin=*]
\item{$=$}
	\begin{enumerate}	 
	\setlength{\itemsep}{7pt}
		\reglaverb{$\{exp1\} = exp2$}{\nlgfun{verb(exp1)} + \nlgtext{es el único elemento de} + \nlgfun{verb(exp2)}}
		\reglaverb{$exp1$ $=$ $\{\}$}{\nlgtext{no hay ningún elemento en}  + \nlgfun{verb(exp1)}}
		\reglaverb{$exp1 \cap exp2 = \{\}$}{\nlgfun{verb(exp1)} + \nlgtext{y} + \nlgfun{verb(exp2)} + \nlgtext{no tienen ningún elemento en común}}
		\reglaverb{$exp1 \cap \{exp2\} = \{\}$}{\nlgfun{verb(exp2)} + \nlgtext{no pertenece a} + \nlgfun{verb(exp1)}}
		\reglaverb{$exp1 = exp2$ (caso default)}{\nlgfun{verb(exp1)} + \nlgtext{es(son) igual(iguales) a} + \nlgfun{verb(exp2)}}
	\end{enumerate}
\item{$\neq$}
	\begin{enumerate}	 
	\setlength{\itemsep}{7pt}
		\reglaverb{$exp1 \neq \{\}$}{\nlgtext{existe al menos un elemento en} + \nlgfun{verb(exp1)}}
		\reglaverb{$exp1 \cap exp2 \neq \{\}$}{\nlgfun{verb(exp1)} + \nlgtext{y} + \nlgfun{verb(exp2)} + \nlgtext{tienen al menos un elemento en común}}
		\reglaverb{$exp1 \neq exp2$ (caso default)}{\nlgfun{verb(exp1)} + \nlgtext{no es(son) igual(iguales) a} + \nlgfun{verb(exp2)}}
	\end{enumerate}
\item{$<$}
	\begin{enumerate}	 
	\setlength{\itemsep}{7pt}
		\reglaverb{$exp1 < exp2$ (caso default)}{\nlgfun{verb(exp1)} + \nlgtext{es menor a} + \nlgfun{verb(exp2)}}
		\reglaverb{$exp1 < 0$}{\nlgfun{verb(exp1)} + \nlgtext{es negativo}}
	\end{enumerate}
\item{$\leq$}
	\begin{enumerate}	 
	\setlength{\itemsep}{7pt}
		\reglaverb{$exp1 \leq exp2$ (caso default)}{\nlgfun{verb(exp1)} + \nlgtext{es menor o igual a} + \nlgfun{verb(exp2)}}
	\end{enumerate}
\item{$>$}
	\begin{enumerate}	 
	\setlength{\itemsep}{7pt}
		\reglaverb{$exp1 > exp2$ (caso default)}{\nlgfun{verb(exp1)} + \nlgtext{es mayor a} + \nlgfun{verb(exp2)}}
		\reglaverb{$exp1 > 0$}{\nlgfun{verb(exp1)} + \nlgtext{es positivo}}
	\end{enumerate}
\item{$\geq$}
	\begin{enumerate}	 
	\setlength{\itemsep}{7pt}
		\reglaverb{$exp1 \geq exp2$ (caso default)}{\nlgfun{verb(exp1)} + \nlgtext{es mayor o igual a} + \nlgfun{verb(exp2)}}
	\end{enumerate}
\item{$\in$}
	\begin{enumerate}	 
	\setlength{\itemsep}{7pt}
		\reglaverb{$exp1 \in exp2$ (caso default)}{\nlgfun{verb(exp1)} + \nlgtext{pertenece(n) a} + \nlgfun{verb(exp2)}}
	\end{enumerate}
\item{$\notin$}
	\begin{enumerate}	 
	\setlength{\itemsep}{7pt}
		\reglaverb{$exp1 \notin exp2$ (caso default)}{\nlgfun{verb(exp1)} + \nlgtext{no pertenece(n) a} + \nlgfun{verb(exp2)}}
	\end{enumerate}
\item{$\subset$}
	\begin{enumerate}	 
	\setlength{\itemsep}{7pt}
		\reglaverb{$exp1 \subset exp2$ (caso default)}{\nlgfun{verb(exp1)} + \nlgtext{está(n) incluido/a(s) en} + \nlgfun{verb(exp2)}}
		\reglaverb{$\{exp1, exp2, \ldots , expn\} \subset expm$ (caso default)}{\nlgfun{verb(exp1)} + \nlgtext{,} + \nlgfun{verb(exp2)} + \nlgtext{, \ldots , y} + \nlgfun{verb(expn)} + \nlgtext{pertenece(n) a} + \nlgfun{verb(expm)}}
		\end{enumerate}
\item{$\not\subset$}
	\begin{enumerate}	 
	\setlength{\itemsep}{7pt}
		\reglaverb{$exp1 \not\subset exp2$ (caso default)}{\nlgfun{verb(exp1)} + \nlgtext{no está(n) incluido/a(s) en} + \nlgfun{verb(exp2)}}
	\end{enumerate}
\item{$\subseteq$}
	\begin{enumerate}	 
	\setlength{\itemsep}{7pt}
		\reglaverb{$exp1 \subseteq exp2$ (caso default)}{\nlgfun{verb(exp1)} + \nlgtext{está incluido o es igual a} + \nlgfun{verb(exp2)}}
		\reglaverb{$\{exp1, exp2, \ldots , expn\} \subseteq exp2$}{ídem inclusión}
	\end{enumerate}
\item{$\not\subseteq$}
	\begin{enumerate}	 
	\setlength{\itemsep}{7pt}
		\reglaverb{$exp1 \not\subseteq exp2$ (caso default)}{\nlgtext{existe al menos un elemento en} + \nlgfun{verb(exp1)} + \nlgtext{que no se está en} + \nlgfun{verb(exp2)}}
	\end{enumerate}
\item{$\mapsto$}
	\begin{enumerate}	 
	\setlength{\itemsep}{7pt}
		\reglaverb{$exp1 \mapsto exp2$ (caso default)}{\nlgtext{el par ordenado formado por:} + \nlgfun{verb(exp1)} + \nlgtext{y} + \nlgfun{verb(exp2)}}
	\end{enumerate}
\item{$\{a, b, \ldots\}$}
	\begin{enumerate}	 
	\setlength{\itemsep}{7pt}
		\reglaverb{$\{\}$}{\nlgtext{el conjunto vacío}}
		\reglaverb{$\{exp1\}$}{\nlgtext{el conjunto formado por } + \nlgfun{verb(exp1)}}
		\reglaverb{$\{exp1, exp2, \ldots ,expn\}$ (caso default)}{\nlgtext{el conjunto formado por} + \nlgfun{verb(exp1)} + \nlgtext{,} + \nlgfun{verb(exp2)} + \nlgtext{, \ldots , y} + \nlgfun{verb(expn)}}
	\end{enumerate}
\item{$\cup$}
	\begin{enumerate}	 
	\setlength{\itemsep}{7pt}
		\reglaverb{$exp1 \cup exp2$ (caso default)}{\nlgtext{elementos en} + \nlgfun{verb(exp1)} + \nlgtext{y en} + \nlgfun{verb(exp2)}}
	\end{enumerate}
\item{$\cap$}
	\begin{enumerate}	 
	\setlength{\itemsep}{7pt}
		\reglaverb{$exp1 \cap exp2$ (caso default)}{\nlgtext{elementos en} + \nlgfun{verb(exp1)} + \nlgtext{que también se encuentren en} + \nlgfun{verb(exp2)}}
	\end{enumerate}
\item{$f~x$ (aplicación)}
	\begin{enumerate}	 
	\setlength{\itemsep}{7pt}
		\reglaverb{$f~exp1$ (caso default)}{\nlgfun{verb(f)} + \nlgtext{aplicada a} + \nlgfun{verb(exp1)}}
	\end{enumerate}
\item{$dom$}
	\begin{enumerate}	 
	\setlength{\itemsep}{7pt}
		\reglaverb{$dom(exp1)$ (caso default)}{\nlgtext{dominio de} + \nlgfun{verb(exp1)}}
	\end{enumerate}
\item{$ran$}
	\begin{enumerate}	 
	\setlength{\itemsep}{7pt}
		\reglaverb{$ran(exp1)$ (caso default)}{\nlgtext{rango de} + \nlgfun{verb(exp1)}}
	\end{enumerate}
\end{itemize}  

\bigskip
Como era de esperarse, podemos observar que, a fin de lograr descripciones mas naturales, deberemos contemplar algunas combinaciones entre los distintos términos posibles. Por ejemplo, podemos ver que en el caso de la igualdad entre conjuntos, se propone una descripción para el caso en que uno de los conjuntos esté formado por un único elemento y otra verbalización distinta en el caso de que éste conjunto se se encuentre vacío.

Otro punto a tener en cuenta al momento de verbalizar una expresión será que las designaciones documentadas por el usuario pueden no contener un artículo que acompañe al sustantivo. En estos casos nuestro sistema deberá identificar los casos en los que sea necesario y agregar el artículo apropiado de forma que concuerde en género y número con el sustantivo. 

Finalmente, en base a las reglas introducidas previamente, podemos notar que la mayoría de las frases a generar se tratan de oraciones bimembres ordenadas como: \emph{sujeto + verbo + objeto}. Todas expresadas en tiempo presente. Además, en algunos casos será necesario conjugar el verbo de una oración de forma tal que concuerde con el número del sujeto de la misma. Algo parecido pasa con el atributo en los casos que utilizamos un verbo copulativo en el que deberemos hacer concordar el mismo con número y género del sujeto. Estas últimas observaciones deberemos tenerlas en cuenta para el desarrollo de nuestro realizador lingüístico.

\chapter{Document Planning}
\label{cap:document_planning}

En la arquitectura presentada en el capítulo~\ref{cap:nlg_intro} mencionamos que el \emph{document planner} es el responsable de decidir que información comunicar (\emph{determinación de contenido}) y como deberá estar estructurada esta información en el texto final (\emph{estructuración de documento}). El document planner será el encargado de que el documento final contenga toda la información requerida por el usuario y que la misma se encuentre estructurada de una forma razonablemente coherente. El resultado de esta etapa será un \emph{document plan} en el cual se especificará qué contenido debe ser incluido en el texto final y de que forma deberá estar estructurado.


A continuación describiremos brevemente la entrada y salida de nuestro \emph{document planner}, definiremos como modelar los elementos informativos (estos serán elementos de nuestro \emph{document plan}) y finalmente describiremos las tareas de \emph{determinación de contenido} y \emph{estructuración de documento}.

\section{Entrada y salida del document planner}
Como el document planner es el primer módulo de nuestro pipeline, la entrada de éste será la misma que la entrada de nuestro sistema. Reiter y Dale~\cite{reiter_dale} generalizan la entrada de un sistema de NLG  como una cuádrupla compuestas por los siguientes componentes:

\bigskip
\noindent
\emph{Fuente de conocimiento:} Se refiere a las bases de datos e información del dominio de aplicación que nos proporcionará el contenido de la información que los textos generados deberán contener.
En nuestro caso la fuente de conocimiento estará compuesta por la especificación a testear, las clases de prueba generadas para ésta y las designaciones de la misma. TODO: deberíamos explayar un poco mas? (mas que nada sobre el tema designaciones...?)

\bigskip
\noindent
\emph{Objetivo comunicacional:} Especifica el propósito que debe cumplir el sistema. En general esta compuesto por un ``tipo de objetivo'' y un parámetro.
En este trabajo tendremos solo un tipo de objetivo comunicacional: \emph{Describir(x)}, dónde el parámetro \emph{x} será un conjunto de identificadores de las clases de prueba a describir.

\bigskip
\noindent
\emph{Modelo de usuario:} Provee información acerca del usuario (nivel de experiencia, preferencias, etc.). En nuestro caso el sistema se comportará de la misma forma independientemente del usuario, por lo que no tendremos en cuenta información del mismo.

\bigskip
\noindent
\emph{Historial de discurso:} Consta de información sobre interacciones previas entre el usuario y el sistema. Este historial puede servir para algunos sistemas interactivos donde las interacciones previas con el usuario pueden resultar de utilidad. 

\bigskip
Como mencionamos anteriormente, la salida del document planner será un document plan que en nuestra arquitectura estará estructurado como un árbol, donde las hojas representarán el contenido y los nodos internos especificarán información estructural, por ejemplo sobre como debe agruparse la el contenido a comunicar. En la sección~\ref{sec:document_structure} desarrollaremos este tema mas en profundidad.

\section{Representación del dominio}

En los sistemas de NLG el texto generado se utiliza principalmente para transmitir información. Esta información será expresada generalmente en frases y palabras, pero estas  frases y palabras no son en si mismo la información; la información subyace estos constructores lingüísticos y es ``llevada'' por ellos. Nos deberemos concentrar entonces en como representar este conocimiento y como ``mapear'' estas estructuras a una representación semántica. 

En lo que queda de esta sección intentaremos definir los \emph{mensajes} que manipulará nuestro sistema. Llamamos mensajes \emph{mensajes}~\cite{reiter_dale} a los elementos informativos que conceptualizan la información que queremos comunicar; son paquetes de información que debe estar presente en el texto final. Estos estarán compuestos por elementos del dominio de aplicación.


El \emph{corpus de descripciones} (apéndice~\ref{ape:corpus}) resulta una buena fuente para estudiar el modelado del dominio y los tipos de \emph{mensajes} que necesitamos comunicar. Anteriormente, vimos del \emph{corpus} que existe una relación entre la información a comunicar y las expresiones Z pertenecientes a las clases de prueba generadas por Fastest. Observamos que las mismas están compuestas por una conjunción de predicados atómicos y que cada uno de estos predicados se corresponde con una oración en lenguaje natural dentro de la descripción de la clase de prueba. Por ejemplo, cada una de las siguientes líneas de la descripción de LookUp\_SP\_1 (pág.~\pageref{fig:ej_desc_lookup_sp_1}):

\medskip
\begin{enumerate}
 \item{\emph{``El símbolo a buscar pertenece a los símbolos cargados en la tabla de símbolos.''}}
 \item{\emph{``El símbolo a buscar es el único elemento del conjunto formado por los símbolos cargados en la tabla de símbolos.''}}
\end{enumerate}

\medskip
\noindent
están respectivamente caracterizadas por la expresiones Z que intentan describir:

\medskip
\begin{enumerate}
  \item{$s? \in \dom st$}
  \item{$\dom st = \{ s? \}$}
\end{enumerate}

Por otro lado, también vimos que los textos correspondientes para cada uno de los predicados que componen una clase de prueba son independientes. Por lo que podríamos separar esta información en distintos mensajes facilitándole la tarea a las siguientes etapas de procesamiento. Es por esto que decidimos definir un único tipo de \emph{mensaje} para nuestro sistema: \emph{VerbalizacionExpresion} que representa, como su nombre lo indica, una verbalización de una expresión Z. Idealmente, tendremos un mensaje \emph{VerbalizacionExpresion} por cada predicado atómico perteneciente a una clase de prueba.

En la figura~\ref{fig:ej_mensajes} podemos ver como quedarían definidos los mensajes para las dos lineas de la descripción de \emph{LookUp\_SP\_1}.

\begin{figure}[H]
  	\centering
	\includegraphics[scale=0.4]{img/mensajes.png}
	\caption{Mensajes a comunicar para el ejemplo de la figura~\ref{fig:ej_desc_lookup_sp_1}.}
  	\label{fig:ej_mensajes}
\end{figure}

Por último, vale la pena aclarar que en nuestro caso, los datos con los que trabajamos resultan esquemas Z de las clases de prueba, por lo que ya se encuentran modelados de antemano y no fue necesario realizar un nuevo modelo del dominio.
 

\section{Determinación del contenido}

La determinación del contenido es el nombre que se le da a la tarea de decidir y obtener la información que se debe comunicar en un texto. Este proceso generalmente involucra una o más tareas de \emph{selección}, \emph{resumen} y \emph{razonamiento con los datos} de entrada.
El proceso de \emph{selección} recopilará un subconjunto de la información de entrada para luego poder ser comunicada al usuario. El objetivo del mismo será el de proveer la información relevante requerida por el mismo.
La tarea de \emph{resumen} es necesaria cuando los datos de entrada son muy ``granulados'' para ser comunicados directamente o si la información relevante consiste alguna generalización o abstracción de los mismos, que no es el caso de nuestro sistema donde las expresiones de Z con las que trabajaremos contienen exactamente la información que se desea comunicar.
Por último el \emph{razonamiento con los datos} resulta un caso general de las dos anteriores, generalmente mas sofisticado y específico del dominio de aplicación. 

Una vez seleccionada y procesada la información necesaria será ésta etapa la encargada de construir los mensajes introducidos en la sección anterior, que luego formarán parte de nuestro \emph{document plan}.

Para nuestro trabajo, la tarea de \emph{selección} se resume en la búsqueda y filtrado de las clases de prueba indicadas por el usuario dentro de todo el conjunto de clases de prueba que forma parte de la entrada del sistema. Por ejemplo, si deseamos generar una descripción para la clase de prueba \emph{LookUp\_SP\_1} de la figura~\ref{fig:ej_tcl_lookup}, la misión de de esta tarea será la de identificar y seleccionar la clase de prueba \emph{LookUp\_SP\_1} entre todas las clases generadas por \emph{Fastest} que formarán parte de los datos de entrada de nuestro sistema de NLG.

Luego de la \emph{selección}, nuestro sistema deberá procesar los datos de entrada con el fin de obtener mejores descripciones. Hemos observado\footnote{Análisis en base a clases de prueba generados utilizando \emph{Fastest 1.6}} que trabajando ciertas expresiones en las clases de prueba podemos mejorar considerablemente los textos producidos. Veamos por ejemplo la figura~\ref{fig:ej_update_sp_4} donde se muestra una clase de prueba generada con \emph{Fastest} en la que se pueden ver dos problemas que abordaremos en esta etapa.

\begin{figure}[H]
  \centering
  \begin{schema}{Update\_ SP\_ 4}\\
   st : SYM \pfun VAL \\
   s? : SYM \\
   v? : VAL 
  \where
   st \neq \{ \} \\
   \{ s? \mapsto v? \} \neq \{ \} \\
   \dom st = \dom \{ s? \mapsto v? \}
  \end{schema}
  \caption{Clase de prueba para operación Update (pág.~\pageref{fig:spec_symbol_table}).}
  \label{fig:ej_update_sp_4}
\end{figure}

Podemos observar que la siguiente expresión del ejemplo anterior:

\begin{figure}[H]
  \centering
  $\{ s? \mapsto v? \} \neq \{ \}$ 
\end{figure}

\noindent
no aporta información relevante para el usuario, de hecho esta expresión no agrega ninguna restricción para el caso de prueba ya que será siempre verdadera. Si intentamos describir este predicado, terminaríamos con un texto generando parecido al siguiente:

\begin{figure}[H]
  \emph{``el conjunto formado por el par de el símbolo a actualizar y el nuevo valor, es distinto al conjunto vacío''}
\end{figure}

\noindent
que además de poder resultar algo difícil de interpretar, no aportaría nada al objetivo comunicacional.

Por otro lado, en un primer intento por describir automáticamente la expresión:

\begin{figure}[H]
  \centering
  $\dom st = \dom \{ s? \mapsto v? \}$ 
\end{figure}

\noindent
podríamos describirla como\footnote{Esta descripción sería la generada utilizando el sistema de reglas propuesto en el capítulo~\ref{TODO} si no trabajamos la expresión en una etapa previa.}:

\begin{figure}[H]
  \emph{``el conjunto de símbolos cargados en la tabla es igual a el dominio del par formado por el símbolo a actualiza y el nuevo valor''}
\end{figure}

Veamos que es posible simplificar notablemente esta descripción si antes trabajamos la expresión anterior, que resulta equivalente a:

\begin{figure}[H]
  \centering
  $\dom st = \{s?\}$ 
\end{figure}

\noindent
que luego podríamos describir como:

\begin{figure}[H]
  \emph{``el símbolo a actualizar es el único elemento en la tabla de símbolos cargados''}
\end{figure}

En conclusión, el procesamiento que nos proponemos a realizar en esta etapa tendrá dos objetivos:
\begin{enumerate}
  \item Eliminar tautologías de las expresiones que forman parte de las clases de prueba seleccionadas.
  \item Realizar algunas simplificaciones o reducciones triviales.
\end{enumerate}

\bigskip
Una vez seleccionada y procesada la información deberemos construir los \emph{mensajes} (\emph{VerbalizacionExpresion}) que luego formarán parte del \emph{document plan} desarrollado en el siguiente capítulo.

\section{Estructuración del documento}

Como dijimos antes, el texto generado no podrá ser una colección al azar de frases y palabras. Deberá tener coherencia y poseer una estructura que le permita al lector interpretar con facilidad el contenido del mismo.
Necesitaremos considerar como organizar y estructurar la información que debemos comunicar con el fin de producir un texto razonablemente fácil de leer y comprender.

%~\footnote{Las decisiones sobre como debe estar ordenado y agrupado el documento final son resultado del análisis del \emph{corpus} de descripciones.}
En esta tarea nos concentraremos en construir una estructura que contenga los \emph{mensajes} seleccionados en la etapa de \emph{determinación de contenido}; estableciendo el agrupamiento y ordenamiento de los mismos. Esta estructura deberá caracterizar la disposición de los elementos pertenecientes a los textos recopilados en el \emph{corpus}. 

Tomando el \emph{corpus} de descripciones como una especificación de los documentos que debemos generar podemos observar que estos documentos poseen una estructura bastante simple y rígida a la vez. Estos documentos deben estar formados por una secuencia de descripciones para las clases de prueba indicadas por el usuario, ordenadas alfabéticamente según el nombre de la clase de prueba. A su vez, cada una de estas descripciones deberá agrupar las verbalizaciones de las expresiones seleccionadas en la etapa de \emph{determinación de contenido}, ordenadas de la misma forma en la que aparecen en el esquema de la clase de prueba en cuestión. En la figura~\ref{fig:png_document_plan} podemos observar una representación abstracta de la estructura propuesta para modelar el documento.

\begin{figure}[H]
  	\centering
	\includegraphics[scale=0.4]{img/document_plan.png}
	\caption{Document plan.}
  	\label{fig:png_document_plan}
\end{figure}

Llamaremos \emph{DocumentoDP} a la raíz de nuestro \emph{document plan}, \emph{DocumentoDP} contendrá a su vez una lista ordenada de las descripciones de las clases de prueba (\emph{DescripcionClasePrueba}) que debemos incluir en el texto final. El elemento \emph{DescripcionClasePrueba} representa el texto a generar para una clase de prueba (por ejemplo el texto de la figura~\ref{fig:ej_lookup_sp_1}) y tendremos uno de estos elementos por cada clase de prueba indicada por el usuario. Finalmente los mensajes seleccionados en la etapa anterior formarán se encontrarán agrupados en la \emph{DescripcionClasePrueba} correspondiente. Podemos ver en la figura~\ref{fig:png_document_plan_ej} un ejemplo del document plan para la descripción de la clase de prueba \emph{LookUp\_SP\_1} introducida anteriormente. 

\begin{figure}[H]
  	\centering
	\includegraphics[scale=0.4]{img/document_plan_ej.png}
	\caption{Document plan correspondiente al texto de la figura~\ref{fig:ej_lookup_sp_1}.}
  	\label{fig:png_document_plan_ej}
\end{figure}


\label{sec:document_structure}
\chapter{Microplanning}
\label{cap:microplanning}

La etapa de \textit{microplanning} será la encargada de, a partir del \textit{document plan} producido por la etapa anterior, producir una especificación mas detallada para el texto a generar. 

En éste capítulo presentaremos las tres tareas que, según Reiter y Dale~\cite{reiter_dale}, deberían llevarse a cabo en esta etapa: lexicalización, agregación y generación de expresiones de referencia. Luego definiremos en detalle la entrada y salida del \textit{microplanner}. Finalmente profundizaremos particularmente sobre la tarea de lexicalización que debemos llevar a cabo para este trabajo.

A lo lago de este capítulo continuaremos con el ejemplo utilizado en la etapa anterior, ilustrando como a partir del \textit{document plan} presentado en la la figura \ref{fig:png_document_plan_ej} construiremos una especificación mas detallada del documento a generar.

\section{Tareas del \textit{Microplanner}}

Como mencionamos previamente, el \textit{microplanner} será el encargado de transformar el \textit{document plan} generado en la etapa anterior en una especificación mas refinada del texto a generar. Cabe aclarar que el resultado de esta etapa no será todavía el texto final ya que quedarán por tomar decisiones acerca de la sintaxis, morfología y cuestiones de presentación, de las cuales se encargará el realizador de superficie.

Como mencionamos en el capítulo~\ref{cap:nlg_intro}, dentro de las tareas generalmente realizadas por el \emph{microplanner} podemos destacar las siguientes:

\medskip
\noindent
\textbf{Lexicalización.} Esta tarea se encarga de elegir que palabras particulares y que constructores sintácticos usar para comunicar la información contenida en el \textit{document plan}. Desarrollaremos más en detalle el trabajo realizado por esta etapa en la sección~\ref{sec:microplanning_lexicalization}


\medskip
\noindent
\textbf{Agregación.} La función de esta tarea es la de combinar los elementos informativos del \emph{document plan} con el fin de conseguir un texto más fluido y legible. La agregación decide que elementos se pueden agrupar para generar oraciones mas complejas sin modificar el significado de las mismas. Por ejemplo, consideremos las siguientes dos descripciones posibles para describir una clase de prueba perteneciente a la especificación de un \emph{scheduler}:

\begin{center}
\begin{enumerate}
  \item \emph{``El proceso a borrar se encuentra en la tabla de procesos. El estado del proceso a borrar es waiting.''} 
  \item \emph{``El proceso a borrar se encuentra en la tabla de procesos y el estado del mismo es waiting.''}
\end{enumerate}
\end{center}

\medskip
\noindent
Para este trabajo, decidimos expresar nuestras descripciones siguiendo el estilo de la primer frase del ejemplo anterior, es por esto que nuestro \textit{microplanner} no realizará tareas de agregación. En nuestro caso en particular creemos que será útil para el lector que cada oración de nuestra descripción haga referencia a una única restricción del esquema de la clase de prueba. De esta forma podríamos identificar con mayor facilidad cual es la descripción para cada expresión particular de una clase de prueba.


\medskip
\noindent
\textbf{Generación de expresiones de referencia.} Esta tarea se encarga de determinar que frases deben ser usadas para identificar las diferentes menciones al mismo elemento en un texto a fin de aportar fluidez al mismo. Por ejemplo, en los casos que se hace referencia a una entidad que ya ha aparecido en el texto se puede remplazar la misma por otra frase que la referencie. La elección de qué expresión utilizar para referirse a la entidad dependerá del contexto y deberá hacerse sin generar ambigüedad para el lector. Por ejemplo, siguiendo con el ejemplo del \emph{scheduler}, introducido anteriormente, podríamos reemplazar la segunda ocurrencia de ``el proceso a borrar'' en la primer frase por el pronombre ``mismo'', quedando entonces:

\smallskip
\begin{center}
\emph{``El proceso a borrar se encuentra en la tabla de procesos. El estado del mismo es waiting.''} 
\end{center}

\smallskip
La generación de expresiones de referencia se encuentra fuera del alcance de este trabajo, por lo que nuestro \textit{microplanner} no realizará tareas este tipo de tareas, pero de la misma forma que la tarea de agregación, podría considerarse para un trabajo futuro. Además, como podemos observar en el \emph{corpus} nuestras descripciones de clases de prueba están formadas por una serie de oraciones individuales, donde cada una de estas describe una restricción de la clase de prueba dada; estas oraciones provienen de la verbalización de predicados atómicos quedando como resultado oraciones relativamente concisas y es extraño que hagan referencia en más de una oportunidad a un mismo elemento en la misma oración, por lo tanto creemos que no resulta necesario que nuestro \textit{microplanner} cuente con un generador de expresiones de referencia en nuestro trabajo.

%TODO en trabajo futuro se puede relacionar la agregacion con generacion de expresiones de referencia. Diciendo que la inclusion de tareas de agregacion probablemente requieran tareas de generacion de expresiones de referencia para la generacion de textos mas fluidos.


\section{Entrada y salida del \textit{microplanner}}
La entrada del \textit{microplanner} será un \textit{document plan} producido por la etapa anterior. Observemos por ejemplo el \textit{document plan} presentado en la figura \ref{fig:png_document_plan_ej} del capítulo anterior, utilizado para modelar la descripción de la clase de prueba \emph{LookUp\_SP\_1}. Esta abstracción no especifica las frases que nuestro sistema debe generar, ni si deben estar enumeradas en una lista de ítems o agrupadas en secciones por ejemplo. Necesitaremos una especificación mas concreta, un modelo mas detallado del documento y de las frases a generar. Será entonces responsabilidad del \textit{microplanner}  construir a partir del \textit{document plan} una especificación mas concreta del texto a generar.

Llamaremos \emph{text specification} (o especificación del texto) a la especificación resultado de esta etapa. Ésta se encargará de modelar los distintos elementos que compondrán el documento final (como párrafos, lista de ítems, etc.). Esta especificación estará compuesta en base a \textit{phrase specification} (o especificaciones de frase) encargadas de modelar las distintas oraciones que serán incluidas en el texto final (veremos que cada una de estas se construirán a partir de los mensajes contenidos en el \textit{document plan}). Será luego tarea de la siguiente etapa convertir los nodos internos de la especificación del texto en anotaciones especificas para el sistema de presentación (realización de estructura) y transformar las \emph{phrase specification} en oraciones o frases sintáctica, morfológica y ortográficamente correctas (realización lingüística). 

%para que luego, en la etapa de realización de superficie podamos generar el texto final en base a los requerimientos analizados en el capítulo \ref{cap:corpus}. 

En lo que queda de esta sección estudiaremos como se encuentra constituida nuestra especificación del texto, describiendo también como están formadas nuestras especificaciones de frase.

\subsection{Especificación del texto}

%TODO faltaría explayar un poco mas y hablar sobre conceptos de phrase y text specification
%Como vimos en el capítulo anterior, la salida del  \textit{document planner} es una estructura donde se encuentran agrupados los elementos informativos que deseamos comunicar. Estos elementos o \emph{mensajes} contenidos en el \emph{document plan} especifican de una manera abstracta la información que debemos comunicar en el texto final, pero no especifican, por ejemplo, que palabras debemos usar para hacerlo. 

%Será el \textit{microplanner} el encargado de tomar este tipo de decisiones. Éste tomará como entrada un \textit{document plan} y deberá producir una especificación mas refinada del texto que deseamos generar, la cual será utilizada luego por el \emph{realizador de superficie} para producir el texto final.

La especificación de texto para nuestro sistema, deberá caracterizar la estructura del documento final que nuestro sistema debe producir. Es por esto que modelaremos los mismos utilizando un árbol, donde los hojas especificarán las frases u oraciones a generar (las \emph{phrase specification}), y los nodos internos establecerán cómo estas frases tendrán que ser agrupadas en elementos del documento (como párrafos, secciones, lista de ítems, etc). 

La estructura de los documentos que debemos generar en este trabajo resulta relativamente simple. Como vimos en el capítulo \ref{cap:corpus}, los documentos de descripciones poseen un título y luego se detallan una por una las descripciones de las distintas clases de prueba, donde para cada una de éstas aparece el nombre de la clase de prueba, junto a una pequeña descripción de la operación a testear y luego una lista de ítems que describirán cada una de las restricciones pertenecientes a la clase de prueba que se describe. Es por esto que para este trabajo utilizaremos sólo dos elementos para modelar la estructura interna del documento \emph{TSDocumento} y \emph{TSListaItems}.

\medskip
\noindent
\textbf{TSDocumento:} modela el documento final, por lo tanto solo tendremos un elemento de este tipo en nuestra \emph{text specification} y éste será la raíz del documento. Éste elemento contendrá información general sobre el documento, como el título y una especificación el para cada descripción de clase de prueba, modeladas mediante: \emph{TSListaItems}.

\medskip
\noindent
\textbf{TSListaItems:} modela el texto que describirá a una clase de prueba. Este elemento contiene una \emph{phrase specification} modelar el texto correspondiente al titulo y al detalle de la operación testeada. Además contendrá una lista de \emph{phrase specification} que modelarán las frases para cada una de las verbalizaciones de las expresiones contenidas en la clase de prueba en cuestión.

\begin{figure}[H]
  	\centering
	\includegraphics[scale=0.3]{img/text_spec.png}
	\caption{\emph{Text Specification}.}
  	\label{fig:text_spec}
\end{figure}

En la figura anterior podemos observar la estructura abstracta que tendrán nuestras \emph{text specification} y por ejemplo, sin meternos en detalle todavía sobre la estructura de las especificaciones de frase, en la figura \ref{fig:text_spec} podemos ver una especificación de frase para el ejemplo introducido anteriormente (pág. \pageref{fig:ej_corpus}).

\begin{figure}[H]
  	\centering
	\includegraphics[scale=0.35]{img/ej_text_spec.png}
	\caption{Ejemplo \emph{Text Specification}.}
  	\label{fig:text_spec}
\end{figure}

Veremos en el próximo capítulo como, en la etapa de realización estructural, transformaremos estos elementos en anotaciones para el sistema de presentación.

\subsection{Especificación de frase}

En la literatura sobre NLG Podemos encontrar muchas alternativas en lo que respecta a la especificación de frases. Todas estas varían en el nivel de abstracción que poseen las mismas. Las representaciones mas abstractas le darán mas flexibilidad a las etapas de \textit{document planning} y \textit{microplanning}, pero al mismo tiempo nos obligarán a tener un realizador de superficie mas sofisticado. Por otro lado, las especificaciones menos abstractas, requieren que el \textit{document planner} y el \textit{microplanner} realicen un mayor trabajo, pero también tendrán mas control sobre el texto a producir. Uno de los objetivos que tuvimos a la hora de idear una estructura para nuestra especificación de frases fue que ésta sea independiente de nuestro problema, pretendemos que hable en términos de la lengua (castellano en nuestro caso) que queremos generar y no en términos específicos de Z. De esta forma podremos implementar un realizador de superficie que sea independiente de este problema y que pueda ser reutilizado. %TODO nota sobra la falta de un realizador en español al momento de desarrollar el trabajo

%\footnote{Grupo de palabras que ejercen una función sintáctica dentro de una oración}
Es por esto que decidimos especificar las oraciones a generar mediante árboles sintácticos, donde los constituyentes de éstos son los sintagmas de la oración que deseamos generar. Esto le dará la posibilidad al realizador lingüístico de poder identificar la función de cada uno de los constituyentes de la oración. Por ejemplo, como detallamos en los requerimientos de la sección \ref{sec:corpus_gramatica}, el \emph{realizador de superficie} necesitará identificar el núcleo de un sintagma nominal (núcleo del sujeto) para poder producir una oración en la que haya concordancia de número y persona entre el verbo y el sujeto. Como consecuencia del ejemplo anterior, nuestro realizador lingüístico deberá tener identificar también el sujeto, predicado y verbo de una oración. Creemos que con los elementos presentes en la figura~\ref{fig:phase_spec} podremos modelar la frases incluidas en el \emph{corpus}.

\begin{figure}[H]
  	\centering
	\includegraphics[scale=0.7]{img/phrase_spec.png}
	\caption{\textit{Phrase Specification}.}
  	\label{fig:phase_spec}
\end{figure}

No pretendemos modelar todo la lengua castellana con estos elementos sino solo un subconjunto que nos provea las herramientas necesarias para permitirle al \emph{realizador de superficie} generar las frases definidas en el capítulo \ref{sec:corpus_analisis}, ya que el desarrollo de un realizador lingüístico que considere todas las construcciones sintácticas de nuestra lengua escapa el alcance de este trabajo. Es por esto que sólo modelamos los sintagmas nominales (FraseNominal) y verbales (FraseVerbal) y nos vemos obligados a incluir otros elementos como \emph{ElementosYuxtapuestos} para salvaguardar la falta de algunos constituyentes sintácticos como sintagmas adjetivales, preposicionales, etc. 

A continuación describiremos brevemente cada uno de estos elementos, profundizando sobre la realización de los mismos en el capítulo \ref{cap:linguistic_realization}.


\medskip
\begin{itemize}
\item{\emph{\textbf{FraseEnlatada}}: Representa texto que no necesita ningún tipo de procesamiento posterior a realizar durante la realización lingüística, será incluido en el texto tal cual fue establecido.}
\item{\emph{\textbf{Oración}}: Modela oraciones bimembres. El realizador lingüístico deberá procesarlas en base a una serie de reglas gramaticales para producir un texto sintáctica, morfológica y ortográficamente correcto para éstas.}
\item{\emph{\textbf{FraseVerbal}}: Representa un sintagma verbal que corresponderá al predicado de una \emph{Oración}.}
\item{\emph{\textbf{FraseNominal}}: Modela un sintagma nominal. Generalmente conformará el sujeto en una \emph{Oración}.}
\item{\emph{\textbf{ElementosCoordinados}}: Representa una serie de elementos que se deberán transformar en una conjunción de frases en la etapa de realización lingüística, por ejemplo: \emph{``frase1\textbf{,} frase2 \textbf{y} frase3''}}
\item{\emph{\textbf{ElementosYuxtapuestos}}: Representa una lista ordenada de elementos que deberán ser realizados y \emph{concatenados} en la oración final. Nos vimos obligados a introducir este tipo de elementos para salvaguardar la falta de algunos constituyentes sintácticos como sintagmas adjetivales, preposicionales, etc.}
\end{itemize}

En la siguiente sección veremos como, la tarea de lexicalización, construirá estas especificaciones de frase en base a la información contenida en los mensajes del \textit{document plan}.
%\section{Arquitectura}

\section{Lexicalización}
\label{sec:microplanning_lexicalization}

Como mencionamos anteriormente, el proceso de lexicalización será el encargado de elegir que palabras particulares y constructores sintácticos usar para comunicar la información contenida en el \textit{document plan}. En esta etapa deberemos producir una especificación de frase para cada mensaje contenido en el \textit{document plan}. En nuestro caso debemos hacerlo contemplando todos los casos definidos en el capítulo \ref{sec:corpus_reglas}, es decir, nuestro proceso de lexicalización tendrá que comportarse de forma similar a la función \emph{verb} que estudiamos durante el análisis de requerimientos. Tanto las palabras, como los sintagmas a utilizar se desprenderán de las frases presentes en la definición de ésta.

Como analizamos en el capítulo \ref{cap:corpus}, nuestro sistema deberá producir una oración en lenguaje natural para cada predicado incluido dentro del cuerpo de cada clase de prueba, a su vez, la verbalización para cada una de estas expresiones se encuentra caracterizada por un mensaje dentro del \textit{document plan}. Es por esto, que el módulo encargado de esta tarea deberá ser capaz de generar una \emph{especificación de frase} a partir de la expresión Z contenida en cada uno de estos mensajes. De acuerdo a los requerimientos introducidos en el capítulo \ref{cap:corpus}, nuestro lexicalizador primero deberá verificar si la expresión en cuestión se encuentra designada, en este caso, tendrá que construir una especificación de frase en base a su designación. De lo contrario deberá intentar construirla recursivamente contemplando los casos para los distintos operadores y las posibles combinaciones. En la figura~\ref{fig:algoritmo_lexicalizacion} podemos ver un bosquejo del comportamiento esperado para esta tarea, de acuerdo al análisis realizado en el capítulo \ref{sec:corpus_reglas}, trabajando esta vez con las \emph{phrase specification} definidas en la sección anterior. Incluimos sólo un bosquejo ya que ilustrar el comportamiento completo de esta tarea resulta extenso debido a la construcción y composición de elementos que debemos realizar para cada caso. 

%TODO ver como cambiar el caption de esto
\begin{figure}
\begin{algorithm}[H]
\begin{algorithmic}
\Function {lexicalizacion}{$exp$}
\If{$esta\_designada(exp)$}
\State $ret\gets \text{designacion}(exp)$
\Else
\State $ret\gets \text{lexicalizacion'}(exp)$
\EndIf
\State \textbf{return} $ret$
\EndFunction
\Statex
\Function {lexicalizacion'}{$x \protect\in y$}
\State $oracion.sujeto\gets \Call{lexicalizacion}{x}$
\State $fraseVerbal.verbo\gets \text{\textit{``pertenece''}}$
\State $fraseEnlatada.texto\gets \text{\textit{``a''}}$
\State $elemYuxtapuesto.elementos\gets \{fraseEnlatada, \Call{lexicalizacion}{y}\}$
\State $fraseVerbal.complemento\gets elemYuxtapuesto$
\State $oracion.predicado\gets fraseVerbal$
\State \textbf{return} $oracion$
\EndFunction
\end{algorithmic}
\end{algorithm}
\label{fig:algoritmo_lexicalizacion}
\caption{Bosquejo Lexicalización.}
\end{figure}

La función \emph{designacion} deberá ser capaz de construir una especificación de frase a partir de una expresión designada. Analizaremos este caso con mayor profundidad en la siguiente sección. Por otro lado, notemos que en el caso que la expresión a lexicalizar no se encuentre designada, se deberá analizar recursivamente la expresión para generar el texto adecuado. 

A continuación, retomaremos el ejemplo de la figura \ref{fig:text_spec} y veremos como deberá nuestro sistema generar la especificación de frase para uno de los mensajes incluido en el \textit{document plan}. En la figura \ref{fig:phase_spec_ej} podemos observar este mensaje y el resultado de la lexicalización del mismo. Veremos a continuación los pasos que deberá realizar nuestro sistema durante la tarea de lexicalización para lograr el resultado ilustrado en la imagen.

%A continuación detallaremos los pasos realizados por nuestra tarea de lexicalización para obtener dicho resultado.

\begin{figure}
  	\centering
	\includegraphics[scale=0.25]{img/phrase_spec_ej.png}
	\caption{Phrase Specification para $s? \protect\in \protect\dom st$.}
  	\label{fig:phase_spec_ej}
\end{figure}

El objetivo del lexicalizador será construir una especificación de frase escogiendo adecuadamente las palabras y constructores sintácticos para siguiente la expresión:

\begin{center}
$s? \in \dom st$
\end{center}

\noindent
Recordemos que contaremos con las siguientes designaciones:

%TODO ver esto
\begin{figure}[H]
\begin{align*} 
  &s? && \approx \text{el símbolo a buscar} \\
  &dom~x && \approx \text{símbolos cargados en la tabla de símbolos}
\end{align*}
\end{figure}

En primer instancia, al no encontrarse designada $s? \in \dom st$ nuestro lexicalizador intentará construir la especificación en base a los operadores que componen la expresión a describir. En el bosquejo anterior (figura \ref{fig:algoritmo_lexicalizacion}) incluimos el caso para lexicalizar el predicado $x \in y$ que deberemos utilizar para lexicalizar $s? \in \dom st$. 

Será necesario lexicalizar recursivamente las expresiones: $s?$ y $\dom st$. El resultado de estas lexicalizaciones será usado para construir el sujeto y parte del predicado de la oración a especificar. Estas expresiones se encuentran ambas designadas y por lo tanto su especificación de frase se construirá a partir del texto incluido en sus designaciones. Veremos luego en la siguiente sección siguiente la tarea de la función \emph{designacion} utilizada en el algoritmo anterior encargada de producir una especificación de frase a partir de una expresión designada. 

Como ya mencionamos, utilizaremos el resultado de la lexicalización de $s?$ para formar el sujeto de la oración que deseamos especificar y deberemos luego construir la \emph{FraseVerbal} que cumplirá el rol de predicado. Para construir esta última usaremos la especificación de frase que modelará la verbalización $\dom st$, también mencionada anteriormente. 

Finalmente escogeremos las palabras indicadas a fin de que nuestro sistema pueda dar con una descripción similar a la presentada en el capítulo \ref{sec:corpus_reglas} del análisis de requerimientos. Notemos que para esto utilizamos el verbo ``pertenece'', en infinitivo, siendo luego tarea del realizador lingüístico la de conjugar el mismo de acuerdo a las reglas gramaticales introducidas en el capítulo~\ref{sec:corpus_gramatica}. Otra cuestión a mencionar es el uso del elemento \emph{ElementoYuxtapuestos} para salvaguardar la falta de un elemento que nos sirva para modelar un sintagma preposicional en este caso. Nuestro realizador lingüístico deberá procesar los elementos contenidos en cada \emph{ElementoYuxtapuestos} generando un texto resultado de la concatenación de la realización los mismos.

A continuación, en la siguiente sección estudiaremos finalmente, la lexicalización de expresiones designadas.

%TODO cambiar nombre
\section{Lexicalización de expresiones designadas}
\label{sec:verbalizacion_designaciones}
Como vimos en la sección anterior, nuestra tarea de lexicalización deberá hacer uso de las designaciones presentes en la especificación para la construir una especificación de frase. Para esto, cuando una expresión se encuentre designada, nuestro sistema tendrá que procesar la misma, construyendo una especificación de frase que la caracterice. Esto será necesario ya que, como mencionamos previamente, en la etapa de \emph{realización de superficie} nuestro sistema necesitará conocer los distintos constituyentes sintácticos de las oraciones que les provee nuestra especificación de frase y en algunos casos también deberá modificar levemente los textos presentes en las designaciones (por ejemplo, como mencionamos en el capítulo \ref{sec:corpus_gramatica}, puede ser necesario agregarle el artículo correspondiente a la frase utilizada en la designación).

Para este trabajo, estudiaremos por separado las designaciones parametrizadas y las no parametrizadas.

Comencemos por analizar la lexicalización de una expresión que se encuentra designada por medio de una designación no parametrizada. Por ejemplo, supongamos que queremos construir una especificación de frase para la expresión $s?$ del ejemplo utilizado en la sección anterior. Recordemos que la designación para la misma es:

\begin{center} 
  $s? \approx \text{el símbolo a buscar}$ 
\end{center}

La oración utilizada en la designación anterior, como en los casos observados en el \textit{corpus} (para designaciones no parametrizadas) resulta un \emph{sintagma nominal}. En este caso ``símbolo'' es el núcleo, ``el'' cumple la función de determinante y `` a buscar'' es el complemento. Será posible entonces, para nuestra tarea de lexicalización, modelar estas frases utilizando una \emph{FraseNominal}. 

Para que nuestro sistema sea capaz de esto deberá analizar síntacticamente las designaciones, \textit{parseando} las mismas con la ayuda de un analizador morfológico que nos permitirá obtener la función sintáctica de cada constituyente de la frase. Además de esto, para simplificar la la tarea de \emph{parseo} de nuestro sistema, requeriremos que el usuario escriba las designaciones mediante un sintagma nominal. Es decir, respetando la siguiente estructura:

\begin{figure}[H]
  \centering
   \textbf{Sintagma Nominal} = [\textbf{Determinante}] + \textbf{Núcleo} + [\textbf{Complemento}]
\end{figure}

En el capítulo \ref{TODO} veremos mas en detalle cómo utilizamos un analizador morfológico para ayudarnos con la especificación de las frases incluidas en las designaciones.

Por otro lado, las frases incluidas en las designaciones parametrizadas no poseen la misma estructura. La tarea de modelar minuciosamente estos textos resulta más compleja que para el caso anterior, por ejemplo, podríamos tener uno o más parámetros presentes dentro del texto, para los cuales deberíamos identificar el rol que cumple cada uno de los anteriores dentro de la oración. Por otro lado nuestro realizador lingüístico sólo soportará oraciones de la forma SVO (sujeto, verbo, objeto) lo cual podría no respetarse en una designación introducida por el usuario. Es por esto que nuestro sistema proveerá solo soporte parcial para las designaciones parametrizadas, aceptando sólo designaciones con un único parámetro y para describir una expresión parametrizada requeriremos también que el argumento de la misma también se encuentre designado. De esta forma podríamos describir una designación parametrizada de la misma forma que vimos en el capítulo \ref{cap:corpus} reemplazando el parámetro presente en el texto de la designación por el texto incluido en la designación del argumento.

Veamos por ejemplo las siguientes designaciones para una especificación que modela un pequeño sistema de monitoreo de sensores:

\begin{figure}[H]
\begin{align*} 
  &x \in \dom smax && \approx \text{x es un identificador válido} \\
  &s? && \approx \text{el identificador del sensor leído}
\end{align*}
\end{figure}

Donde para describir la expresión $s? \in \dom smax$ bastará con reemplazar el parámetro dentro del texto de la designación parametrizada con el texto incluido en la designación de $s?$ como vemos en la figura \ref{fig:ej_lexicalizacion_desig}.

\begin{figure}[H]
  	\centering
	\includegraphics[scale=0.5]{img/ej_lexicalizacion_desig.png}
	\caption{Lexicalización $s? \protect\in \protect\dom smax$.}
  	\label{fig:ej_lexicalizacion_desig}
\end{figure}

Como podemos observar en el ejemplo anterior, al desconocer la estructura de las oraciones introducidas por el usuario para las designaciones parametrizadas, deberemos modelar el texto producido por la composición de ambas designaciones utilizando una \emph{FraseEnlatada}. Este texto texto contenido en estas será luego realizado sin ningún procesamiento previo, por lo que será responsabilidad del usuario que el mismo sea sintáctica y ortográficamente correcto.

\bigskip
En este capítulo vimos las tareas necesarias para, partiendo de la salida producida por el \textit{document planner}, constituir una especificación mas refinada del texto final. En el próximo capítulo veremos finalmente las tareas que deberán llevarse a cabo para transformar esta especificación del texto en el documento final que contendrá todas las descripciones requeridas por el usuario.


\chapter{Realización de superficie}
\label{cap:realization}
En los capítulos anteriores vimos los procesos que necesitan ser llevados a cabo para lograr una especificación del texto a generar. Finalmente, en esta última etapa de nuestro sistema nos concentraremos en estudiar las tareas que deben ser realizadas a fin de transformar esta especificación del texto en texto de superficie, formado por frases, símbolos de puntuación y algunas etiquetas de mark-up necesarias.

Como mencionamos en el capítulo~\ref{cap:microplanning}, nuestra especificación del texto es un árbol en el que las horas representan las frases individuales y los nodos internos establecen como éstas están agrupadas (según estructuras como párrafos, secciones, listas de ítems, etc). En base a esta diferenciación entre los elementos de nuestra especificación del texto podemos distinguir dos grandes aspectos dentro de esta tarea: la \emph{realización estructural} responsable de transformar los nodos internos de nuestra especificación de texto en anotaciones particulares del sistema de presentación y por otro lado, la tarea de \emph{realización lingüística} que se concentrará en generar frases sintáctica, morfológica, ortográficamente correctas.
%Además, como señalamos en el capítulo~\ref{sec:corpus_analisis}, deberemos agregar un artículo como determinante en el caso de ser necesario. 

\section{Realización estructural}
\label{cap:structure_realization}

%La tarea de realización estructural puede resultar relativamente sencilla, pero es importante que se realice por separado a fin de poder generar textos para distintos sitemas de presentacion utilizando la misma realizacion linguistica. Es decir, separar las cuestiones del sistema de presentacion utilizado de las frases u oraciones generadas por nuestro sistema.

Como mencionamos anteriormente, en la etapa de realización estructural se deberán transformar los constructores lógicos existentes en la especificación del texto en constructores del sistema de presentación. Hoy en día no debemos ocuparnos de cuestiones de formateo de bajo nivel, sino que la mayoría de los sistemas de procesamiento de texto nos permiten indicar mediante el uso de símbolos o etiquetas la naturaleza de una estructura determinada; estas luego estas serán procesadas y renderizadas de manera apropiada permitiéndole al lector una correcta visualización.

Es evidente que esta etapa será dependiente del sistema de presentación escogido para el documento final y a su vez independiente del proceso de realización lingüística de la las oraciones o frases a generar.  Por ejemplo, podemos observar en la figura~\ref{fig:ej_latex} el código \LaTeX para una lista de items de una descripción y en la figura~\ref{fig:ej_html} la realización para la misma especificación de texto pero en lenguaje HTML, donde el texto contenido en ambos ejemplos es exactamente el mismo, sólo diferencian en las diferentes anotaciones utilizadas para cada sistema de presentación.

\begin{figure}[H]
  \begin{verbatim}
  LookUp\_SP\_1: Se busca un símbolo en la tabla.  
  \begin{itemize}
    \item{Cuando:}
    \begin{itemize}
      \item{El símbolo a buscar pertenece ...}
      \item{El símbolo a buscar es el único ...}   
    \end{itemize}
  \end{itemize}
  \end{verbatim}
  \caption{Texto final en \LaTeX.}
  \label{fig:ej_latex}
\end{figure}

\begin{figure}[H]
  \begin{verbatim}
  LookUp_SP_1: Se busca un símbolo en la tabla.  
  <ul>
    <li>Cuando:</li>
    <ul>
      <li>El símbolo a buscar pertenece ...</li>
      <li>El símbolo a buscar es el único ...</li>
    </ul>
  </ul>
  \end{verbatim}
  \caption{Texto final en HTML.}
  \label{fig:ej_html}
\end{figure}

Podemos pensar el problema de realización estructural como el proceso de mapear los constructores lógicos de nuestra especificación del texto en constructores lógicos del lenguaje de presentación a utilizar. En nuestro caso sólo tenemos que considerar dos elementos: \emph{TSDocumento} y \emph{TSListaItems}.

Para realizar \emph{TSDocumento} se deberán agregar algunas etiquetas de encabezado necesarias según el tipo del documento, deberemos especificar el título y generalmente deberemos delimitar el principio y final del texto a incluir en el documento. Luego deberemos realizar recursivamente los elementos contenidos en el documento (\emph{TSListaItems}) y ubicarlos adecuadamente en el cuerpo del mismo. En la figura~\ref{fig:ej_latex2} podemos ver un ejemplo del código que deberíamos generar para la realización estructural del documento, la realización de \emph{TSListaItems} se puede apreciar en el ejemplo anterior de la figura~\ref{fig:ej_latex}.

\begin{figure}[H]
  \begin{verbatim}
  \documentclass{article}
  \title{titulo}
  
  \begin{document}
  \maketitle
      ...
  \end{document}
  \end{verbatim}
  \caption{Código \LaTeX para estructura del documento.}
  \label{fig:ej_latex2}
\end{figure}


\section{Realización lingüística}
\label{cap:linguistic_realization}

% introducir reglas del leguaje (concordancia)
% explicar lexicalizacion para cada elemento de nuestra especificacion de frase

La realización lingüística desempeñará su tarea al nivel de la oración. Como mencionamos anteriormente, la misión de esta tarea será transformar las especificaciones de frases de nuestro sistema en oraciones bien formadas. Entendiendo por oración bien formada aquellas que cumplan con las reglas gramaticales del lenguaje (español en nuestro caso); estas reglas se ocuparán tanto de la morfología como de la sintaxis de las mismas. Entonces, la realización lingüística consistirá en aplicar alguna caracterización de estas reglas a cada especificación de frase a fin de producir un texto que sea sintáctica y morfológicamente correcto.

Es importante también, que el texto producido sea ortográficamente correcto. Para esto, asumiremos que las designaciones provistas por el usuario son ortogáficamente correctas, no tendremos que comprometer a que las palabras generadas por nuestro sistema también lo sean y finalmente nos deberemos encargar de que la primer palabra de una oración comience con mayúscula y de colocar un signo de puntuación al final de la misma.

En base a las reglas introducidas en el capítulo~\ref{sec:corpus_analisis} podemos extraer algunos de los aspectos sintácticos y morfológicos que debemos tener en cuenta durante esta etapa para ser capaces de producir los textos esperados. En particular nos focalizaremos en tres reglas de \emph{concordancia gramatical} que nuestro realizador lingüístico deberá contemplar, extraídas del ``Diccionario panhispánico de dudas'' de la Real Academia Española~\cite{???}.

\medskip
\noindent
\textbf{\emph{Concordancia entre artículo y sustantivo.}} Establece que el artículo debe concordar en género y número con el sustantivo al que acompaña.
\noindent
Como mencionamos anteriormente, el usuario podría no incluir el artículo en una designación. Por ejemplo, continuando con el ejemplo de la figura~\ref{fig:ej_designacion}, se podría haber omitido el artículo de ``el símbolo a buscar'' y solo haber designado:
\begin{figure}[H]
	\center
    $s? \approx \text{símbolo a buscar}$
\end{figure}
\noindent
Será entonces nuestro sistema el encargado de agregar el artículo de ser necesario\footnote{Podría no ser necesario, por ejemplo si el núcleo de la frase designada fuese un nombre propio.}, asegurándose de que el mismo concuerde con el sustantivo. Para esto deberemos ser capaces de identificar el género y número del sustantivo mediante el uso de un analizador morfológico.

\medskip
\noindent
\textbf{\emph{Concordancia entre sujeto y atributo de verbo copulativo.}} Establece que el atributo de un verbo copulativo (ser, estar, parecer) debe concordar en género y número con el sujeto.
\noindent
Veamos por ejemplo la regla para describir el operador $\subset$ definida en el capítulo~\ref{sec:corpus_analisis}

\begin{figure}[H]
	$exp1 \subset exp2$ \\
	\hspace*{0.56cm} $\rightarrow$ \nlgfun{verb(exp1)} + \nlgtext{está(n) incluido/a(s) en} + \nlgfun{verb(exp2)}
\end{figure}
\noindent
Como podemos ver, tenemos el verbo copulativo \emph{``está''} acompañado del atributo \emph{``incluido''}. En este caso deberemos escojer el genero y numero del atributo de forma tal que concuerde con el sujeto.

\medskip
\noindent
\textbf{\emph{Concordancia entre sujeto y verbo.}} El verbo debe concordar con el sujeto en número y persona. En el caso de haber varios sujetos, la concordancia debe hacerse con el verbo en plural.
\noindent
Observemos por ejemplo la regla por defecto para el operador $\subset$:

\begin{figure}[H]
	$\{exp1, exp2, \ldots , expn\} \subset expm$ \\
	\hspace*{0.56cm} $\rightarrow$\nlgfun{verb(exp1)} + \nlgtext{,} + \nlgfun{verb(exp2)} + \nlgtext{, \ldots , y} + \nlgfun{verb(expn)} + \nlgtext{pertenece(n) a} + \nlgfun{verb(expm)}
\end{figure}

\noindent
Si la expresión a la izquierda del $\subset$ es un conjunto unitario deberemos usar el verbo \emph{``pertenece''} (en singular), pero de tratarse de un conjunto con más de un elemento, deberemos usar el verbo en plural.

\medskip
Otra cuestión a observar, además de las reglas de concordancia ya vistas, es que nuestro sistema deberá generar casi exclusivamente oraciones bimembres ordenadas como \emph{sujeto + verbo + objeto}. Siempre expresadas en tiempo presente. Por lo tanto nuestro realizador siempre deberá respetar el orden de palabras y tiempo verbal mencionado para realizar los elementos de tipo \emph{Oracion} de nuestra especificación de frase. 

Debemos aclarar que en las reglas mencionadas anteriormente podemos observar oraciones en las cuales el orden de palabras difiere del mencionado anteriormente, por ejemplo: la frase ``no hay elementos en ...'' cuyo orden es \emph{verbo + sujeto + objeto}. Para poder generar estas frases no modelaremos estos casos como una \emph{Oracion} sino como \emph{ElementosYuxtapuestos} entre una \emph{FraseEnlatada} con el texto ``no hay elementos en'' y el objeto al que se haga referencia. Podemos hacerlo de esta forma ya que no tenemos necesidad procesar la especificación de la frase ``no hay elementos en'' ya que sabemos de antemano que es sintáctica y morfológicamente correcta. Esto no podríamos asegurarlo si el sujeto de la oración pudiese ser una designación por ejemplo, pero en este caso, ya lo conocemos.

%hablar un poco sobre tiempo verbal
%Introducir acá con ejemplos las reglas sobre concordancia necesarias
%TODO agregar ejemplos para cada frase
%TODO hablar algo de negacion?

\bigskip
En base a los aspectos mencionados anteriormente analizaremos la tarea de realización lingüística para cada elemento de nuestra especificación de frase ilustrando cada caso con un pseudocódigo para el algoritmo de verbalización. Llamaremos \emph{verbalizar} a la función encargada de generar una oración sintáctica y mofológicamente correcta en lenguaje natural en base a una especificación de frase. Ésta resultará la función principal de nuestro realizador lingüístico dando como resultado una oración a la que sólo deberemos realizarle algunas pequeñas modificaciones para satisfacer los aspectos ortográficos antes mencionados y dar por finalizada la tarea de realización lingüística.

\medskip
\noindent
\textbf{\emph{FraseEnlatada.}} La realización de una frase enlatada será trivial, simplemente habrá que extraer el texto contenido en la misma sin necesidad de realizar ningún tipo de procesamiento.

\begin{algorithm}[H]
\caption{Realización lingüística frase enlatada.}
\begin{algorithmic}[1]
\Function {verbalizar}{FraseEnlatada $frase$}
\State \textbf{return} $frase.texto$
\EndFunction
\end{algorithmic}
\end{algorithm}


\medskip
\noindent
\textbf{\emph{ElementosYuxtapuestos.}} Representa una concatenación de frases. El resultado de la verbalización deberá ser un texto que resulte de la unión de las verbalizaciones individuales de cada uno de los elementos contenidos, agregando los espacios correspondientes entre estos\footnote{Suponemos que la función \emph{concat} además de concatenar los elementos agrega un espacio entre cada uno de éstos.} y respetando el orden en que se encuentren. 

\begin{algorithm}[H]
\caption{Realización lingüística elementos yuxtapuestos.} 
\begin{algorithmic}[1]
\Function {verbalizar}{ElementosYuxtapuestos $elem$}
\For{\textbf{each} $e$ \textbf{in} $elem.elementos$}
\State $resultado\gets \text{concat}(resultado, \text{verbalizar}(e))$
\EndFor
\State \textbf{return} $resultado$
\EndFunction
\end{algorithmic}
\end{algorithm}

\medskip
\noindent
\textbf{\emph{ElementosCoordinados.}} Se trata de una serie de elementos que deberán ser verbalizados individualmente y unidos, de una forma similar a la anterior, a fin de obtener una conjunción de frases.

\begin{algorithm}[H]
\caption{Realización lingüística elementos yuxtapuestos.}
\begin{algorithmic}[1]
\Function {verbalizar}{ElementosCoordinados $elem$}
\For{\textbf{each} $e$ \textbf{in} $elem.elementos$}
\If{esPrimerElemento($e$, $elem$)}
\State $resultado\gets \text{verbalizar}(e)$
\ElsIf{esUltimoElemento($e$, $elem$)}
\State $resultado\gets \text{concat}(resultado, \text{\emph{``y''}}, \text{verbalizar}(e))$
\Else
\State $resultado\gets \text{concat}(resultado, \text{\emph{``,''}}, \text{verbalizar}(e))$
\EndIf
\EndFor
\State \textbf{return} $resultado$
\EndFunction
\end{algorithmic}
\end{algorithm}

\medskip
\noindent
\textbf{\emph{FraseNominal.}} Para verbalizar una frase nominal, deberemos unir en orden: \emph{determinante} (de ser requerido), \emph{nucleo} y la verbalización del \emph{complemento}, agregando los espacios correspondientes entre medio.

\begin{algorithm}[H]
\caption{Realización lingüística FraseNominal.}
\begin{algorithmic}[1]
\Function {verbalizar}{FraseNominal $frase$}
\State $nucleo\gets frase.nucleo$
\State $complemento\gets \text{verbalizar}(frase.complemento)$
\If{esNombre($nucleo$)}
\State $determinante\gets \text{determinar\_articulo}(frase)$
\State $resultado\gets \text{concat}(determinante, nucleo, complemento)$
\Else
\State $resultado\gets \text{concat}(nucleo, complemento)$
\EndIf
\Statex
\State \textbf{return} $resultado$
\EndFunction
\end{algorithmic}
\end{algorithm}

\noindent
La función \emph{determinar\_articulo} deberá recuperar el determinate apropiado para la frase, en el caso de encontrarse ya establecido de antemano devolverá el mismo, en caso contrario (de ser necesario) deberá determinar el artículo indicado según número y género del núcleo de la frase. Observemos que se deberá verbalizar recursivamente el complemento de la frase (éste probablemente sea una frase enlatada o una yuxtaposición de éstas). Finalmente se construye un texto con el \emph{determinante}, \emph{núcleo} y la verbalización del \emph{complemento}, en este orden.

\medskip
\noindent
\textbf{\emph{FraseVerbal.}} No realizaremos un análisis individual de este caso ya que la realización del mismo dependerá de otros elementos dentro de la oración. Al verbalizar un elemento de tipo \emph{Oracion} analizaremos la \emph{FraseVerbal} que corresponde al predicado del mismo.


\medskip
\noindent
\textbf{\emph{Oracion.}} Este es el caso mas interesante para nuestro verbalizador, deberemos generar una oración correcta en base a las reglas antes vistas. 

\begin{algorithm}[H]
\caption{Realización lingüística Oracion.}
\begin{algorithmic}[1]
\Function {verbalizar}{Oracion $oracion$}
\State $sujeto\gets \text{verbalizar}(oracion.sujeto)$
\State $verbo\gets \text{determinar\_verbo}(oracion)$
\State $atributo\gets \text{determinar\_atributo}(oracion)$
\State $complemento\gets \text{verbalizar}(oracion.predicado.complemento)$
\State $estaNegada\gets oracion.predicado.negada$
\Statex
\If{$estaNegada$}
\State $resultado\gets \text{concat}(sujeto, \text{negacion}(verbo), atributo, complemento)$
\Else
\State $resultado\gets \text{concat}(sujeto, verbo, atributo, complemento)$
\EndIf
\Statex
\State \textbf{return} $resultado$
\EndFunction
\end{algorithmic}
\end{algorithm}

\noindent
A fin de contemplar las reglas de concordancia introducidas anteriormente, le prestaremos especial atención al verbo y al atributo de la frase verbal corresponde predicado de la oración, verbalizando individualmente tanto el sujeto (que probablemente sea una frase nominal) como el complemento de la frase verbal. La función \emph{determinar\_verbo} será la encargada de conjugar el verbo de manera que concuerde en número y persona con el sujeto, mientras que la función \emph{determinar\_atributo} será la encargada de determinar el atributo de forma que concuerde en número y género también con el sujeto. Si se trata de una negación (por ejemplo \emph{``... no pertenece a ...''}) deberemos negar el verbo en cuestión en la oración final. Finalmente se unirán los constituyentes en el orden mencionado previamente: \emph{sujeto - verbo - predicado}.

\include{sec/realizacion_sup}
\chapter{Implementación}
\label{cap:implementacion}

En éste capítulo introduciremos los detalles mas relevantes del desarrollo realizado para este trabajo. Comenzaremos mostrando la integración de nuestro sistema de NLG con \textit{Fastest}, viendo el modo de uso y los nuevos comandos introducidos para la generación de descripciones en lenguaje natural. Luego presentaremos los aspectos mas destacados de nuestra implementación presentes en cada una de las etapas del \emph{pipeline}.

\section{Integración con \emph{Fastest}}

La implementación realizada para este trabajo se encuentra incluida en la última versión de \emph{Fastest}, cuyo código está disponible públicamente en https://github.com/rosacris/fastest. La mayor parte del código referente a este trabajo la podremos encontrar dentro del paquete \textit{nlg}.

\subsection*{Requisitos}

%TODO referenciar tutorial / notas a algun sitio mas lindo
Para garantizar el correcto funcionamiento de Fastest y nuestro sistema de NLG, deberemos cumplir con los siguientes requerimientos:

\begin{itemize}
 \item  \emph{Fastest 1.7 o superior}: La distribución de este incluye un pequeño manual de uso.
 \item  \emph{Java SE Runtime Environment 1.6 o superior}: requerido para el correcto funcionamiento de \emph{Fastest}.
 \item  \emph{SWI\_Prolog \footnote{http://http://www.swi\_prolog.org/}}: también requerido para el correcto funcionamiento de Fastest.
 \item  \emph{FreeLing 3.1 \footnote{http://nlp.lsi.upc.edu/freeling/}\footnote{Tutorial para instalación de \emph{FreeLing} en \emph{Ubuntu}: https://gist.github.com/juliandt/}}: suite de análisis de lenguajes, necesaria para el módulo de NLG de \emph{Fastest}
\end{itemize}

\subsection{Modo de uso}

Como resultado de este trabajo introdujimos un nuevo comando en \textit{Fastest}, el comando \textbf{showdesc}. Podremos usar el mismo para generar la descripción de una o mas clases de prueba pasándole al anterior el titulo deseado para el documento junto con el o los nombres de las clases de prueba a describir, también es posible pasarle como argumento el nombre de una operación de la especificación y nuestro sistema buscará todas las clases de prueba generadas para esa operación produciendo luego las descripciones para las mismas. En la figura \ref{ej:showdesc_fastest} podemos ver la ayuda de \textit{Fastest} para el comando anterior.

\begin{figure}[H]
\begin{Verbatim}[frame=single,fontsize=\scriptsize]
showdesc [-t <title>] [<sch_name>]
	Generates natural language description for specified schemas.
\end{Verbatim}
\caption{Ayuda para el comando showdesc en \emph{Fastest}.}
\label{ej:showdesc_fastest}
\end{figure}

Veremos a continuación un ejemplo de uso del comando introducido, mediante un ejemplo similar al introducido en la sección \ref{sec:fastest} (donde ilustramos el funcionamiento general de \textit{Fastest}):

\begin{figure}[H]
\begin{Verbatim}[frame=single,fontsize=\scriptsize]
Fastest version 1.6, (C) 2013, Maximiliano Cristiá
Loading pruning rewrite rules...
Loading pruning theorems...
Fastest> loadspec symbolTable.tex
Loading specification..
Specification loaded.
Fastest> selop LookUp
Fastest> addtactic LookUp SP \in s? \in \dom st      
Fastest> genalltt
Generating test tree for 'LookUp' operation.
Fastest> showdesc -t "Descripción clase de prueba: LookUp_SP_1" LookUp_SP_1 

\documentclass{article}
\title{Descripción clase de prueba: LookUp_SP_1}

\begin{document}
\maketitle

LookUp_SP_1: Se intenta buscar un simbolo en la tabla
Cuando:
\begin{itemize}
 \item{El simbolo a buscar pertenece a los simbolos cargados en la tabla.}
 \item{El simbolo a buscar es el único elemento en los simbolos cargados en la tabla.}
\end{itemize}

\end{document}
\end{Verbatim}
\caption{Comandos para generación de descripciones en en \emph{Fastest}.}
\label{ej:comandos_fastest_nlg}
\end{figure}

\section{Documentación implementación}

En esta sección detallaremos los aspectos mas importantes de la implementación realizada. Para la misma respetamos la arquitectura y desarrollo realizado a lo largo de este trabajo, sin embargo podremos observar algunas diferencias menores, como por ejemplo: necesitamos modificar los nombres de las entidades intermedias utilizadas entre las distintas etapas del \textit{pipeline} para respetar las convenciones de estilo utilizadas en Fastest.

A continuación describiremos la función de los componentes mas importantes usados en cada etapa de nuestro sistema.

\subsection{\textit{Document Planner}}

El módulo \emph{DocumentPlanner} será el encargado de llevar a cabo las tareas de determinación de contenido y estructuración del documento detalladas en el capítulo \ref{cap:document_planning}. En la figura \ref{fig:imp_documentplanner} podemos observar los componentes mas importantes involucrados en esta etapa, cuya función describiremos a continuación. 

\begin{figure}[H]
  	\centering
	\includegraphics[scale=0.25]{img/documentplanner_imp.png}
	\caption{Diagrama clases \textit{DocumentPlanner}}
  	\label{fig:imp_documentplanner}
\end{figure}

\bigskip
\noindent
\textbf{DocumentPlanner.} Es el módulo encargado de crear un \textit{document plan} a partir de la entrada de nuestro sistema. Construye los mensajes y estructuras intermedias de éste, delegando las tareas de determinación de contenido en los módulos que describiremos a continuación.

\bigskip
\noindent
\textbf{SchFinder.} La tarea de selección detallada en el capítulo \ref{cap:determinacion_contenido} será desarrollada por este módulo. Este será el encargado de recuperar el conjunto de clases de prueba indicadas. El mismo posee una referencia al \emph{Controller} \footnote{Es el módulo encargado de mantener las referencias a elementos de la especificación, arboles de prueba, etc.} de Fastest que le permitirá identificar y recuperar las clases de prueba necesarias. 


\bigskip
\noindent
\textbf{ExprRefiner.} Es la \textit{fachada} \cite{gof} encargada de delegar los distintos procesamientos a realizar sobre las expresiones a fin de desarrollar las tareas de eliminación de tautologías y reducción de expresiones estudiadas en el capítulo \ref{cap:determinacion_contenido}. Podemos observar que utilizamos la clase \emph{Optimal} de Java 8 para modelar el hecho de que el método \textbf{refine()} podría no devolver un valor, por ejemplo en el caso que la expresión procesada sea una tautología y no deba ser incluida en el \textit{document plan}.

\bigskip
Fastest utiliza el \textit{framework} CZT \footnote{http://czt.sourceforge.net/} que integra un conjunto de herramientas para trabajar con el lenguaje de especificación Z. La forma en la que CZT modela las expresiones de Z nos resultó compleja para este trabajo, por lo que decidimos transformar las expresiones contempladas dentro del alcance de este trabajo a un modelo más simple (los que nos simplificó la implementación del algoritmo de lexicalización, por ejemplo). En la figura \ref{fig:imp_exprz} podemos observar la jerarquía de clases de las expresiones utilizadas en este trabajo para modelar las expresiones Z. 

\begin{figure}[H]
  	\centering
	\includegraphics[scale=0.31]{img/exprz_imp.png}
	\caption{Diagrama jerarquía de clases ExprZ}
  	\label{fig:imp_exprz}
\end{figure}

\bigskip
\noindent
\textbf{ASTToExprZVisitor.} Este módulo será el responsable de la transformación entre el modelo de CZT y el utilizado por este trabajo presentado en la figura \ref{fig:imp_exprz}.


\subsection{\textit{Microplanner}}

En esta sección presentaremos los componentes encargados de la tarea de \textit{microplanning} detallada en el capítulo \ref{cap:microplanning}. En la figura \ref{fig:imp_microplanner} podemos observar los módulos involucrados en la implementación de esta tarea.

\begin{figure}[H]
  	\centering
	\includegraphics[scale=0.2]{img/microplanner_imp.png}
	\caption{Diagrama clases \textit{Microplanner}}
  	\label{fig:imp_microplanner}
\end{figure}

\bigskip
\noindent
\textbf{MicroPlanner.} Es el módulo principal de esta etapa, encargado de construir la especificación del documento a partir del \textit{document plan} generado por la etapa anterior. Éste delega la tarea de lexicalización en el módulo \textit{PhraseSpecBuilder} que presentaremos a continuación.

\bigskip
\noindent
\textbf{PhraseBuilder.} Este módulo implementa la tarea de lexicalización detallada en el capítulo \ref{cap:microplanning}, la función del mismos será construir una especificación de frase a partir de una expresión Z. Para realizar esta tarea necesitaremos recorrer el modelo utilizado para las expresiones Z. Encapsulamos este trabajo en el módulo \textit{ExprZToPSVisitor}, encargado del análisis de casos para las distintas expresiones y sus posibles combinaciones de acuerdo a la definición presentada en el capítulo \ref{sec:microplanning_lexicalization}.

\bigskip
\noindent
\textbf{ExprZToPSVisitor.} Encapsula las reglas de lexicalización para todas las posibles expresiones y combinaciones de las mismas. Implementa la función auxiliar \textit{lexicalización'()} utilizada en el bosquejo de la figura \ref{fig:algoritmo_lexicalizacion}.

\bigskip
\noindent
\textbf{PhraseSpecFactory.} Construir y componer las especificaciones de frase de nuestro sistema resulta una tarea relativamente compleja, es por esto que encapsulamos esta funcionalidad en el módulo \textit{PhraseSpecFactory} encargado de construir apropiadamente las distintas especificaciones de frase de nuestro sistema. Por ejemplo: \textit{createPSElemUnico()} tomará dos especificaciones de frase y compondrá para generar una nueva especificación para la frase \textit{``... es el único elemento de ...''}, siendo las dos especificaciones mencionadas anteriormente las encargadas de modelar el texto que precede y antecede respectivamente a la anterior.

\bigskip
\noindent
\textbf{FreeLingAnalizer.} Utilizamos el analizador morfosintáctico \textit{FreeLing} para obtener las características morfológicas de los distintos constituyentes de las frases utilizadas en las designaciones. Este módulo tendrá la tarea de interactuar con las librerías provistas por la herramienta proveyéndonos de la información necesaria (como el núcleo de la frase, cual es su genero, número, etc.) para poder construir una especificación de frase a partir de una designación.

\bigskip
\noindent
\textbf{FreeLingUtils.} Contiene algunos métodos útiles para el trabajo con \textit{FreeLing}. Por ejemplo, \textit{FreeLing} genera una estructura en la que etiqueta cada palabra con anotaciones morfosintácticas, el método \textit{getInfoMorfologica()} procesa estas anotaciones produciendo una estructura mas sencilla con la que trabajará nuestro sistema.

\bigskip
\noindent
\textbf{DesignationsRepo.} Es necesario para el algoritmo de lexicalización saber si una expresión se encuentra designada y cual es su designación en el caso de estarlo. Será a través de este módulo que podrá acceder a las designaciones presentes en la especificación. El mismo se inicializa al cargar la especificación en Fastest. 

%TODO cambiar nombres a español
\subsection{\textit{Surface Realizer}}

Finalmente, el módulo \emph{SurfaceRealizer} será el encargado de generar el texto de superficie para el documento final en base a una especificación del mismo. En esta etapa se realizan las tareas de realización estructural y lingüística presentadas en el capítulo \ref{cap:realization}. En la figura \ref{fig:imp_surfrealizer} podemos ver los componentes mas relevantes involucrados en esta etapa, cuya función describiremos a continuación. 

\begin{figure}[H]
  	\centering
	\includegraphics[scale=0.17]{img/realizer_imp.png}
	\caption{Diagrama clases \textit{Surface Realizer}}
  	\label{fig:imp_surfrealizer}
\end{figure}

\bigskip
\noindent
\textbf{SurfaceRealizer.} Como mencionamos en el capítulo \ref{cap:realization} la tarea de realización estructural resulta dependiente del sistema de presentación utilizado. Es por esto que definimos la interfaz abstracta \textit{SurfaceRealizer} y una implementación especifica encargada de producir código \LaTeX~apropiado para el documento final. El método principal de esta clase toma una especificación del documento final y tiene la misión de producir el texto en lenguaje natural esperado, para esto deberá hacer uso del \textit{LinguisticRealizer} para lograr la realización lingüística de cada una de las especificaciones de frase presentes en la especificación de documento dada. 


\bigskip
\noindent
\textbf{LinguisticRealizer.} Es el módulo encargado de la realización lingüística. Es responsabilidad de este modulo asegurarse de que las frases del texto generado respeten las reglas gramaticales introducidas en el capítulo \ref{cap:linguistic_realization}. Éste hace uso de los módulos \textit{ArticleRealizer} y \textit{VerbRealizer} para resolver las cuestiones de concordancia gramatical, delegando la generación del texto final en el módulo \textit{PhraseRealizer}. 

\bigskip
\noindent
\textbf{PhraseSpecVisitor.} Utilizamos el patrón \textit{visitor} \cite{gof} para iterar sobre la estructura de las especificaciones de frase. A continuación veremos la función de las tres implementaciones mas importantes de esta interfaz. 

\bigskip
\noindent
\textbf{ArticleRealizer.} Es el módulo encargado de recorrer la estructura de una especificación de frase y escoger el articulo apropiado (de ser necesario en una frase nominal) de forma tal que concuerde de acuerdo a las reglas gramaticales introducidas previamente. El artículo será agregado sobre la misma estructura utilizada para la especificación de frase.

\bigskip
\noindent
\textbf{VerbRealizer.} Este módulo tiene la responsabilidad de conjugar el verbo de una oración (establecido en infinitivo por el \textit{microplanner}) y escoger el atributo apropiado en caso de ser un verbo copulativo, teniendo en cuenta los aspectos gramaticales detallados en el capítulo \ref{cap:linguistic_realization}. Al igual que con \textit{ArticleRealizer}, reutilizaremos la estructura utilizada para la especificación de frases para establecer el verbo correcto y atributo (reemplazando el verbo infinitivo por el verbo correctamente conjugado y el atributo previamente establecido por el atributo correspondiente).

\bigskip
\noindent
\textbf{PhraseRealizer.} Será el módulo finalmente encargado de recorrer la estructura de una especificación de frase y generar el texto final para la misma. En esta etapa casi todo el trabajo fue realizado, la función del mismo será extraer el texto contenido en la especificación de frase con el fin de generar la frase final respetando el orden establecido e introduciendo los espacios y signos de puntuación necesarios. Es importante aclarar, que resulta necesario que ejecutemos en primer instancia las tareas realizadas por \textit{PhraseSpecVisitor} y \textit{VerbRealizer} ya que este módulo asumirá que ya fueron resueltos previamente todos los aspectos gramaticales.
\include{sec/evaluacion}
\chapter{Conclusiones y trabajo a futuro}
\label{cap:conclusion}

\appendix
\chapter{Corpus de descripciones}
\label{ape:corpus}

Para construir el corpus utilizado para este trabajo utilizamos 5 especificaciones de distintos dominios de aplicación con sus respectivos casos y clases de prueba generados por Fastest. De estas clases de prueba escogimos las más interesantes para traducir acabando con un total de 98 clases de prueba.

A continuación presentaremos 3 de los ejemplos recolectados que forman parte del corpus utilizado y resultan representativos del total de ejemplos recolectados. En estos incluimos tanto la especificación utilizada para generar las clases de prueba a traducir, como también las designaciones necesarias para poder describir las clases de prueba seleccionadas y las descripciones de las mismas.

\section*{Ejemplo: \textit{Symbol Table}}

\subsection*{Especificación y designaciones}

\begin{zed}
[SYM, VAL] \also
REPORT ::= ok | symbolNotPresent
\end{zed}

\begin{itemize}
  \item Tabla de símbolos $\approx st$ \\
  \item Símbolos cargados en la tabla $\approx dom~st$ \\
  \item Valor de $x$ $\approx st~x$ 
\end{itemize}

\begin{schema}{ST}
st: SYM \pfun VAL
\end{schema}

\begin{itemize}
  \item Intenta actualizar el valor de un símbolo $\approx Update$ \\
  \item Símbolo a actualizar $\approx s?$ \\
  \item Valor a actualizar  $\approx v?$ 
\end{itemize}

\begin{schema}{Update}
\Delta ST \\
s?: SYM \\
v?: VAL \\
rep!: REPORT
\where
st' = st \oplus \{s? \mapsto v?\} \\
rep! = ok
\end{schema}

\begin{itemize}
  \item Símbolo a buscar $\approx s?$ \\
  \item Valor del símbolo buscado  $\approx v?$ 
\end{itemize}

\begin{schema}{LookUpOk}
\Xi ST \\
s?: SYM \\
v!: VAL \\
rep!: REPORT
\where
s? \in \dom st \\
v! = st~s? \\
rep! = ok
\end{schema}

\begin{schema}{LookUpE}
\Xi ST \\
s?: SYM \\
rep!: REPORT
\where
s? \notin \dom st \\
rep! = symbolNotPresent
\end{schema}

\begin{itemize}
  \item Intenta buscar el valor de un símbolo $\approx LookUp$ 
\end{itemize}

\begin{zed}
LookUp == LookUpOk \lor LookUpE
\end{zed}

\begin{itemize}
  \item Símbolo a eliminar $\approx s?$ 
\end{itemize}

\begin{schema}{DeleteOk}
\Delta ST \\
s?: SYM \\
rep!: REPORT
\where
s? \in \dom st \\
st' = \{s?\} \ndres st \\
rep! = ok
\end{schema}

\begin{itemize}
  \item Intenta eliminar un símbolo de la tabla $\approx Delete$ 
\end{itemize}

\begin{zed}
DeleteE == LookUpE \also
Delete == DeleteOk \lor DeleteE
\end{zed}

\subsection*{Clases de prueba y descripciones}

\begin{schema}{LookUp\_ SP\_ 1}\\
 st : SYM \pfun VAL \\
 s? : SYM 
\where
 s? \in \dom st \\
 \dom st = \{ s? \}
\end{schema}

\begin{tcolorbox}[colback=gray!5!white,colframe=gray!50!black,
  colbacktitle=gray!75!black,title=LookUp\_ SP\_ 1]
  Se intenta buscar el valor de un símbolo, cuando:
     \begin{itemize}
        \item[--]{El símbolo a buscar pertenece a los símbolos cargados en la tabla.}
        \item[--]{El símbolo a buscar es el único elemento de los símbolos cargados en la tabla.}
     \end{itemize}
\end{tcolorbox}
%TODO aca se podria eliminar la primera restriccion

\begin{schema}{LookUp\_ SP\_ 4}\\
 st : SYM \pfun VAL \\
 s? : SYM 
\where
 s? \notin \dom st \\
 \dom st \neq \{ s? \} \\
 s? \in \dom st
\end{schema}

\begin{tcolorbox}[colback=gray!5!white,colframe=gray!50!black,
  colbacktitle=gray!75!black,title=LookUp\_ SP\_ 4]
  Se intenta buscar el valor de un símbolo, cuando:
     \begin{itemize}
        \item[--]{El símbolo a buscar no pertenece a los símbolos cargados en la tabla.}
        \item[--]{Los símbolos cargados en la tabla no son iguales al conjunto formado por el símbolo a buscar.}
        \item[--]{El símbolo a buscar pertenece a los símbolos cargados en la tabla.}
     \end{itemize}
\end{tcolorbox}

\begin{schema}{Update\_ SP\_ 1}\\
 st : SYM \pfun VAL \\
 s? : SYM \\
 v? : VAL 
\where
 st = \{ \} \\
 \{ s? \mapsto v? \} = \{ \}
\end{schema}

\begin{tcolorbox}[colback=gray!5!white,colframe=gray!50!black,
  colbacktitle=gray!75!black,title=Update\_ SP\_ 1]
  Se intenta actualizar el valor de un símbolo, cuando:
     \begin{itemize}
        \item[--]{No hay ningún elemento en la tabla de símbolos.}
        \item[--]{No hay elementos en el conjunto formado por el par ordenado por: el símbolo a actualizar y el valor a actualizar.}
      \end{itemize}
\end{tcolorbox}


\begin{schema}{Update\_ SP\_ 2}\\
 st : SYM \pfun VAL \\
 s? : SYM \\
 v? : VAL 
\where
 st = \{ \} \\
 \{ s? \mapsto v? \} \neq \{ \}
\end{schema}

\begin{tcolorbox}[colback=gray!5!white,colframe=gray!50!black,
  colbacktitle=gray!75!black,title=Update\_ SP\_ 2]
  Se intenta actualizar el valor de un símbolo, cuando:
     \begin{itemize}
        \item[--]{No hay ningún elemento en la tabla de símbolos.}
        \item[--]{Existe al menos un elemento en el conjunto formado por el par ordenado por: el símbolo a actualizar y el valor a actualizar.}
     \end{itemize}
\end{tcolorbox}


\begin{schema}{Update\_ SP\_ 3}\\
 st : SYM \pfun VAL \\
 s? : SYM \\
 v? : VAL 
\where
 st \neq \{ \} \\
 \{ s? \mapsto v? \} = \{ \}
\end{schema}

\begin{tcolorbox}[colback=gray!5!white,colframe=gray!50!black,
  colbacktitle=gray!75!black,title=Update\_ SP\_ 3]
  Se intenta actualizar el valor de un símbolo, cuando:
     \begin{itemize}
        \item[--]{Existe al menos un elemento en la tabla de símbolos.}
        \item[--]{No hay elementos en el conjunto formado por el par ordenado por: el símbolo a actualizar y el valor a actualizar.}
     \end{itemize}
\end{tcolorbox}


\begin{schema}{Update\_ SP\_ 4}\\
 st : SYM \pfun VAL \\
 s? : SYM \\
 v? : VAL 
\where
 st \neq \{ \} \\
 \{ s? \mapsto v? \} \neq \{ \} \\
 \dom st = \dom \{ s? \mapsto v? \}
\end{schema}

\begin{tcolorbox}[colback=gray!5!white,colframe=gray!50!black,
  colbacktitle=gray!75!black,title=Update\_ SP\_ 4]
  Se intenta actualizar el valor de un símbolo, cuando:
     \begin{itemize}
        \item[--]{Existe al menos un elemento en la tabla de símbolos.}
        \item[--]{Existe al menos un elemento en el conjunto formado por el par ordenado por: el símbolo a actualizar y el valor a actualizar.}
        \item[--]{El símbolo a actualizar pertenece a los símbolos cargados en la tabla.}
     \end{itemize}
\end{tcolorbox}

\begin{schema}{Update\_ SP\_ 5}\\
 st : SYM \pfun VAL \\
 s? : SYM \\
 v? : VAL 
\where
 st \neq \{ \} \\
 \{ s? \mapsto v? \} \neq \{ \} \\
 \dom \{ s? \mapsto v? \} \subset \dom st
\end{schema}

\begin{tcolorbox}[colback=gray!5!white,colframe=gray!50!black,
  colbacktitle=gray!75!black,title=Update\_ SP\_ 5]
  Se intenta actualizar el valor de un símbolo, cuando:
     \begin{itemize}
        \item[--]{Existe al menos un elemento en la tabla de símbolos.}
        \item[--]{Existe al menos un elemento en el conjunto formado por el par ordenado por: el símbolo a actualizar y el valor a actualizar.}
        \item[--]{El símbolo a actualizar es el único elemento de los símbolos cargados en la tabla.}
     \end{itemize}
\end{tcolorbox}


\begin{schema}{Update\_ SP\_ 6}\\
 st : SYM \pfun VAL \\
 s? : SYM \\
 v? : VAL 
\where
 st \neq \{ \} \\
 \{ s? \mapsto v? \} \neq \{ \} \\
 ( \dom st \cap \dom \{ s? \mapsto v? \} ) = \{ \}
\end{schema}

\begin{tcolorbox}[colback=gray!5!white,colframe=gray!50!black,
  colbacktitle=gray!75!black,title=Update\_ SP\_ 6]
  Se intenta actualizar el valor de un símbolo, cuando:
     \begin{itemize}
        \item[--]{Existe al menos un elemento en la tabla de símbolos.}
        \item[--]{Existe al menos un elemento en el conjunto formado por el par ordenado por: el símbolo a actualizar y el valor a actualizar.}
        \item[--]{Los símbolos cargados en la tabla están incluidos en el conjunto formado por el símbolo a actualizar.}
     \end{itemize}
\end{tcolorbox}


\begin{schema}{Delete\_ SP\_ 2}\\
 st : SYM \pfun VAL \\
 s? : SYM 
\where
 s? \in \dom st \\
 st \neq \{ \} \\
 \{ s? \} = \{ \}
\end{schema}

\begin{tcolorbox}[colback=gray!5!white,colframe=gray!50!black,
  colbacktitle=gray!75!black,title=Delete\_ SP\_ 2]
  Se intenta eliminar un símbolo de la tabla, cuando:
     \begin{itemize}
        \item[--]{El símbolo a eliminar pertenece a los símbolos cargados en la tabla.}
        \item[--]{Existe al menos un elemento en la tabla de símbolos.}
        \item[--]{No hay ningún elemento en el conjunto formado por: el símbolo a eliminar.}
     \end{itemize}
\end{tcolorbox}


\begin{schema}{Delete\_ SP\_ 3}\\
 st : SYM \pfun VAL \\
 s? : SYM 
\where
 s? \in \dom st \\
 st \neq \{ \} \\
 \{ s? \} = \dom st
\end{schema}

\begin{tcolorbox}[colback=gray!5!white,colframe=gray!50!black,
  colbacktitle=gray!75!black,title=Delete\_ SP\_ 3]
  Se intenta eliminar un símbolo de la tabla, cuando:
     \begin{itemize}
        \item[--]{El símbolo a eliminar pertenece a los símbolos cargados en la tabla.}
        \item[--]{Existe al menos un elemento en la tabla de símbolos.}
        \item[--]{El símbolo a eliminar es el único elemento de los símbolos cargados en la tabla.}
     \end{itemize}
\end{tcolorbox}


\begin{schema}{Delete\_ SP\_ 4}\\
 st : SYM \pfun VAL \\
 s? : SYM 
\where
 s? \in \dom st \\
 st \neq \{ \} \\
 \{ s? \} \neq \{ \} \\
 \{ s? \} \subset \dom st
\end{schema}

\begin{tcolorbox}[colback=gray!5!white,colframe=gray!50!black,
  colbacktitle=gray!75!black,title=Delete\_ SP\_ 4]
  Se intenta eliminar un símbolo de la tabla, cuando:
     \begin{itemize}
        \item[--]{El símbolo a eliminar pertenece los símbolos cargados en la tabla.}
        \item[--]{Existe al menos un elemento en la tabla de símbolos.}
        \item[--]{Existe al menos un elemento en el conjunto formado por: el símbolo a eliminar.}
        \item[--]{El símbolo a eliminar pertenece los símbolos cargados en la tabla.}
     \end{itemize}
\end{tcolorbox}


\begin{schema}{Delete\_ SP\_ 5}\\
 st : SYM \pfun VAL \\
 s? : SYM 
\where
 s? \in \dom st \\
 st \neq \{ \} \\
 \{ s? \} \neq \{ \} \\
 \{ s? \} \cap \dom st = \{ \}
\end{schema}

\begin{tcolorbox}[colback=gray!5!white,colframe=gray!50!black,
  colbacktitle=gray!75!black,title=Delete\_ SP\_ 5]
  Se intenta eliminar un símbolo de la tabla, cuando:
     \begin{itemize}
        \item[--]{El símbolo a eliminar pertenece los símbolos cargados en la tabla.}
        \item[--]{Existe al menos un elemento en la tabla de símbolos.}
        \item[--]{Existe al menos un elemento en el conjunto formado por: el símbolo a eliminar.}
        \item[--]{El conjunto formado por el símbolo a eliminar y los símbolos cargados en la tabla no tienen ningún elemento en común.}
     \end{itemize}
\end{tcolorbox}


\begin{schema}{Delete\_ SP\_ 7}\\
 st : SYM \pfun VAL \\
 s? : SYM 
\where
 s? \in \dom st \\
 st \neq \{ \} \\
 \{ s? \} \cap \dom st \neq \{ \} \\
 \lnot \dom st \subseteq \{ s? \} \\
 \lnot \{ s? \} \subseteq \dom st
\end{schema}

\begin{tcolorbox}[colback=gray!5!white,colframe=gray!50!black,
  colbacktitle=gray!75!black,title=Delete\_ SP\_ 7]
  Se intenta eliminar un símbolo de la tabla, cuando:
     \begin{itemize}
        \item[--]{El símbolo a eliminar pertenece los símbolos cargados en la tabla.}
        \item[--]{Existe al menos un elemento en la tabla de símbolos.}
        \item[--]{El conjunto formado por el símbolo a eliminar y los símbolos cargados en la tabla tienen tienen al menos un elemento en común.}
        \item[--]{Existe al menos un elemento en los símbolos cargados en la tabla que no pertenece al conjunto formado por: el símbolo a eliminar.}
        \item[--]{El símbolo a eliminar no pertenece al los símbolos cargados en la tabla.}
     \end{itemize}
\end{tcolorbox}


\begin{schema}{Delete\_ SP\_ 12}\\
 st : SYM \pfun VAL \\
 s? : SYM 
\where
 s? \notin \dom st \\
 st \neq \{ \} \\
 \{ s? \} \neq \{ \} \\
 \{ s? \} \cap \dom st = \{ \}
\end{schema}

\begin{tcolorbox}[colback=gray!5!white,colframe=gray!50!black,
  colbacktitle=gray!75!black,title=Delete\_ SP\_ 12]
  Se intenta eliminar un símbolo de la tabla, cuando:
     \begin{itemize}
        \item[--]{El símbolo a eliminar no pertenece los símbolos cargados en la tabla.}
        \item[--]{Existe al menos un elemento en la tabla de símbolos.}
        \item[--]{Existe al menos un elemento en el conjunto formado por: el símbolo a eliminar.}
        \item[--]{El conjunto formado por el símbolo a eliminar y los símbolos cargados en la tabla no tienen ningún elemento en común.}
     \end{itemize}
\end{tcolorbox}


\begin{schema}{Delete\_ SP\_ 13}\\
 st : SYM \pfun VAL \\
 s? : SYM 
\where
 s? \notin \dom st \\
 st \neq \{ \} \\
 \{ s? \} \cap \dom st \neq \{ \} \\
 \dom st \subset \{ s? \}
\end{schema}

\begin{tcolorbox}[colback=gray!5!white,colframe=gray!50!black,
  colbacktitle=gray!75!black,title=Delete\_ SP\_ 13]
  Se intenta eliminar un símbolo de la tabla, cuando:
     \begin{itemize}
        \item[--]{El símbolo a eliminar no pertenece los símbolos cargados en la tabla.}
        \item[--]{Existe al menos un elemento en la tabla de símbolos.}
        \item[--]{El conjunto formado por el símbolo a eliminar y los símbolos cargados en la tabla tienen tienen al menos un elemento en común.}
        \item[--]{Los símbolos cargados en la tabla están incluidos en el conjunto formado por: el símbolo a eliminar.}
     \end{itemize}
\end{tcolorbox}

\section*{Ejemplo: Banco}

\subsection*{Especificación y designaciones}

\begin{zed}
[NCTA] \also

SALDO == \nat \also

MENSAJES ::= ok | numeroClienteEnUso | noPoseeSaldoSuficiente | saldoNoNulo
\end{zed}

\begin{itemize}
  \item Cajas de ahorro existentes en el banco $\approx cajas$ \\
  \item Números de cuenta cargados en el banco $\approx dom~cajas$ \\
  \item Dinero depositado en la caja de ahorro de $x$ $\approx caja~x$ 
\end{itemize}

\begin{schema}{Banco}
cajas: NCTA \pfun SALDO
\end{schema}

\begin{itemize}
  \item Número de cuenta del cliente $\approx num?$ 
\end{itemize}

\begin{schema}{DepositarOk}
\Delta Banco \\
num?: NCTA; m?: \num
\where
num? \in \dom cajas \\
m? > 0 \\
cajas' = cajas \oplus \{num? \mapsto cajas~num? + m?\}
\end{schema}

\begin{zed}
DepositarE1 == [\Xi Banco; num?:NCTA | num? \notin \dom cajas]
\end{zed}

\begin{zed}
DepositarE2 == [\Xi Banco; m?: \num | m? \leq 0]
\end{zed}

\begin{itemize}
  \item Intenta depositar dinero en una cuenta $\approx Depositar$ 
\end{itemize}

\begin{zed}
Depositar == DepositarOk \lor DepositarE1 \lor DepositarE2
\end{zed}

\begin{itemize}
  \item Número de cuenta del nuevo cliente $\approx num?$ 
\end{itemize}

\begin{schema}{NuevoClienteOk}
\Delta Banco \\
num?:NCTA \\
rep!:MENSAJES
\where
num? \notin \dom cajas \\
cajas' = cajas \cup \{num? \mapsto 0\} \\
rep! = ok
\end{schema}

\begin{schema}{NuevoClienteE}

\Xi Banco \\
num?:NCTA \\
rep!:MENSAJES
\where
num? \in \dom cajas \\
rep! = numeroClienteEnUso
\end{schema}

\begin{itemize}
  \item Intenta cargar un nuevo cliente $\approx NuevoCliente$ 
\end{itemize}

\begin{zed}
NuevoCliente == NuevoClienteOk \lor NuevoClienteE
\end{zed}

\begin{itemize}
  \item Número de cuenta del cliente $\approx num?$ 
\end{itemize}

\begin{schema}{ExtraerOk}
\Delta Banco \\
num?:NCTA \\
m?:SALDO \\
rep!:MENSAJES
\where
num? \in \dom cajas \\
0 < m? \\
m? \leq cajas~num? \\
cajas' = cajas \oplus \{num? \mapsto (cajas~num?) - m?\} \\
rep! = ok
\end{schema}

\begin{zed}
ExtraerE1 == DepositarE1 \also

ExtraerE2 == DepositarE2
\end{zed}

\begin{schema}{ExtraerE3}
\Xi Banco \\
num?:NCTA \\
m?:SALDO \\
rep!:MENSAJES
\where
m? > cajas~num? \\
num? \in \dom cajas \\
rep! = noPoseeSaldoSuficiente
\end{schema}

\begin{itemize}
  \item Intenta realizar una extracción de dinero $\approx Extraer$ 
\end{itemize}

\begin{zed}
ExtraerE == ExtraerE1 \lor ExtraerE2 \lor ExtraerE3 \also

Extraer == ExtraerOk \lor ExtraerE
\end{zed}

\begin{itemize}
  \item Número de cuenta del cliente $\approx num$ 
\end{itemize}

\begin{schema}{PedirSaldoOk}
\Xi Banco \\
num?:NCTA \\
saldo!:SALDO \\
rep!:MENSAJES
\where
num? \in \dom cajas \\
saldo! = cajas~num? \\
rep! = ok
\end{schema}

\begin{itemize}
  \item Intenta consultar el saldo de un cliente $\approx PedirSaldo$ 
\end{itemize}

\begin{zed}
PedirSaldoE == DepositarE1 \also

PedirSaldo == PedirSaldoOk \lor PedirSaldoE
\end{zed}

\subsection*{Clases de prueba y descripciones}

\begin{schema}{NuevoCliente\_ SP\_ 2}\\
 cajas : NCTA \pfun SALDO \\
 num? : NCTA 
\where
 num? \notin \dom cajas \\
 cajas = \{ \} \\
 \{ num? \mapsto 0 \} \neq \{ \}
\end{schema}

\begin{tcolorbox}[colback=gray!5!white,colframe=gray!50!black,
  colbacktitle=gray!75!black,title=NuevoCliente\_SP\_2]
  Se intenta cargar un nuevo cliente, cuando:
     \begin{itemize}
        \item[--]{El número de cuenta del nuevo cliente no pertenece a los números de cuenta cargados en el banco.}
        \item[--]{No hay ningún elemento en las cajas de ahorro existentes en el banco.}
        \item[--]{Existe al menos un elemento en el conjunto formado por el par: número de cuenta y 0.}
     \end{itemize}
\end{tcolorbox}


\begin{schema}{NuevoCliente\_ SP\_ 4}\\
 cajas : NCTA \pfun SALDO \\
 num? : NCTA 
\where
 num? \notin \dom cajas \\
 cajas \neq \{ \} \\
 \{ num? \mapsto 0 \} \neq \{ \} \\
 \dom cajas = \dom \{ num? \mapsto 0 \}
\end{schema}

\begin{tcolorbox}[colback=gray!5!white,colframe=gray!50!black,
  colbacktitle=gray!75!black,title=NuevoCliente\_SP\_4]
  Se intenta cargar un nuevo cliente, cuando:
     \begin{itemize}
        \item[--]{El número de cuenta del nuevo cliente no pertenece a los números de cuenta cargados en el banco.}
        \item[--]{Existe al menos un elemento en las cajas de ahorro existentes en el banco.}
        \item[--]{Existe al menos un elemento en el conjunto formado pope el par: número de cuenta y 0.}
        \item[--]{El número de cuenta del nuevo cliente es el único elemento de los números de cuenta cargados en el banco.}
     \end{itemize}
\end{tcolorbox}


\begin{schema}{NuevoCliente\_ SP\_ 5}\\
 cajas : NCTA \pfun SALDO \\
 num? : NCTA 
\where
 num? \notin \dom cajas \\
 cajas \neq \{ \} \\
 \{ num? \mapsto 0 \} \neq \{ \} \\
 \dom \{ num? \mapsto 0 \} \subset \dom cajas
\end{schema}

\begin{tcolorbox}[colback=gray!5!white,colframe=gray!50!black,
  colbacktitle=gray!75!black,title=NuevoCliente\_SP\_5]
  Se intenta cargar un nuevo cliente, cuando:
     \begin{itemize}
        \item[--]{El número de cuenta del nuevo cliente no pertenece a los números de cuenta cargados en el banco.}
        \item[--]{Existe al menos un elemento en las cajas de ahorro existentes en el banco.}
        \item[--]{Existe al menos un elemento en el conjunto formado pone el par: número de cuenta y 0.}
        \item[--]{El número de cuenta del nuevo cliente pertenece a los números de cuenta cargados en el banco.}
     \end{itemize}
\end{tcolorbox}


\begin{schema}{NuevoCliente\_ SP\_ 6}\\
 cajas : NCTA \pfun SALDO \\
 num? : NCTA 
\where
 num? \notin \dom cajas \\
 cajas \neq \{ \} \\
 \{ num? \mapsto 0 \} \neq \{ \} \\
 ( \dom cajas \cap \dom \{ num? \mapsto 0 \} ) = \{ \}
\end{schema}

\begin{tcolorbox}[colback=gray!5!white,colframe=gray!50!black,
  colbacktitle=gray!75!black,title=NuevoCliente\_SP\_6]
  Se intenta cargar un nuevo cliente, cuando:
     \begin{itemize}
        \item[--]{El número de cuenta del nuevo cliente no pertenece a los números de cuenta cargados en el banco.}
        \item[--]{Existe al menos un elemento en las cajas de ahorro existentes en el banco.}
        \item[--]{Existe al menos un elemento en el conjunto formado pode el par: número de cuenta y 0.}
        \item[--]{El número de cuenta del nuevo cliente no pertenece a los números de cuenta cargados en el banco.}
     \end{itemize}
\end{tcolorbox}


\begin{schema}{NuevoCliente\_ SP\_ 7}\\
 cajas : NCTA \pfun SALDO \\
 num? : NCTA 
\where
 num? \notin \dom cajas \\
 cajas \neq \{ \} \\
 \{ num? \mapsto 0 \} \neq \{ \} \\
 \dom cajas \subset \dom \{ num? \mapsto 0 \}
\end{schema}

\begin{tcolorbox}[colback=gray!5!white,colframe=gray!50!black,
  colbacktitle=gray!75!black,title=NuevoCliente\_SP\_7]
  Se intenta cargar un nuevo cliente, cuando:
     \begin{itemize}
        \item[--]{El número de cuenta del nuevo cliente no pertenece a los números de cuenta cargados en el banco.}
        \item[--]{Existe al menos un elemento en las cajas de ahorro existentes en el banco.}
        \item[--]{Los números de cuenta cargados en el banco están incluidos en el conjunto formado por el número de cuenta del nuevo cliente.}
     \end{itemize}
\end{tcolorbox}


\begin{schema}{NuevoCliente\_ SP\_ 8}\\
 cajas : NCTA \pfun SALDO \\
 num? : NCTA 
\where
 num? \notin \dom cajas \\
 cajas \neq \{ \} \\
 \{ num? \mapsto 0 \} \neq \{ \} \\
 ( \dom cajas \cap \dom \{ num? \mapsto 0 \} ) \neq \{ \} \\
 \lnot \dom \{ num? \mapsto 0 \} \subseteq \dom cajas \\
 \lnot \dom cajas \subseteq \dom \{ num? \mapsto 0 \}
\end{schema}

\begin{tcolorbox}[colback=gray!5!white,colframe=gray!50!black,
  colbacktitle=gray!75!black,title=NuevoCliente\_SP\_8]
  Se intenta cargar un nuevo cliente, cuando:
     \begin{itemize}
        \item[--]{El número de cuenta del nuevo cliente no pertenece a los números de cuenta cargados en el banco.}
        \item[--]{Existe al menos un elemento en las cajas de ahorro existentes en el banco.}
        \item[--]{Existe al menos un elemento en el conjunto formado por el par: número de cuenta y 0.}
        \item[--]{El número de cuenta del nuevo cliente no pertenece a los números de cuenta cargados en el banco.}
        \item[--]{Existe al menos un elemento en los números de cuenta cargados en el banco que no está en el conjunto formado por el número de cuenta del nuevo cliente.}
     \end{itemize}
\end{tcolorbox}


\begin{schema}{NuevoCliente\_ SP\_ 12}\\
 cajas : NCTA \pfun SALDO \\
 num? : NCTA 
\where
 num? \in \dom cajas \\
 cajas \neq \{ \} \\
 \{ num? \mapsto 0 \} \neq \{ \} \\
 \dom cajas = \dom \{ num? \mapsto 0 \}
\end{schema}


\begin{tcolorbox}[colback=gray!5!white,colframe=gray!50!black,
  colbacktitle=gray!75!black,title=NuevoCliente\_SP\_12]
  Se intenta cargar un nuevo cliente, cuando:
     \begin{itemize}
        \item[--]{El número de cuenta del nuevo cliente pertenece a los números de cuenta cargados en el banco.}
        \item[--]{Existe al menos un elemento en las cajas de ahorro existentes en el banco.}
        \item[--]{Existe al menos un elemento en el conjunto formado por el par: número de cuenta y 0.}
        \item[--]{El número de cuenta del nuevo cliente es el único elemento de los números de cuenta cargados en el banco.}
     \end{itemize}
\end{tcolorbox}


\begin{schema}{PedirSaldo\_ SP\_ 1}\\
 cajas : NCTA \pfun SALDO \\
 num? : NCTA 
\where
 num? \in \dom cajas \\
 \dom cajas = \{ num? \}
\end{schema}

\begin{tcolorbox}[colback=gray!5!white,colframe=gray!50!black,
  colbacktitle=gray!75!black,title=PedirSaldo\_SP\_1]
  Se intenta consultar el saldo de un cliente, cuando:
     \begin{itemize}
        \item[--]{El número de cuenta indicado pertenece a los números de cuenta cargados en el banco.}
        \item[--]{El número de cuenta indicado es el único elemento de los números de cuenta cargados en el banco.}
     \end{itemize}
\end{tcolorbox}


\begin{schema}{PedirSaldo\_ SP\_ 2}\\
 cajas : NCTA \pfun SALDO \\
 num? : NCTA 
\where
 num? \in \dom cajas \\
 \dom cajas \neq \{ num? \} \\
 num? \in \dom cajas
\end{schema}

\begin{tcolorbox}[colback=gray!5!white,colframe=gray!50!black,
  colbacktitle=gray!75!black,title=PedirSaldo\_SP\_2]
  Se intenta consultar el saldo de un cliente, cuando:
     \begin{itemize}
        \item[--]{El número de cuenta indicado pertenece a los números de cuenta cargados en el banco.}
        \item[--]{Los números de cuenta cargados en el banco no son iguales al conjunto formado por el número de cuenta indicado.}
        \item[--]{El número de cuenta indicado pertenece a los números de cuenta cargados en el banco.}
     \end{itemize}
\end{tcolorbox}


\begin{schema}{Depositar\_ SP\_ 3}\\
 cajas : NCTA \pfun SALDO \\
 num? : NCTA \\
 m? : \num 
\where
 num? \in \dom cajas \\
 m? > 0 \\
 cajas \neq \{ \} \\
\end{schema}

\begin{tcolorbox}[colback=gray!5!white,colframe=gray!50!black,
  colbacktitle=gray!75!black,title=Depositar\_SP\_3]
  Se intenta depositar dinero en una cuenta, cuando:
     \begin{itemize}
        \item[--]{El número de cuenta indicado pertenece a los números de cuenta cargados en el banco.}
        \item[--]{El monto a depositar es positivo.}
        \item[--]{Existe al menos un elemento en las cajas de ahorro existentes en el banco.}
     \end{itemize}
\end{tcolorbox}


\begin{schema}{Depositar\_ SP\_ 4}\\
 cajas : NCTA \pfun SALDO \\
 num? : NCTA \\
 m? : \num 
\where
 num? \in \dom cajas \\
 m? > 0 \\
 cajas \neq \{ \} \\
 \dom cajas = \dom \{ num? \mapsto ( cajas~num? + m? ) \}
\end{schema}

\begin{tcolorbox}[colback=gray!5!white,colframe=gray!50!black,
  colbacktitle=gray!75!black,title=Depositar\_SP\_4]
  Se intenta depositar dinero en una cuenta, cuando:
     \begin{itemize}
        \item[--]{El número de cuenta indicado pertenece a los números de cuenta cargados en el banco.}
        \item[--]{El monto a depositar es positivo.}
        \item[--]{Existe al menos un elemento en las cajas de ahorro existentes en el banco.}
        \item[--]{El número de cuenta indicado es el único elemento de los números de cuenta cargados en el banco.}
     \end{itemize}
\end{tcolorbox}


\begin{schema}{Depositar\_ SP\_ 5}\\
 cajas : NCTA \pfun SALDO \\
 num? : NCTA \\
 m? : \num 
\where
 num? \in \dom cajas \\
 m? > 0 \\
 cajas \neq \{ \} \\
 \dom \{ num? \mapsto ( cajas~num? + m? ) \} \subset \dom cajas
\end{schema}

\begin{tcolorbox}[colback=gray!5!white,colframe=gray!50!black,
  colbacktitle=gray!75!black,title=Depositar\_SP\_5]
  Se intenta depositar dinero en una cuenta, cuando:
     \begin{itemize}
        \item[--]{El número de cuenta indicado pertenece a los números de cuenta cargados en el banco.}
        \item[--]{El monto a depositar es positivo.}
        \item[--]{Existe al menos un elemento en las cajas de ahorro existentes en el banco.}
        \item[--]{El número de cuenta indicado pertenece a los números de cuenta cargados en el banco.}
     \end{itemize}
\end{tcolorbox}


\begin{schema}{Depositar\_ SP\_ 6}\\
 cajas : NCTA \pfun SALDO \\
 num? : NCTA \\
 m? : \num 
\where
 num? \in \dom cajas \\
 m? > 0 \\
 cajas \neq \{ \} \\
 ( \dom cajas \cap \dom \{ num? \mapsto ( cajas~num? + m? ) \} ) = \{ \}
\end{schema}

\begin{tcolorbox}[colback=gray!5!white,colframe=gray!50!black,
  colbacktitle=gray!75!black,title=Depositar\_SP\_6]
  Se intenta depositar dinero en una cuenta, cuando:
     \begin{itemize}
        \item[--]{El número de cuenta indicado pertenece a los números de cuenta cargados en el banco.}
        \item[--]{El monto a depositar es positivo.}
        \item[--]{Existe al menos un elemento en las cajas de ahorro existentes en el banco.}
        \item[--]{El número de cuenta indicado no pertenece a los números de cuenta cargados en el banco.}
     \end{itemize}
\end{tcolorbox}


\begin{schema}{Depositar\_ SP\_ 7}\\
 cajas : NCTA \pfun SALDO \\
 num? : NCTA \\
 m? : \num 
\where
 num? \in \dom cajas \\
 m? > 0 \\
 cajas \neq \{ \} \\
 \dom cajas \subset \dom \{ num? \mapsto ( cajas~num? + m? ) \}
\end{schema}

\begin{tcolorbox}[colback=gray!5!white,colframe=gray!50!black,
  colbacktitle=gray!75!black,title=Depositar\_SP\_7]
  Se intenta depositar dinero en una cuenta, cuando:
     \begin{itemize}
        \item[--]{El número de cuenta indicado pertenece a los números de cuenta cargados en el banco.}
        \item[--]{El monto a depositar es positivo.}
        \item[--]{Existe al menos un elemento en las cajas de ahorro existentes en el banco.}
        \item[--]{Los números de cuenta cargados en el banco están incluidos en el conjunto formado por el número de cuenta indicado.}
     \end{itemize}
\end{tcolorbox}


\begin{schema}{Depositar\_ SP\_ 8}\\
 cajas : NCTA \pfun SALDO \\
 num? : NCTA \\
 m? : \num 
\where
 num? \in \dom cajas \\
 m? > 0 \\
 cajas \neq \{ \} \\
 ( \dom cajas \cap \dom \{ num? \mapsto ( cajas~num? + m? ) \} ) \neq \{ \} \\
 \lnot \dom \{ num? \mapsto ( cajas~num? + m? ) \} \subseteq \dom cajas \\
 \lnot \dom cajas \subseteq \dom \{ num? \mapsto ( cajas~num? + m? ) \}
\end{schema}

\begin{tcolorbox}[colback=gray!5!white,colframe=gray!50!black,
  colbacktitle=gray!75!black,title=Depositar\_SP\_8]
  Se intenta depositar dinero en una cuenta, cuando:
     \begin{itemize}
        \item[--]{El número de cuenta indicado pertenece a los números de cuenta cargados en el banco.}
        \item[--]{El monto a depositar es positivo.}
        \item[--]{Existe al menos un elemento en las cajas de ahorro existentes en el banco.}
        \item[--]{El número de cuenta indicado no pertenece a los números de cuenta cargados en el banco}
        \item[--]{Existe al menos un elemento en los números de cuenta cargados en el banco que no está en el conjunto formado por el número de cuenta indicado.}
     \end{itemize}
\end{tcolorbox}

\section*{Ejemplo: Sistema de sensores}

\subsection*{Especificación y designaciones}

\begin{itemize}
  \item $x$ es un identificador de sensor válido $\approx x \in SENSOR$
\end{itemize}

\begin{zed}
[SENSOR]
\end{zed}

\begin{itemize}
  \item Conjunto de identificadores válidos $\approx dom~smax$ \\
  \item $x$ es un identificador válido $\approx x \in dom~smax$ \\
  \item Valor máximo registrado para $x$ $\approx smax~x$ 
\end{itemize}

\begin{zed}
MaxReadings == [smax: SENSOR \pfun \num]
\end{zed}

\begin{itemize}
  \item Identificador del sensor leído $\approx s?$ \\
  \item Valor de medición leído $\approx r?$ 
\end{itemize}

\begin{schema}{KeepMaxReadingOk}
\Delta MaxReadings \\
s?:SENSOR; r?:\num \\
\where
s? \in \dom smax \\
smax~s? < r?\\
smax' = smax \oplus \{s? \mapsto r?\}
%smax' = (\dom \{s? \mapsto r?\} \ndres smax) \oplus \{s? \mapsto r?\}
\end{schema}

\begin{zed}
KeepMaxReadingE1 == [\Xi MaxReadings; s?:SENSOR | s? \notin \dom smax]
\end{zed}

\begin{schema}{KeepMaxReadingE2}
\Xi MaxReadings \\
s?:SENSOR; r?:\num
\where
s? \in \dom smax\\
r? \leq smax~s? 
\end{schema}

\begin{itemize}
  \item Intenta actualizar un valor máximo sensado $\approx KeepMaxReading$
\end{itemize}

\begin{zed}
KeepMaxReading == KeepMaxReadingOk \lor KeepMaxReadingE1 \lor KeepMaxReadingE2
\end{zed}

\subsection*{Clases de prueba y descripciones}

\begin{schema}{KeepMaxReading\_ SP\_ 3}\\
  smax : SENSOR \pfun \num \\
  s? : SENSOR \\
  r? : \num
\where
  s? \in \dom smax \\
  smax~s? < r? \\
  smax~s? < 0 \\
  r? > 0
\end{schema}

\begin{tcolorbox}[colback=gray!5!white,colframe=gray!50!black,
  colbacktitle=gray!75!black,title=KeepMaxReading\_SP\_3]
  Se intenta actualizar un valor máximo sensado, cuando:
     \begin{itemize}
  	    \item[--]{El identificador del sensor leído pertenece al conjunto de identificadores válidos.}
  	    \item[--]{El valor máximo registrado para el identificador del sensor leído es menor a el valor de medición leído.}
	      \item[--]{El valor máximo registrado para el identificador del sensor leído es negativo.}
	      \item[--]{El valor de medición leído es positivo.}
     \end{itemize}
\end{tcolorbox}

\begin{schema}{KeepMaxReading\_ SP\_ 4}\\
  smax : SENSOR \pfun \num \\
  s? : SENSOR \\
  r? : \num
\where
  s? \in \dom smax \\
  smax~s? < r? \\
  smax~s? = 0 \\
  r? > 0
\end{schema}

\begin{tcolorbox}[colback=gray!5!white,colframe=gray!50!black,
  colbacktitle=gray!75!black,title=KeepMaxReading\_SP\_4]
  Se intenta actualizar un valor máximo sensado, cuando:
     \begin{itemize}
        \item[--]{El identificador del sensor leído pertenece al conjunto de identificadores válidos.}
        \item[--]{El valor máximo registrado para el identificador del sensor leído es menor a el valor de medición leído.}
        \item[--]{El valor máximo registrado para el identificador del sensor leído es igual a cero.}
        \item[--]{El valor de medición leído es positivo.}
     \end{itemize}
\end{tcolorbox}


\begin{schema}{KeepMaxReading\_ SP\_ 7}\\
  smax : SENSOR \pfun \num \\
  s? : SENSOR \\
  r? : \num
\where
  s? \notin \dom smax \\
  smax~s? < 0 \\
  r? = 0
\end{schema}

\begin{tcolorbox}[colback=gray!5!white,colframe=gray!50!black,
  colbacktitle=gray!75!black,title=KeepMaxReading\_SP\_7]
  Se intenta actualizar un valor máximo sensado, cuando:
     \begin{itemize}
        \item[--]{El identificador del sensor leído no pertenece al conjunto de identificadores válidos.}
        \item[--]{El valor máximo registrado para el identificador del sensor leído es negativo.}
        \item[--]{El valor de medición leído es igual a cero.}
     \end{itemize}
\end{tcolorbox}

\begin{schema}{KeepMaxReading\_ SP\_ 14}\\
  smax : SENSOR \pfun \num \\
  s? : SENSOR \\
  r? : \num
\where
  s? \in \dom smax \\
  r? \leq smax~s? \\
  smax~s? = 0 \\
  r? > 0
\end{schema}

\begin{tcolorbox}[colback=gray!5!white,colframe=gray!50!black,
  colbacktitle=gray!75!black,title=KeepMaxReading\_SP\_14]
  Se intenta actualizar un valor máximo sensado, cuando:
     \begin{itemize}
        \item[--]{El identificador del sensor leído pertenece al conjunto de identificadores válidos.} 
        \item[--]{El valor de medición leído es menor o igual a el valor máximo registrado para el identificador del sensor leído.}
        \item[--]{El valor máximo registrado para el identificador del sensor leído es positivo.}
        \item[--]{El valor de medición leído es positivo.}
     \end{itemize}
\end{tcolorbox}
\chapter{Guía de estilo para designaciones}
\label{ape:designaciones}

Las designaciones son la principal fuente de conocimiento del dominio. Éstas son fundamentales para que el sistema de NLG pueda generar descripciones independientes del dominio de aplicación. A continuación, se enumera una serie de pautas que el ingeniero realizando la especificación debe tener en cuenta para que el sistema produzca textos lo más fluidos y naturales posibles.

\bigskip
A fin de un correcto funcionamiento de el sistema, resulta indispensable designar los siguientes elementos de una especificación:

\bigskip
\begin{enumerate}
	\item Nombres de esquemas de operación totales
	\item Variables de entrada/salida
	\item Variables de estado
	\item Tipos básicos
	\item Nombres de constructores de tipos libres
\end{enumerate}

\bigskip
Cuando el objeto a designar sea una función es recomendable designar:

\bigskip
\begin{enumerate}
	\item La función (f)
	\item El dominio de la función (dom f)
	\item La aplicación de la función (f x)
	\item El rango de la función (ran f)
\end{enumerate}

\bigskip
Como mencionamos en la sección \ref{cap:designaciones}, la función que cumplen las designaciones en este trabajo difiere un poco de la propuesta por Jackson~\cite{jackson}. Jackson propone escribir la menor cantidad de designaciones posibles y definir otros términos en base a las éstas. Para nosotros, las designaciones resultan la principal fuente de conocimiento para el sistema de NLG y en algunos casos (como los enumerados anteriormente para designar una función) designar los ítems propuestos le permitirá a nuestro sistema la generación de descripciones de mayor calidad. De no hacerlo, si por ejemplo escribiéramos sólo la designación para $f$, nuestro sistema se vería obligado a verbalizar $\dom f$ de la siguiente manera:

\begin{figure}[H]
\center
$\texttt{verb'}(\dom f) \rightarrow \text{\emph{``el dominio''}} + \texttt{verb}(f)$
\end{figure}

Algo similar ocurriría para los casos en los que se quiera verbalizar $\ran f$ y $f~x$. Esto introduciría términos como ``dominio'' o ``rango'' y ``aplicación'', que refieren a elementos del modelo y no a elementos del dominio de aplicación del sistema a testear, generando de esta forma descripciones poco naturales.

\chapter{Guía de notación}
\label{ape:notacion}

A continuación describiremos brevemente la notación utilizada en los diagramas utilizados a lo largo de este trabajo para modelar estructuras de clase y relaciones estáticas entre éstas.

\section{Diagrama de clases}

En la figura \ref{fig:png_arquitectura} podemos observar la notación utilizada para modelar tanto clases concretas como abstractas. Una clase se denota por un rectángulo con el nombre de la clase en la parte superior. En el caso de tratarse de una clase abstracta encontraremos el nombre en cursiva y la flecha que une a la misma con la clase concreta que la implementa estará graficada mediante una línea punteada. Por otro lado, la línea que une a una subclase con la clase padre de la cual hereda estará dibujada utilizando una línea continua. Abajo del nombre de la clase pueden aparecer las operaciones o métodos de la misma y también las variables de clase que pudiese tener; para diferenciar entre estos, antecederemos el signo $+$ al nombre de las operaciones, mientras que utilizaremos un signo $-$ para indicar las variables de clase. La información del tipo de datos de retorno de una operación así como el de los argumentos del mismo y de las variables de instancia se encuentra especificado en cada uno de los casos. 

\begin{figure}[H]
  	\centering
	\includegraphics[scale=0.17]{img/ref_clases.png}
	\caption{Diagramas de clase}
  	\label{fig:png_arquitectura}
\end{figure}

En la figura \ref{fig:png_arquitectura} podemos ver la notación usada para indicar distintas referencias entre clases. Utilizamos una línea con un rombo blanco en el otro extremo para indicar agregación (en este caso \textbf{ClaseConcreta} tiene una referencia a \textbf{ClaseA} y este último puede existir de forma independiente de A). Por otro lado, utilizamos un rombo negro para indicar composición (en este caso caso \textbf{ClaseConcreta} tiene una referencia a \textbf{ClaseC} y este último no puede existir por separado). En ambos casos, es posible indicar la cardinalidad de estas relaciones como se muestra en la figura. Finalmente, en algunos diagramas del trabajo se puede ver una flecha con línea de puntos y el texto \emph{uses}, esto indica que una clase hace uso de una o más operaciones de la otra, se diferencia de la composición y agregación ya que no hace referencia a ningún objeto. Lo utilizamos, por ejemplo, para diferenciar las llamadas a un método estático de una librería.

\begin{figure}[H]
  	\centering
	\includegraphics[scale=0.17]{img/ref_comp-agregation.png}
	\caption{Diagramas de relaciones entre clases}
  	\label{fig:png_arquitectura}
\end{figure}




\bibliographystyle{alpha}
%\nocite{*}
\bibliography{biblio}

\end{document} 
